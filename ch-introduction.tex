\chapter{INTRODUCTION}
\label{sec:org2924b89}

Perovskites have historically been materials of great interest for a variety of optoelectronic applications with special interest in the past ten years (see figure \ref{fig:nrel}) in their potential as photovoltaic absorbers.
As absorbers for solar cells, they offer opportunities to reduce cost and environmental impact as well as increase performance.
\autocite{ansari-2018-front-oppor,yin-2015-halid-perov,manser-2016-intrig-optoel,brenner-2016-hybrid-organ}
A cubic phase perovskite unit cell with general formula \(\ce{ABX3}\) contains two cations A and B at the corners and body center, and an anion X at each of the face centers.
The symbolic 3D perovskite structure is a network of \(\ce{BX6}\) octahedra robustly held together by large A-site cations.
This unique structure means that perovskite properties are highly tunable by changing the size and number of A/B/X species, by manipulating relative octahedral arrangements, and by creating non-cubic and metastable phases.
Halide perovskites (HaP), as opposed to the oxide perovskites that have been well researched over the past century are so characterized because their X-site anions are halogens.
Their B-site cations may be divalent elements, and the A-site is occupied by large monovalent cations that are either inorganic elements or organic molecules.

The most commonly studied hybrid organic-inorganic HaPs, \(\ce{MAPbI3}\) and \(\ce{FAPbI3}\), have demonstrated large power conversion efficiency (PCE) values between 20\% and 25\% when used as absorbers in single- or multi-junction solar cells.
\autocite{cui-2019-planar-p,jeong-2020-stabl-perov}
This is a five-fold improvement over the efficiencies of the same compositions first reported in 2009 and demonstrates the most attractive feature of HaPs, their unique tunability.
A theoretical cubic perovskite structure is considered stable if the ionic radii of A, B, and X-site species satisfy the well-known tolerance (\(t\)) and octahedral (\(o\)) factors.
\autocite{bartel-2019-new-toler}
Even accounting for stability constraints, the chemical space of perovskites experiences combinatorial scaling with the number of candidate elements which could be incorporated at each site.
This poses a multidimensional optimization problem for which determining the optimal atomic fractions for a particular performance target requires acceleration.
My research focused on accelerating this search through the development and application of data-driven design methods in the composition space of halide perovskites.
The most recent work presented in this dissertation is being submitted for publication as \fullcite{manganaris-2023-multi-fidel}. 
Publications from earlier work leading to this dissertation include
\fullcite{yang-2023-high-throug},
and \fullcite{manganaris-2022-mrs-comput}.

 
\begin{figure}[htbp]
\centering
\includesvg[width=400pt]{./.ob-jupyter/f34ddd159d9c661cb7098eb9bfd43dcfa5ab8286}
\caption{\label{fig:nrel} Rapid rise in cumulative maximum of HaP PCEs}
\end{figure}

\section{Design Goals in Perovskite Absorbers}
\label{sec:orgbd6587f}
Perovskite properties may be tuned in various ways.
The introduction of dopants and defects \autocite{kim-2020-upper-limit,dahliah-2021-high-throug} and the mutation of their cubic structure \autocite{kar-2018-comput-screen,kim-2017-hybrid-organ} are each promising areas of design.
However, the work presented here focuses specifically on the identification of a reasonable number of candidate compositions for future laboratory trials.
The most promising HaP compositions for PV absorption explored to date usually contain a mix of MA, FA, and Cs at the A-site, primarily Pb at the B-site with minor fractions of other divalent cations such as Sn and Ge, and I or Br at the X-site often with little Cl.
Discovery of novel HaP compositions with attractive properties is on the rise as researchers expand the search into more complex alloys, novel A-site organic molecules, and substitutes for Pb at the B-site from Group IV, Group II, or transition elements.
\autocite{zhu-2019-struc-elect,banerjee-2019-rashb-trigg,ding-2019-cesium-decreas,greenland-2020-correl-phase}
Mixing at the A-site has been shown to improve formability \autocite{zhang-2019-perov-photov}, while B-site and X-site mixing can tune and optimize band gap and therefore the maximum optical wavelength absorbable by the semiconductor.
Band gap is the energy required to promote an electron from the valence band to the conduction band.
Absorption of visible light from the plentiful green to red portion of the spectrum, by the simple relation \(E_{bg} = E_\gamma = \frac{hc}{\lambda}\), corresponds to band gaps of 1-2 \unit{\electronvolt}.
The allure of A/B/X-site mixing, even the creation of high entropy perovskite alloys, is in the promise to obtain dramatically different properties than those of pure compositions.
Perovskite properties have demonstrated highly nonlinear responses to changes in composition.
It is hoped that exploiting this could lead to possibly eliminating toxic lead, reducing degradation under light exposure, and even improving resistance to adverse environmental conditions while also targeting specific optoelectronic performance markers.

\begin{figure}[htbp]
\centering
\includegraphics[width=250pt]{hybrid-HaP.png}
\caption{\(2\times{}2\times{}2\) \(\alpha\)-phase supercell with Methylammonium at the A-site}
\end{figure}

The chemical design space of HaPs is intractably large to effectively screen by physical laboratory methods.
The halide perovskite chemical space covered by this dataset was based on fourteen species commonly appearing in study of these materials.
The five constituents making up the A-site occupants include three inorganic and two organic cations, namely \(\ce{CH3NH3+}\) Methylammonium (MA) and \(\ce{CH(NH2)2+}\) Formamidinium (FA).
\autocite{yan-2016-defec-physic,dimesso-2016-inves-formam}
Six divalent metals represent the possible B-site occupants and three halogen anions make up the possible X-site occupants.
See table \ref{tbl:site_tbl}.
The total number of distinct compositions possible in a \(2\times{}2\times{}2\) supercell is over 207 million.
Of these compositions, 37695 contain mixing at only one site and ninety are pure having no mixing at any site.
I refer to the combination of these subsets as the "cardinal mixing set" and it equally represents each of the constituent species of interest.
See figures \ref{fig:domainstats} and \ref{fig:domainmix}.

First principles density functional theory (DFT) simulations have been systematically performed to study the optoelectronic properties of HaPs as a function of structure, composition, and defects.
Recently, DFT simulations have been reliably used to predict structural information, band gaps, optical absorption spectra, and defect formation energies of a variety of HaPs with reasonable accuracy.
\autocite{mannodi-kanakkithodi-2022-data-driven,yin-2015-halid-perov}
An examination of the HaP-related computational literature reveals that there have been numerous medium (\textasciitilde{}10\textsuperscript{2} data points) to large (\textasciitilde{}10\textsuperscript{3} or more data points) DFT datasets reported for HaPs.
\autocite{castelli-2014-bandg-calcul,park-2019-explor-new,kar-2018-comput-screen,pu-2021-screen-perov}
These have been successfully screened to identify promising materials with desired stability and formability as well as PV-suitable band gaps, among other properties.

A clear limitation of High-Throughput (HT) DFT driven screening is the computational expense of applying a suitably advanced level of theory across a very large number of materials.
This problem is typically addressed by coupling DFT computations with machine learning (ML) techniques.
Within the area of perovskites, there are many examples in the literature where DFT datasets and suitable atomic/structural/compositional descriptors have been used to train a variety ML-based predictive and classification models, leading to accelerated prediction of lattice constants, formation energies, band gaps, and other important properties.
\autocite{park-2019-explor-new,stanley-2020-machin-learn,lee-2021-discov-lead}
Such DFT-ML models, once rigorously trained and tested, are deployed for high-throughput screening across massive sample spaces of unknown perovskites.
\autocite{yang-2022-high-throug}

\begin{table}[htbp]
\caption{\label{tbl:site_tbl} \(\ce{ABX3}\) candidate species per site}
\centering
\begin{tabular}{l|llllll}
A-site & MA & FA & Cs & Rb & K & \\[0pt]
B-site & Pb & Sn & Ge & Ba & Sr & Ca\\[0pt]
X-site & I & Br & Cl &  &  & \\[0pt]
\end{tabular}
\end{table}

 
\begin{figure}[htbp]
\centering
\includesvg[inkscapeformat=png, inkscapedpi=300,width=300pt]{./.ob-jupyter/835ba724468229103da0fd534aadf2acc7bb610c}
\caption{\label{fig:domainstats} The cardinal mixing sample space contains equal fractions of each element}
\end{figure}

 
\begin{figure}[htbp]
\centering
\includesvg[inkscapeformat=png, inkscapedpi=300,width=300pt]{./.ob-jupyter/9f2a6b3e88bd87edad16a8293ec22e2f0ce74780}
\caption{\label{fig:domainmix} The cardinal mixing sample space contains mostly B-site mixed compounds}
\end{figure}

\section{Multi-fidelity Dataset}
\label{sec:orgc33343b}
For my work I used a large DFT dataset collected over the past three years by my advisor and fellow student Jiaqi Yang.
This dataset consists of approximately 1300 calculations based on approximately 500 chemically distinct, pseudo-cubic, halide perovskite alloys reported in the Mannodi research group's prior work.
\autocite{mannodi-kanakkithodi-2022-data-driven,yang-2023-high-throug}
There are more calculations than there are distinct compositions because each composition is simulated multiple times to obtain results of varying fidelity.
Also, it is supplemented by an additional \textasciitilde{}100 points of data aggregated from reputable sources by \textcite{almora-2020-devic-perfor}.
In total it is one of the largest first principles cubic halide perovskite datasets and represents years of work.
It provided an excellent foundation for my work.

The relatively large size of this dataset samples the space of all possible single-site mixed compositions with good coverage.
This enables the training of interpolative models in the HaP composition space promising lower and less frequent error than if lesser coverage were used.
In this dataset, all perovskite structures are cubic or pseudo-cubic, which aids my focus on investigating effects of composition and alloying on photovoltaic performance.

\section{Perovskite Formability}
\label{sec:org0cdae8d}
In order to simplify the search for viable candidates, some standard empirical rating of perovskite formability are employed.
It has been observed that the A-site member must be much larger than its counterpart at the B-site for a Perovskite to be stable.
B-site elements are usually large (e.g. Pb, Sn) in Halide Perovskites, which incidentally motivates the use of organic cations at A.
\autocite{kieslich-2015-exten-toler}
Various tolerance factors have been proposed in the literature to describe this constraint.
Each tolerance factor is expressed as a function of the ionic radii of the species at each site \(r_A\), \(r_B\), and \(r_X\).
The Goldschmidt tolerance is defined as expression \ref{eq:t}.
\autocite{yin-2015-halid-perov}
The octahedral Factor is defined as the simple ratio \ref{eq:o}.
A recent tolerance factor proposed by \textcite{bartel-2019-new-toler} is defined as expression \ref{eq:b}.

\begin{equation}
\label{eq:t}
t = \frac{r_A+r_X}{\sqrt{2}*(r_B+r_X)}
\end{equation}

\begin{equation}
\label{eq:o}
o=\frac{r_B}{r_X}
\end{equation}

\begin{equation}
\label{eq:b}
b = \frac{r_X}{r_B}-\left[ 1-\frac{\frac{r_A}{r_B}}{\ln(\frac{r_A}{r_B})} \right]
\end{equation}

The approximate definitions \(t \approx 1\), \(o \approx 0.67\), and \(b \underset{\sim}{<} 4\) quantify perovskite \(\alpha\)-phase stability.
\autocite{yin-2015-halid-perov,bartel-2019-new-toler}
These criteria are useful for efficiently evaluating if a proposed perovskite can be formed.
They supplemented the cuts on band gap we developed for identifying high-performing Perovskite absorbers.

\section{Thesis Overview}
\label{sec:org3722640}
In chapter \ref{sec:org44ac04c}, I review the work we did in \citetitle{yang-2023-high-throug}\autocite{yang-2023-high-throug} to create a multi-fidelity dataset of perovskite band gaps.
I then detail the methods I used for predicting band gaps and the models I obtained in chapter \ref{sec:org37dcca4}.
Results and a discussion of their significance is given in chapter \ref{sec:orgbfd2f81}.
The open-source tools I developed for this work have been made available to the broader materials science community.
The essential qualities of these tools have been detailed in Appendix \ref{sec:orgc7f868a}.

\chapter{DENSITY FUNCTIONAL THEORY SIMULATION}
\label{sec:org44ac04c}
All material properties are fundamentally a function of the motion of electrons and atomic nuclei.
These fundamental building blocks of matter can be modeled as point particles in a many bodied system governed by an equation of motion.
Due to the quantum nature of these particles, position and momentum are defined in terms of a complex-valued wave function that describes the probability of finding a particle at a point in space.
Therefore, the equation of motion is the Schrödinger equation \eqref{eq:scheq}, the solution of which yields the energy of a single configuration of particles.
For electronic and optical properties, the electron configuration is most significant, allowing for some simplification by using the Born-Oppenheimer approximation (BOA).

\begin{equation}
\label{eq:scheq}
\imath\hbar\frac{\partial}{\partial t}\Psi(R_{3N}, r_{3n}, t) = \mathcal{H}(R_{3N}, r_{3n}, P_{3N}, p_{3n})\Psi(R_{3N}, r_{3n}, t) \stackrel{BOA}{\implies} \hat{H}\psi(r_{3n}) = E\psi(r_{3n})
\end{equation}

The solution to the Schrödinger equation is unfortunately intractable for even tens of electrons due to the huge expense of integrating the \(3n\) dimensional electron wave function \(\psi(r_{3n})\).
Density functional theory is the leading method for tractably computing the energy of many electron interactions in quantum systems.
DFT is founded on the Hohenberg Kohn relation proving an electron configuration fully determines the potential energy due to the nuclear configuration.
To illustrate, the electron density at a position is the expectation value of the position of all electrons not at that position.
This density emerges when computing the expectation value of the potential energy of a single electron \(r_k\)

\begin{equation}
\label{eq:V_ext}
\braket{\psi|\hat{V}_{nuc}(r_k)|\psi} = \int \di^3r_k \hat{V}_{nuc}(r_k)\int \Pi_{i=1}^{k-1}\di^3r_i\Pi_{i=k+1}^{n}\di^3r_i|\psi({r_i})|^2 = \int \hat{V}_{nuc}(r_k)n(r_k)\di^3r = \hat{V}(n)
\end{equation}

due to the nuclei.
This expression is reversible, showing that for a given density there must be a unique \(\psi(n(r))\) expressed as a functional of the density.
With the establishment of this density functional, any property of interest may be found entirely in terms of the density.
The major benefit of this re-framing is that the density functional lives in 3-D space, whereas the wave function lives in 3n-D space.
By circumventing the wave function, the many-electron problem solving for the configurational energy

\begin{equation}
\label{eq:funco}
E = \braket{\psi|\hat{H}|\psi} = \braket{\psi(n)|\hat{T}(n) + \hat{U}(n) + \hat{V}(n)|\psi(n)} \implies E(n) = \hat{T}(n) + \hat{U}(n) + \hat{V}(n)
\end{equation}

can be effectively cast to a sum of single-electron problems with theoretically no compromise on accuracy.
Minimizing \(E(n(r))\) yields the ground state energy and ground state electron configuration.
The electron-electron interaction \(\hat{U}(n(r))\) lacks a perfect solution and requires approximation.
The system potential energy \(\hat{V}(n(r))\), as discussed, is uniquely defined in 3D space by the nuclei in the system.
The kinetic energy operator \(\hat{T}(n(r))\) in combination with the system energy is used to derive the Kohn-Sham equations \eqref{eq:kseq}.

\begin{equation}
\label{eq:kseq}
\left(-\frac{\hbar^2}{2m}\nabla+\hat{V}_s\right)\psi(r_k) = \epsilon_k\psi(r_k)
\end{equation}

The solution for these single particle wave functions yield the electron density \(n(r) = \sum_{i=1}^n|\psi_i(r)|^2\) necessary to solve \eqref{eq:funco}.
The single-particle potential \(\hat{V}_s\) is the sum of the system energy, the electron-electron Coulomb repulsion, and the so called "exchange correlation."
This latter functional is what makes a solution for \(\hat{U}\) so elusive.
The more physics accounted for in the approximations used for exchange correlation, the better the accuracy but the greater the cost.
So these approximations are ranked by "level of theory."
My work is necessary because perfect simulations do not exist due to the conceptual difficulty in efficiently modeling the exchange correlation.
For the remained of this chapter I present how DFT was used to create the multi-fidelity dataset that enabled my work.

All DFT computations were performed using VASP version 6.2
\autocite{kresse-1996-effic-iterat,kresse-1993-ab-initio}
employing projector-augmented-wave (PAW) pseudo-potentials.
\autocite{kresse-1996-effic-ab,kresse-1999-from-ultras,kresse-1994-norm-conser}
Multiple levels of theory (LoT) were used in most computations.
Each simulation was conducted on the same set of compositions allowing \emph{at most} single-site mixing of our 14 constituent candidates for 3 sites (table \ref{tbl:site_tbl}).
Each HaP composition was simulated in a \(2\times{}2\times{}2\) supercell, which allowed A and B-site mixing to be performed in discrete 1/8\textsuperscript{th} fractions of the total site occupancy, and X-site mixing in 1/24\textsuperscript{th} fractions, though for simplicity, we restricted X-site mixing to fractions of 3/24.
For simulating mixed perovskites, the special quasi-random structures (SQS) method was applied to build periodic structures that made the first nearest-neighbor shells as similar to the target random alloy as possible.
\autocite{jiang-2016-special-quasir}
The final tally of successfully converged calculations is listed in table \ref{tbl:LoTs}.
The Perdew-Burke-Ernzerhof (PBE) functional within the generalized gradient approximation (GGA) as well as the hybrid Heyd-Scuseria-Ernzerhof functional with parameters (\(\alpha=0.25\)) and (\(\omega=0.2\)) (HSE06) are used for exchange-correlation energy.
\autocite{perdew-1996-gener-gradien,heyd-2003-hybrid-funct}
The energy cutoff for the plane-wave basis was set to 500 \unit{\electronvolt}.
For all PBE geometry optimization calculations, the Brillouin zone was sampled using a \(6\times{}6\times{}6\) Monkhorst-Pack mesh for unit cells and a \(3\times{}3\times{}3\) for supercells.
Using the PBE optimized structure as input, the electronic band structure was calculated along high-symmetry k-points to obtain accurate band gaps, and the optical absorption spectrum is further calculated using the LOPTICS tag, setting the number of energy bands to 1000 for each structure.
\autocite{hinuma-2016-band-struc,ganose-2018-sumo}
For HSE calculations, geometry optimization was performed using only the Gamma point, and subsequent computations used a reduced \(2\times{}2\times{}2\) Monkhorst-Pack mesh.
The force convergence threshold is set to be -0.05 eV/Å.
Spin-orbit coupling (SOC) is also applied to two flavors of HSE computations using the LORBIT tag and the non-collinear magnetic version of VASP 6.2.
\autocite{steiner-2016-calcul-magnet}
Optical absorption spectra from different HSE functionals were obtained by using the difference between the respective PBE and HSE band gap, and shifting the PBE-computed spectrum.

 
\begin{table}[htbp]
\caption{\label{tbl:LoTs} Sample counts by density functional represented in dataset}
\centering
\begin{tabular}{lr}
 & LoT\\[0pt]
\hline
PBE & 492\\[0pt]
HSE & 297\\[0pt]
HSE(SOC) & 282\\[0pt]
HSE-PBErel(SOC) & 244\\[0pt]
EXP & 90\\[0pt]
\hline
 & 1405\\[0pt]
\end{tabular}
\end{table}

\section{DFT Computed Band Gaps}
\label{sec:org7daa035}
Four types of electronic band gaps were computed by my advisor and group members using a \(2\times{}2\times{}2\) Monkhorst-Pack mesh.
These four measures E\textsubscript{bg}\textsuperscript{PBE}, E\textsubscript{bg}\textsuperscript{HSE}, E\textsubscript{bg}\textsuperscript{HSE(SOC)}, and E\textsubscript{bg}\textsuperscript{HSE-PBErel(SOC)} populated the multi-fidelity dataset I used for my work.
I aimed to accurately predict the band gaps of entirely hypothetical HaP compounds at the experimental fidelity using multi-fidelity models. 
This is motivated by the fact that the absorption spectrum of a perovskite determines its performance in a photovoltaic.
\autocite{mannodi-kanakkithodi-2019-compr-comput}
A well defined relationship therefore exists between the efficiency of a perovskite absorber and its true band gap.
\autocite{yu-2012-ident-poten}
The actual band gap (measured at the experimental fidelity) thus strongly predicts photovoltaic performance, and analysis of this relation informs the screening criterion.
My work aims to accelerate the design of high-performing photovoltaic devices by enabling more accurate rapid identification of candidate perovskites with desirable band gap properties.

\section{Spectroscopic Limited Maximum Efficiency (SLME)}
\label{sec:org67dad11}
Introduced by \textcite{yu-2012-ident-poten}, the SLME is a convenient metric for evaluating a semiconductor's suitability for single junction photovoltaic (PV) absorption.
In this work, SLME was calculated considering a 5\unit{\micro\meter} sample thickness for every perovskite using equations \ref{eq:absorption_alpha}, \ref{eq:slme_int}, and \ref{eq:slme_sum}, combining the original SL3ME.py code from \textcite{yu-2012-ident-poten} with our DFT computed absorption spectra and band gaps.

\begin{equation}
\label{eq:absorption_alpha}
a(E)=1-e^{-2\alpha(E)L}
\end{equation}

Here, \(\alpha(E)\) is the DFT computed optical absorption coefficient
as a function of incident photon energy and \(L\) is the thickness of
the absorber.

\begin{equation}
\label{eq:slme_int}
J=e\int_{0}^{\infty} a(E)I_{sun}(E)dE - J_{0}(1-e^{\frac{eV}{kT}})
\end{equation}

\begin{equation}
\label{eq:slme_sum}
\eta = \frac{P_{m}}{P_{in}}=\frac{\max(J \times V)}{P_{in}}
\end{equation}

To calculate SLME efficiency the current density \(J\), the light spectrum intensity of sunlight \(I_{sun}\), and the power \(P\) are all that is needed.
Using the DFT computed optical absorption spectrum as well as the magnitude and type (direct or indirect) of band gap as input, SLME is directly calculated using an open-source package.
\autocite{williams-2022-sl3me}
This calculation was performed using all four functionals and compliments the PCE measurements at the experimental fidelity.
SLME accounts for more energetic processes than the Shockley-Queisser criterion (\(bg \approx 1.3\)) allowing for a range of performant band gaps to be identified according to level of theory.
\autocite[p.1]{yu-2012-ident-poten}
Experimental data \autocite{almora-2020-devic-perfor} broadly agrees with PBE simulation, so the range of 1 to 2 \unit{\electronvolt} was justified (see figure \ref{fig:slme}).
Also, notice that even in just the sample dataset, there were candidates with potential to overtake the state of the art absorbers reported by NREL in figure \ref{fig:nrel}.
This is propitious for the screening I conducted on the 40000 point sample space.

 
\begin{figure}[htbp]
\centering
\includesvg[width=400pt]{./.ob-jupyter/1111b77f7e7ed1fa50791a404ec9652e766a082b}
\caption{\label{fig:slme} PBE SLME of sample compares to experimental PCE and cleanly demarcates competitive range of band gaps}
\end{figure}

\section{Improving Property Predictions using HSE06 and Spin-Orbit Coupling}
\label{sec:org417a4e0}
For a set of selected HaP compositions, while PBE-optimized lattice constants match well with experiments, PBE band gaps are underestimated, and HSE-PBE-SOC band gaps match better with measured values.
GGA-PBE computations reliably compute relaxed structures of both hybrid and purely inorganic HaPs.
However, advanced levels of theory such as the HSE06 functional with and without the inclusion of spin orbit coupling (SOC) to account for the relativistic effects of heavy atoms such as Pb, are of paramount importance when it comes to simulating electronic and optical properties.

The data set I used contains a series of \textasciitilde{}300 expensive HSE calculations across the 500 sampled compositions.
These are intended to yield insight into the effects of full geometry optimization at hybrid levels of theory to those of PBE-optimized structures.
Also, the effect of incorporating SOC in the calculation was examined.
In review, the sample of 500 band gaps available for training predictors was supplemented by 299 calculations conducted entirely at the HSE level of theory.
Furthermore, an additional 282 calculations were performed with HSE in addition to SOC, and 244 calculations were performed by running HSE(SOC) electronic structure calculations on PBE-relaxed structures.

The range of band gaps sampled by each simulation method are similar and are characterized by similar variance.
The descriptive statistics of each greatly exceeded those of the experimental subset (see figure \ref{fig:bg_dist}).
Nevertheless, the latter undoubtedly represented the smallest error from truth.
The types of mixing per level of theory were apportioned as in figure \ref{fig:lot_mix_org}.
This is the primary challenge I addressed with the multi-fidelity models discussed in chapter 3.

 
\begin{figure}[htbp]
\centering
\includesvg[width=300pt]{./.ob-jupyter/68775ba7ee8917dc6b42530cc8c59f2ed31c967d}
\caption{\label{fig:bg_dist} Variability in sampled band gaps at each fidelity}
\end{figure}

It is important to have a notion of which simulation is most accurate to the experimental measurements.
Figure \ref{fig:expqual} compares the band gaps obtained for a small subset of elements at all five levels of theory.
Theoretically, each functional may be more accurate for certain types of compositions.
For instance, organic-inorganic perovskites might benefit from greater account of Van der Waals forces and Pb-based compounds benefit from the use of spin orbit coupling as opposed to Pb-free compounds.
Note, phase information was not always available for certain experimental data points collected from the literature, and the inclusion of non-cubic phases in the tables may affect the evaluation of the functionals' accuracy.
Also, experimental data is tightly concentrated on the narrow range of performant band gaps likely due to selection bias.

The analysis is summarized in table \ref{tbl:expquant}.
HSE band gaps are heavily overestimated, but may be brought down by the addition of the SOC term.
Overall, HSE-PBErel(SOC) is the best approach for simulating band gaps with respect to computational cost and time.
PBE root mean square error (RMSE) is not significantly different from the HSE-PBErel(SOC) RMSE.
This is due to the accidental accuracy of semi-local functionals without SOC for hybrid organic-inorganic perovskites.
\autocite{mannodi-kanakkithodi-2019-compr-comput,mannodi-kanakkithodi-2022-data-driven}

 
\begin{figure}[htbp]
\centering
\includesvg[width=450pt]{./.ob-jupyter/087a658e6bb000a199d1fd4552bd9bde2db00444}
\caption{\label{fig:expqual} Effect of level of theory on band gap measurement}
\end{figure}

 
\begin{table}[htbp]
\caption{\label{tbl:expquant} RMSE values of band gaps computed from different functionals compared with experimental (EXP) values}
\centering
\begin{tabular}{lr}
 & RMSE vs EXP\\[0pt]
\hline
PBE & 0.55\\[0pt]
HSE & 0.87\\[0pt]
HSE(SOC) & 0.61\\[0pt]
HSE-PBErel(SOC) & 0.44\\[0pt]
\end{tabular}
\end{table}

\section{Sampling the Halide Perovskite Chemical Space}
\label{sec:orgcf27dc8}
Pure (non-alloyed) possibilities were exhaustively sampled using \(5*6*3 = 90\) compounds.
Starting from these pure perovskite structures systematic mixing was performed at the A, B, and X sites.
Figure \ref{fig:lot_mix_org} shows the shares of different types of mixing in our sample.
Again, for simplicity, only cardinal mixing was considered in this study: that is, mixing is not performed at multiple A/B/X-sites simultaneously.
The sample contains a reasonable balance of points representing each one of the cardinal mixing categories.
This helped to ensure the ML algorithms learn relationships between fidelities, not differences in mix site or constituency distributions within each fidelity.
Additionally, within each mix both purely inorganic samples and hybrid organic-inorganic samples were represented equally.

See the coverage of this sample in figure \ref{fig:coverage}.

 
\begin{figure}[htbp]
\centering
\includesvg[inkscapeformat=png, inkscapedpi=300,width=450pt]{./.ob-jupyter/201e8a95043e8f43d649342b896cab0dfa4ef32e}
\caption{\label{fig:lot_mix_org} Share by count of total data apportioned from each experimental subcategory}
\end{figure}

Most importantly, this sample gave even coverage of the cardinal mixing domain as shown in figure \ref{fig:coverage}.
The clusters in this figure were determined using the t-distributed stochastic neighbor embedding (t-SNE) method.
This is a nonparametric dimensionality reduction intended for visualizing statistically relevant clusters in a high dimensional dataset in only two or three dimensions.
In this case, the clusters correspond to the mix site of the member data points.
This sample provides the opportunity to comfortably interpolate the properties of other members of the cardinal mixing domain.
See Discussion.

 
\begin{figure}[htbp]
\centering
\includesvg[width=450pt]{./.ob-jupyter/8fb481208df6e81245038b7eaa4a261669d515f5}
\caption{\label{fig:coverage} Samples overlaid on cardinal mixing chemical domain projected from fourteen to two dimensions via t-SNE}
\end{figure}

Our multi-fidelity computational halide perovskite alloy dataset generated with the methods described here is one of the most comprehensive to date.
It is publicly available in the hopes further physical and engineering insights can be extracted by the broader research community.
It serves as the foundation for the modeling work presented in the following chapter.
