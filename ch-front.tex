\ProvidesFile{ch-front.tex}[2022-10-05 front matter chapter]
%
%  This is the ``front matter'' for the thesis.
%
%  REFERENCES
%
%    TCMOS17
%      The Chicago Manual of Style Online, 17th edition.
%      https://www.chicagomanualofstyle.org/home.html
%      retrieved on 2020-02-29
%
%    TEMPL
%      Thesis and Disertation Office Templates.
%      https://www.purdue.edu/gradschool/research/thesis/templates.html
%      retrieved on 2020-02-29
%
%    WNNCD
%    Webster's Ninth New Collegiate Dictionary.
%

%
%   Only Purdue University uses this page
%
%   Comment out \begin{statement} through \end{statement}
%   if you are not at Purdue University.
%
% Statement of Thesis/Dissertation Approval Page
% This page is REQUIRED.  The page should be numbered "2"
% and should NOT be listed in your TABLE OF CONTENTS.
\begin{statement}
  % Delete or add \entry commands as needed for all committe members.
  \entry{Dr.~Arun Mannodi-Kanakkithodi, Chair}{School of Materials Engineering}
  \entry{Dr.~Alejandro Strachan}{School of Materials Engineering}
  \entry{Dr.~Kendra Erk}{School of Materials Engineering}
  % There should be one \approvedby command containing the
  % "FORM 9 THESIS FORM HEAD NAME HERE" (from TEMPL, retrieved on 2020-03-01).
  \approvedby{Pending FORM 9 Approval}
\end{statement}

% Dedication page is optional.
% A name and often a message in tribute to a person or cause.
% References: WEB9 332.
\begin{dedication}
  To my family
\end{dedication}

% Acknowledgements page is optional but most theses include
% a brief statement of appreciation or recognition of special
% assistance.
\begin{acknowledgments}
  I am grateful for the guidance of my advisor, Dr.~Arun
  Mannodi-Kanakkithodi.
  The Ross Fellowship awarded to me by Purdue University Graduate
  School helped fund my work.
  The Rosen Center of Advanced Computing provided the computational
  resources needed to conduct simulations, store data, and train models.
\end{acknowledgments}

% The preface is optional.
% References: TCMOS17 1.49, WEB9 927.
\begin{preface}
  This is the preface.
\end{preface}

% The Table of Contents is required.
% The Table of Contents will be automatically created for you
% using information you supply in
%     \chapter
%     \section
%     \subsection
%     \subsubsection
%     commands.
\pdfbookmark{TABLE OF CONTENTS}{Contents}
\tableofcontents

% If your thesis has tables, a list of tables is required.
% The List of Tables will be automatically created for you using
% information you supply in
%     \begin{table} ... \end{table}
% environments.
\listoftables

% If your thesis has figures, a list of figures is required.
% The List of Figures will be automatically created for you using
% information you supply in
%     \begin{figure} ... \end{figure}
% environments.
\listoffigures

% If your thesis has listings, a list of listings is required.
% The List of Listings will be automatically created for you using
% information you supply in
%     \begin{ZZlisting} ... \end{ZZlisting}
% environments.
\ZZlistoflistings

% If your thesis has protocols, you may want to do a list of protocols.
% The List of Protocols will be automatically created for you using
% information you supply in
%     \begin{protocol} ... \end{protocol}
% environments.
\listofprotocols

% If your thesis has schemes, you may want to do a list of schemes.
% The List of Schemes will be automatically created for you using
% information you supply in
%     \begin{scheme} ... \end{scheme}
% environments.
\listofschemes

% List of Symbols is optional.
\begin{symbols}
  $m$& mass\cr
  $v$& velocity\cr
\end{symbols}

% List of Abbreviations is optional.
\begin{abbreviations}
  abbr& abbreviation\cr
  bcf& billion cubic feet\cr
  BMOC& big man on campus\cr
\end{abbreviations}

% Nomenclature is optional.
\begin{nomenclature}
  alanine& 2-Aminopropanoic acid\\
\end{nomenclature}

% Glossary is optional.
\begin{glossary}
  philtrum& the groove between the nose and upper lip\\
\end{glossary}

\begin{abstract}%
  A wide range of optoelectronic applications demand semiconductors optimized for purpose.
  My research focused on data-driven identification of \(\ce{ABX3}\) Halide perovskite compositions for optimum photovoltaic absorption in solar cells.
  I identified mixtures of candidate constituents at the A, B or X sites of the perovskite supercell which leveraged how mixed perovskite band gaps ``bow'' away from the linear interpolations predicted by Vegard's law of mixing to obtain a selection of stable perovskites with band gaps in the ideal range of 1 to 2.5 eV for visible light spectrum absorption.

  I trained machine learning models on previously reported datasets of halide perovskite band gaps based on first principles computations performed at different fidelities.
  The primary objective of these models was to predict the perovskite band gap using the composition and inherent elemental properties as descriptors, eventually leading to accurate prediction and screening across the much larger chemical space from which the data samples were drawn.

  I utilized a recently published density functional theory (DFT) dataset of more than 1300 perovskite band gaps from four different levels of theory, added to an experimental perovskite band gap dataset of \textasciitilde{}100 points, to train random forest regression (RFR), Gaussian process regression (GPR), and Sure Independence Screening and Sparsifying Operator (SISSO) regression models, with data fidelity added as one-hot encoded features.
  I found that RFR yields the best model with a band gap root mean square error of 0.12 eV on the total dataset and 0.15 eV on the experimental points.
  SISSO provided compound features and functions for direct prediction of band gap, but errors were larger than from RFR and GPR.
  Additional insights gained from Pearson correlation and Shapley additive explanation (SHAP) analysis of learned descriptors suggest the RFR models performed best because of (a) their focus on identifying and capturing relevant feature interactions and (b) their flexibility to represent nonlinear relationships between such interactions and the band gap.
  % and (c) their benefit as an ensemble model to improve predictions by reducing the variance of errors
  The best model was deployed for predicting experimental band gap of \textasciitilde{}40,000 hypothetical compounds, based on which we screened \textasciitilde{}3000 stable compounds with band gap predicted to be between 1 and 2.5 eV at experimental accuracy.
\end{abstract}
