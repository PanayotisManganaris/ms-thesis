\ProvidesFile{ch-front.tex}[2022-10-05 front matter chapter]
%
%  This is the ``front matter'' for the thesis.
%
%  REFERENCES
%
%    TCMOS17
%      The Chicago Manual of Style Online, 17th edition.
%      https://www.chicagomanualofstyle.org/home.html
%      retrieved on 2020-02-29
%
%    TEMPL
%      Thesis and Disertation Office Templates.
%      https://www.purdue.edu/gradschool/research/thesis/templates.html
%      retrieved on 2020-02-29
%
%    WNNCD
%    Webster's Ninth New Collegiate Dictionary.
%

%
%   Only Purdue University uses this page
%
%   Comment out \begin{statement} through \end{statement}
%   if you are not at Purdue University.
%
% Statement of Thesis/Dissertation Approval Page
% This page is REQUIRED.  The page should be numbered "2"
% and should NOT be listed in your TABLE OF CONTENTS.
\begin{statement}
  % Delete or add \entry commands as needed for all committe members.
  \entry{Dr.~Arun Mannodi-Kanakkithodi, Chair}{School of Materials Engineering}
  \entry{Dr.~Alejandro Strachan}{School of Materials Engineering}
  \entry{Dr.~Kendra Erk}{School of Materials Engineering}
  % There should be one \approvedby command containing the
  % "FORM 9 THESIS FORM HEAD NAME HERE" (from TEMPL, retrieved on 2020-03-01).
  \approvedby{Pending FORM 9 Approval}
\end{statement}

% Dedication page is optional.
% A name and often a message in tribute to a person or cause.
% References: WEB9 332.
\begin{dedication}
  To my family
\end{dedication}

% Acknowledgements page is optional but most theses include
% a brief statement of appreciation or recognition of special
% assistance.
\begin{acknowledgments}
  I am grateful for the guidance of my advisor, Dr.~Arun
  Mannodi-Kanakkithodi.
  The Ross Fellowship awarded to me by Purdue University Graduate
  School helped fund my work.
  The Rosen Center of Advanced Computing provided the computational
  resources needed to conduct simulations, store data, and train models.
\end{acknowledgments}

% The preface is optional.
% References: TCMOS17 1.49, WEB9 927.
\begin{preface}
  This is the preface.
\end{preface}

% The Table of Contents is required.
% The Table of Contents will be automatically created for you
% using information you supply in
%     \chapter
%     \section
%     \subsection
%     \subsubsection
%     commands.
\pdfbookmark{TABLE OF CONTENTS}{Contents}
\tableofcontents

% If your thesis has tables, a list of tables is required.
% The List of Tables will be automatically created for you using
% information you supply in
%     \begin{table} ... \end{table}
% environments.
\listoftables

% If your thesis has figures, a list of figures is required.
% The List of Figures will be automatically created for you using
% information you supply in
%     \begin{figure} ... \end{figure}
% environments.
\listoffigures

% If your thesis has listings, a list of listings is required.
% The List of Listings will be automatically created for you using
% information you supply in
%     \begin{ZZlisting} ... \end{ZZlisting}
% environments.
\ZZlistoflistings

% If your thesis has protocols, you may want to do a list of protocols.
% The List of Protocols will be automatically created for you using
% information you supply in
%     \begin{protocol} ... \end{protocol}
% environments.
\listofprotocols

% If your thesis has schemes, you may want to do a list of schemes.
% The List of Schemes will be automatically created for you using
% information you supply in
%     \begin{scheme} ... \end{scheme}
% environments.
\listofschemes

% List of Symbols is optional.
\begin{symbols}
  $m$& mass\cr
  $v$& velocity\cr
\end{symbols}

% List of Abbreviations is optional.
\begin{abbreviations}
  HaP          & halide perovskite\cr
  VASP         & Vienna Ab initio Simulation Package\cr
  QM/ML        & quantum mechanics machine learning\cr
  SLME         & spectroscopic limited maximum efficiency\cr
  PCE          & power conversion efficiency\cr
  DFT          & density functional theory\cr
  GGA          & generalized gradient approximation\cr
  PBE          & Perdew-Burke-Ernzerhof Functional\cr
  HSE06        & Heyd-Scuseria-Ernzerhof Functional\cr
  PCA          & principal component analysis\cr
  t-SNE        & t-distributed stochastic neighbor embedding\cr
  UMAP         & uniform manifold approximation and projection\cr
  GPR          & Gaussian Process Regression\cr
  RFR          & Random Forest Regression\cr
  SISSO        & Sure Independence Screening and Sparsifying Operator\cr
  SQS          & special quasi-random structures\cr
  PAW          & projector augmented wave\cr
  NIST         & National Institute of Standards and Technology\cr
  PES          & Potential Energy Surface\cr
  SHAP         & Shapley Additive Explaination\cr
  GNN          & Graph Neural Networks\cr
\end{abbreviations}

% Nomenclature is optional.
\begin{nomenclature}
  MA & Methylamonium (Cationic Methylamine) \(\ce{CH3NH3+}\)\\
  FA & Formamidium (Cationic Formamidine) \(\ce{CH(NH2)2+}\)\\
  % GU & Guanidinium (Cationic Guanidine) \\
\end{nomenclature}

% Glossary is optional.
\begin{glossary}
  Law of Mixing& linear interpolation predicts the properties of materials band gaps of pure compounds predicted by the groove between the nose and upper lip\\
  cardinal mixing         & Describes perovskite alloys where no more than one of the A, B, or X sites is occupied by multiple possible constituents\\
  partition               & Portion of sample data reserved for a purpose in model development\\
  cross-validation        & Method for gathering statistics on the abilities of a model to fit to the parent partition\\
  K-fold split            & Data partition divided into K arbitrary groups for use in cross-validation schemes\\
  groupwise K-fold        & Data partition divided into K-folds where each fold corresponds to a category label\\
  level of theory         & Refers to the rank of a DFT functional in the hierarchy of phenomenological comprehensiveness. A proxy for accuracy.\\
  Materials Project       & US Government-led multidisciplinary collaboration founded in 2011 as the Materials Genome Initiative.\\
  machine learning        & a science concerned with algorithms which improve their performance with exposure to new data\\
  features                & attributes of an observed event or object which might empirically explain the event or object\\
  hyper-parameter         & a setting that controls how a learning algorithm works\\
  classical learning      & a paradigm of machine learning that is dependent on expert knowledge to extract quality features from samples in a dataset\\
  surrogate model         & a representation which attempts to capture as much of the relationship between a domain and a target property as possible\\
  deep learning           & a paradigm of machine learning differing from classical learning in that the features of the input data are themselves learned by the algorithm\\
  latent space            & a multidimensional abstraction of a problem space. the relationship between coordinates in this space and observation in the real world can be formulated to guarantee the viability of solutions in the abstraction\\
  evolutionary algorithms & a class of nature-inspire algorithms often applied to optimization in high dimensional discontinuous functions\\
  ALIGNN                  & the Atomistic Line Graph Neural Network considers relative positions of atoms in a crystal as well as the relative angles between bonds by creating two related node and edge graphs and convluting them in a staggered manner together\\
  Gaussian Process        & Any function which returns samples from an underlying multivariate normal distribution\\
  FAIR                    & Findable Accessible Interoperable and Reusable Data\\
  multi-task learning     & A type of machine learning where an algorithm learns multiple functions simultaneously, while exploiting commonalities and differences between the functions\\
  Spin Orbit Coupling     & An additional term intended to account for the increased relevance of quantum angular momentum to electromagnetic response in heavy atoms\\
\end{glossary}

\begin{abstract}%
  A wide range of optoelectronic applications demand semiconductors optimized for purpose.
  My research focused on data-driven identification of \(\ce{ABX3}\) Halide perovskite compositions for optimum photovoltaic absorption in solar cells.
  I identified mixtures of candidate constituents at the A, B or X sites of the perovskite supercell which leveraged how mixed perovskite band gaps ``bow'' away from the linear interpolations predicted by Vegard's law of mixing to obtain a selection of stable perovskites with band gaps in the ideal range of 1 to 2.5 eV for visible light spectrum absorption.

  I trained machine learning models on previously reported datasets of halide perovskite band gaps based on first principles computations performed at different fidelities.
  The primary objective of these models was to predict the perovskite band gap using the composition and inherent elemental properties as descriptors, eventually leading to accurate prediction and screening across the much larger chemical space from which the data samples were drawn.

  I utilized a recently published density functional theory (DFT) dataset of more than 1300 perovskite band gaps from four different levels of theory, added to an experimental perovskite band gap dataset of \textasciitilde{}100 points, to train random forest regression (RFR), Gaussian process regression (GPR), and Sure Independence Screening and Sparsifying Operator (SISSO) regression models, with data fidelity added as one-hot encoded features.
  I found that RFR yields the best model with a band gap root mean square error of 0.12 eV on the total dataset and 0.15 eV on the experimental points.
  SISSO provided compound features and functions for direct prediction of band gap, but errors were larger than from RFR and GPR.
  Additional insights gained from Pearson correlation and Shapley additive explanation (SHAP) analysis of learned descriptors suggest the RFR models performed best because of (a) their focus on identifying and capturing relevant feature interactions and (b) their flexibility to represent nonlinear relationships between such interactions and the band gap.
  % and (c) their benefit as an ensemble model to improve predictions by reducing the variance of errors
  The best model was deployed for predicting experimental band gap of \textasciitilde{}40,000 hypothetical compounds, based on which we screened \textasciitilde{}3000 stable compounds with band gap predicted to be between 1 and 2.5 eV at experimental accuracy.
\end{abstract}
