\ProvidesFile{ch-front.tex}[2022-10-05 front matter chapter]
%
%  This is the ``front matter'' for the thesis.
%
%  REFERENCES
%
%    TCMOS17
%      The Chicago Manual of Style Online, 17th edition.
%      https://www.chicagomanualofstyle.org/home.html
%      retrieved on 2020-02-29
%
%    TEMPL
%      Thesis and Disertation Office Templates.
%      https://www.purdue.edu/gradschool/research/thesis/templates.html
%      retrieved on 2020-02-29
%
%    WNNCD
%    Webster's Ninth New Collegiate Dictionary.
%

%
%   Only Purdue University uses this page
%
%   Comment out \begin{statement} through \end{statement}
%   if you are not at Purdue University.
%
% Statement of Thesis/Dissertation Approval Page
% This page is REQUIRED.  The page should be numbered "2"
% and should NOT be listed in your TABLE OF CONTENTS.
\begin{statement}
  % Delete or add \entry commands as needed for all committe members.
  \entry{Dr.~Arun Mannodi-Kanakkithodi, Chair}{School of Materials Engineering}
  \entry{Dr.~Alejandro Strachan}{School of Materials Engineering}
  \entry{Dr.~Kendra Erk}{School of Materials Engineering}
  % There should be one \approvedby command containing the
  % "FORM 9 THESIS FORM HEAD NAME HERE" (from TEMPL, retrieved on 2020-03-01).
  \approvedby{Dr.~Nikhilesh Chawla}
\end{statement}

% Dedication page is optional.
% A name and often a message in tribute to a person or cause.
% References: WEB9 332.
\begin{dedication}
  \textit{To my family, especially my brother Tassos.}
\end{dedication}

% Acknowledgements page is optional but most theses include
% a brief statement of appreciation or recognition of special
% assistance.
\begin{acknowledgments}
  I am grateful to
  Professor Arun Mannodi-Kanakkithodi for his mentorship, support, and knowledge,
  Professor Erk for her guidance and counsel, and
  Professor Strachan for his instruction and advice.
  I also thank
  my colleague Jiaqi Yang for his collaboration, and
  Habibur Rahman for his friendship and curiosity.
  My work was funded in part by my advisor's startup grant F.10023800.05.002 and the Ross Fellowship awarded to me by Purdue University for which I am honored and thankful.
  The Rosen Center of Advanced Computing provided the computational
  resources needed to conduct simulations, store data, and train models.
  I deeply appreciate the opportunity to study, learn, and grow at Purdue University.
\end{acknowledgments}

% The preface is optional.
% References: TCMOS17 1.49, WEB9 927.
% \begin{preface}
%   This is the preface.
% \end{preface}

% The Table of Contents is required.
% The Table of Contents will be automatically created for you
% using information you supply in
%     \chapter
%     \section
%     \subsection
%     \subsubsection
%     commands.
\pdfbookmark{TABLE OF CONTENTS}{Contents}
\tableofcontents

% If your thesis has tables, a list of tables is required.
% The List of Tables will be automatically created for you using
% information you supply in
%     \begin{table} ... \end{table}
% environments.
\listoftables

% If your thesis has figures, a list of figures is required.
% The List of Figures will be automatically created for you using
% information you supply in
%     \begin{figure} ... \end{figure}
% environments.
\listoffigures

% If your thesis has listings, a list of listings is required.
% The List of Listings will be automatically created for you using
% information you supply in
%     \begin{ZZlisting} ... \end{ZZlisting}
% environments.
\ZZlistoflistings

% If your thesis has protocols, you may want to do a list of protocols.
% The List of Protocols will be automatically created for you using
% information you supply in
%     \begin{protocol} ... \end{protocol}
% environments.
% \listofprotocols

% If your thesis has schemes, you may want to do a list of schemes.
% The List of Schemes will be automatically created for you using
% information you supply in
%     \begin{scheme} ... \end{scheme}
% environments.
% \listofschemes

% List of Symbols is optional.
\begin{symbols}
  \(\alpha\)& Optical Absorption Coefficient except when used as HSE hybridization parameter\cr
  \(\gamma\)& Photon\cr
  \(\epsilon\)& Electron Energy\cr
  \(\eta\)& Efficiency\cr
  \(\lambda\)& Optical Wavelength\cr
  \(\tau\)& Bartel Tolerance Factor\cr
  \(\psi\)& Time-independent Wave Function\cr
  \(\Psi\)& Time-dependent Wave Function\cr
  \(E\)& Total Energy\cr
  \(\hbar\)& Reduced Planck constant\cr
  \(\mathcal{H}\)& Hamiltonian Operator\cr
  \(\hat{H}\)& Quantum Hamiltonian Operator\cr
  \(I_{sun}\)& Light Spectrum Intensity of Sunlight\cr
  \(J\)& Current Density\cr
  \(k\)& Boltzmann Constant\cr
  \(m\)& Electron Mass\cr
  \(n\)& Electron Density\cr
  \(o\)& Octahedral Tolerance Factor\cr
  \(P\)& Power\cr
  \(T\)& Temperature\cr
  \(t\)& Goldschmidt Tolerance Factor\cr
  \(\hat{T}\)& Kinetic Energy Operator\cr
  \(\hat{U}\)& Electronic Potential Energy Operator\cr
  \(\hat{V}\)& External Potential Energy Operator\cr
\end{symbols}

% List of Abbreviations is optional.
\begin{abbreviations}
  BOA          & Born-Oppenheimer Approximation\cr
  DFT          & Density Functional Theory\cr
  EXP          & experimental fidelity data\cr
  GGA          & Generalized Gradient Approximation\cr
  GPR          & Gaussian Process Regression\cr
  HaP          & Halide Perovskite\cr
  HPO          & Hyper-Parameter Optimization\cr
  HSE          & Heyd-Scuseria-Ernzerhof/Hartree-Folk hybrid functional with \(\alpha = 0.25\) and \(\omega = 02\) (HSE06)\cr
  LoT          & Level of Theory\cr
  PAW          & Projector Augmented Wave\cr
  PBE          & Perdew-Burke-Ernzerhof GGA functional\cr
  PCE          & Power Conversion Efficiency\cr
  rel          & Relaxation\cr
  RFR          & Random Forest Regression\cr
  SHAP         & Shapley Additive Explanation\cr
  SISSO        & Sure Independence Screening and Sparsifying Operator\cr
  SLME         & Spectroscopic Limited Maximum Efficiency\cr
  SOC          & Spin-Orbit Coupling\cr
  SQS          & Special Quasi-random Structures\cr
  t-SNE        & t-Distributed Stochastic Neighbor Embedding\cr
  VASP         & Vienna Ab initio Simulation Package\cr
\end{abbreviations}

% Nomenclature is optional.
\begin{nomenclature}
  % GU & Guanidinium (Cationic Guanidine) \\
  FA & Formamidinium (Cationic Formamidine) \(\ce{CH(NH2)2+}\)\\
  MA & Methylammonium (Cationic Methylamine) \(\ce{CH3NH3+}\)\\
\end{nomenclature}

% Glossary is optional.
\begin{glossary}
  \vspace{8pt}cardinal mixing         & describes perovskite alloys where no more than one of the A, B, or X sites is occupied by multiple possible constituents\cr
  \vspace{8pt}cross-validation        & method for gathering statistics on the abilities of a model to fit to the parent partition\cr
  \vspace{8pt}features                & attributes of an observed event or object which might empirically explain the event or object\cr
  \vspace{8pt}group-wise K-fold       & data partition divided into K-folds where each fold corresponds to a category label\cr
  \vspace{8pt}hyper-parameter         & a setting that controls how a learning algorithm works\cr
  \vspace{8pt}K-fold split            & data partition divided into K arbitrary groups for use in cross-validation schemes\cr
  \vspace{8pt}law of mixing           & the rule stating properties of materials of mixed compositions may be predicted by linear interpolation of the properties of similar materials with pure compositions\cr
  \vspace{8pt}level of theory         & refers to the rank of a DFT functional in the hierarchy of phenomenological comprehensiveness. A proxy for accuracy.\cr
  \vspace{8pt}machine learning        & a science concerned with algorithms which improve their performance with exposure to new data\cr
  \vspace{8pt}Materials Project       & US Government-led multidisciplinary collaboration founded in 2011 as the Materials Genome Initiative.\cr
  \vspace{8pt}multi-task learning     & a type of machine learning where an algorithm learns multiple functions simultaneously, while exploiting commonalities and differences between the functions\cr
  \vspace{8pt}partition               & portion of sample data reserved for a purpose in model development\cr
  \vspace{8pt}spin orbit coupling     & an additional term intended to account for the increased relevance of quantum angular momentum to electromagnetic response in heavy atoms\cr
\end{glossary}

\begin{abstract}%
  A wide range of optoelectronic applications demand semiconductors optimized for purpose.
  My research focused on data-driven identification of \(\ce{ABX3}\) Halide perovskite compositions for optimum photovoltaic absorption in solar cells.
  I trained machine learning models on previously reported datasets of halide perovskite band gaps based on first principles computations performed at different fidelities.
  Using these, I identified mixtures of candidate constituents at the A, B or X sites of the perovskite supercell which leveraged how mixed perovskite band gaps deviate from the linear interpolations predicted by Vegard's law of mixing to obtain a selection of stable perovskites with band gaps in the ideal range of 1 to 2 eV for visible light spectrum absorption.
  These models predict the perovskite band gap using the composition and inherent elemental properties as descriptors.
  This enables accurate, high fidelity prediction and screening of the much larger chemical space from which the data samples were drawn.

  I utilized a recently published density functional theory (DFT) dataset of more than 1300 perovskite band gaps from four different levels of theory, added to an experimental perovskite band gap dataset of \textasciitilde{}100 points, to train random forest regression (RFR), Gaussian process regression (GPR), and Sure Independence Screening and Sparsifying Operator (SISSO) regression models, with data fidelity added as one-hot encoded features.
  I found that RFR yields the best model with a band gap root mean square error of 0.12 eV on the total dataset and 0.15 eV on the experimental points.
  SISSO provided compound features and functions for direct prediction of band gap, but errors were larger than from RFR and GPR.
  Additional insights gained from Pearson correlation and Shapley additive explanation (SHAP) analysis of learned descriptors suggest the RFR models performed best because of (a) their focus on identifying and capturing relevant feature interactions and (b) their flexibility to represent nonlinear relationships between such interactions and the band gap.
  % and (c) their benefit as an ensemble model to improve predictions by reducing the variance of errors
  The best model was deployed for predicting experimental band gap of 37785 hypothetical compounds.
  Based on this, we identified 1251 stable compounds with band gap predicted to be between 1 and 2 eV at experimental accuracy, successfully narrowing the candidates to about 3\% of the screened compositions.
\end{abstract}
