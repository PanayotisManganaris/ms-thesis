% Created 2023-05-24 Wed 09:07
% Intended LaTeX compiler: pdflatex
\documentclass[10pt, aspectratio=169, presentation]{beamer}
\usepackage[utf8]{inputenc}
\usepackage[T1]{fontenc}
\usepackage{graphicx}
\usepackage{longtable}
\usepackage{wrapfig}
\usepackage{rotating}
\usepackage[normalem]{ulem}
\usepackage{amsmath}
\usepackage{amssymb}
\usepackage{capt-of}
\usepackage{hyperref}
\usepackage{ptm}

%\setbeamertemplate{caption}[numbered]
\beamertemplatenavigationsymbolsempty
%
\usepackage{mathtools}%% bugfixing and additional tools for amsmath
\usepackage{amsfonts}%% formal math fonts
\usepackage{physics}%% powerful mathematics shorthand
\usepackage{bm}%% bold math
\usepackage{gensymb}%% inter-environment consistent measurment unit symbols
\usepackage{upgreek}%% easy lower and uppercase nonitalicized greek letters
\usepackage[version=4]{mhchem}%% for writing chemical formulae
\usepackage{siunitx}[=v2]%% version 2 workaround first-frame artifact
%
\usepackage[%
%citestyle=authoryear-icomp,
%bibstyle=authoryear-icomp,
style=authortitle-tcomp,
%style=verbose,
autocite=footnote,
hyperref=true,
backref=true,
maxcitenames=3,
url=true,
backend=biber,
natbib=true,
]{biblatex}
\addbibresource{/home/panos/Documents/masters/local-bib.bib}
\addbibresource{~/org/bibliotex/bibliotex.bib}
%
\usepackage{svg}
\svgsetup{%
inkscapelatex=false,
inkscapeformat=png,
inkscapedpi=300,
}
\svgpath{{/home/panos/Pictures/figs}}
%
\setbeamertemplate{footline}{%
\hfill
\vspace*{0.1cm}
\insertframenumber\slash\inserttotalframenumber
\hspace*{0.2cm}
}
\institute[Mannodi Group]{\large{}
\inst{1} Purdue Materials Engineering\\Advisor Arun Mannodi-Kanakkithodi
}
\AtBeginSection[]
{
\begin{frame}
\frametitle{Table of Contents}
\tableofcontents[currentsection]
\end{frame}
}
\usetheme{default}
\author{\large{}Panayotis Manganaris\inst{1}}
\date{\today}
\title{\Huge{}Machine Learning Perovskite Bandgaps for Improved Photovoltaics}
\hypersetup{
 pdfauthor={\large{}Panayotis Manganaris\inst{1}},
 pdftitle={\Huge{}Machine Learning Perovskite Bandgaps for Improved Photovoltaics},
 pdfkeywords={},
 pdfsubject={},
 pdfcreator={Emacs 28.2 (Org mode 9.5.5)}, 
 pdflang={English}}

% Setup for code blocks [1/2]

\usepackage{fvextra}

\fvset{%
  commandchars=\\\{\},
  highlightcolor=white!95!black!80!blue,
  breaklines=true,
  breaksymbol=\color{white!60!black}\tiny\ensuremath{\hookrightarrow}}

% Make line numbers smaller and grey.
\renewcommand\theFancyVerbLine{\footnotesize\color{black!40!white}\arabic{FancyVerbLine}}

\usepackage{xcolor}

% In case engrave-faces-latex-gen-preamble has not been run.
\providecolor{EfD}{HTML}{f7f7f7}
\providecolor{EFD}{HTML}{28292e}

% Define a Code environment to prettily wrap the fontified code.
\usepackage[breakable,xparse]{tcolorbox}
\DeclareTColorBox[]{Code}{o}%
{colback=EfD!98!EFD, colframe=EfD!95!EFD,
  fontupper=\footnotesize\setlength{\fboxsep}{0pt},
  colupper=EFD,
  IfNoValueTF={#1}%
  {boxsep=2pt, arc=2.5pt, outer arc=2.5pt,
    boxrule=0.5pt, left=2pt}%
  {boxsep=2.5pt, arc=0pt, outer arc=0pt,
    boxrule=0pt, leftrule=1.5pt, left=0.5pt},
  right=2pt, top=1pt, bottom=0.5pt,
  breakable}

% Support listings with captions
\usepackage{float}
\floatstyle{plain}
\newfloat{listing}{htbp}{lst}
\newcommand{\listingsname}{Listing}
\floatname{listing}{\listingsname}
\newcommand{\listoflistingsname}{List of Listings}
\providecommand{\listoflistings}{\listof{listing}{\listoflistingsname}}


% Setup for code blocks [2/2]: syntax highlighting colors

\newcommand\efstrut{\vrule height 2.1ex depth 0.8ex width 0pt}
\definecolor{EFD}{HTML}{000000}
\definecolor{EfD}{HTML}{ffffff}
\newcommand{\EFD}[1]{\textcolor{EFD}{#1}} % default
\definecolor{EFh}{HTML}{7f7f7f}
\newcommand{\EFh}[1]{\textcolor{EFh}{#1}} % shadow
\definecolor{EFsc}{HTML}{228b22}
\newcommand{\EFsc}[1]{\textcolor{EFsc}{\textbf{#1}}} % success
\definecolor{EFw}{HTML}{ff8e00}
\newcommand{\EFw}[1]{\textcolor{EFw}{\textbf{#1}}} % warning
\definecolor{EFe}{HTML}{ff0000}
\newcommand{\EFe}[1]{\textcolor{EFe}{\textbf{#1}}} % error
\definecolor{EFc}{HTML}{b22222}
\newcommand{\EFc}[1]{\textcolor{EFc}{#1}} % font-lock-comment-face
\definecolor{EFcd}{HTML}{b22222}
\newcommand{\EFcd}[1]{\textcolor{EFcd}{#1}} % font-lock-comment-delimiter-face
\definecolor{EFs}{HTML}{8b2252}
\newcommand{\EFs}[1]{\textcolor{EFs}{#1}} % font-lock-string-face
\definecolor{EFd}{HTML}{8b2252}
\newcommand{\EFd}[1]{\textcolor{EFd}{#1}} % font-lock-doc-face
\definecolor{EFm}{HTML}{008b8b}
\newcommand{\EFm}[1]{\textcolor{EFm}{#1}} % font-lock-doc-markup-face
\definecolor{EFk}{HTML}{9370db}
\newcommand{\EFk}[1]{\textcolor{EFk}{#1}} % font-lock-keyword-face
\definecolor{EFb}{HTML}{483d8b}
\newcommand{\EFb}[1]{\textcolor{EFb}{#1}} % font-lock-builtin-face
\definecolor{EFf}{HTML}{0000ff}
\newcommand{\EFf}[1]{\textcolor{EFf}{#1}} % font-lock-function-name-face
\definecolor{EFv}{HTML}{a0522d}
\newcommand{\EFv}[1]{\textcolor{EFv}{#1}} % font-lock-variable-name-face
\definecolor{EFt}{HTML}{228b22}
\newcommand{\EFt}[1]{\textcolor{EFt}{#1}} % font-lock-type-face
\definecolor{EFo}{HTML}{008b8b}
\newcommand{\EFo}[1]{\textcolor{EFo}{#1}} % font-lock-constant-face
\definecolor{EFwr}{HTML}{ff0000}
\newcommand{\EFwr}[1]{\textcolor{EFwr}{\textbf{#1}}} % font-lock-warning-face
\newcommand{\EFnc}[1]{#1} % font-lock-negation-char-face
\definecolor{EFpp}{HTML}{483d8b}
\newcommand{\EFpp}[1]{\textcolor{EFpp}{#1}} % font-lock-preprocessor-face
\newcommand{\EFrc}[1]{\textbf{#1}} % font-lock-regexp-grouping-construct
\newcommand{\EFrb}[1]{\textbf{#1}} % font-lock-regexp-grouping-backslash
\newcommand{\EFob}[1]{#1} % org-block
\definecolor{EFhn}{HTML}{008b8b}
\newcommand{\EFhn}[1]{\textcolor{EFhn}{#1}} % highlight-numbers-number
\definecolor{EFhq}{HTML}{9370db}
\newcommand{\EFhq}[1]{\textcolor{EFhq}{#1}} % highlight-quoted-quote
\definecolor{EFhs}{HTML}{008b8b}
\newcommand{\EFhs}[1]{\textcolor{EFhs}{#1}} % highlight-quoted-symbol
\definecolor{EFrda}{HTML}{707183}
\newcommand{\EFrda}[1]{\textcolor{EFrda}{#1}} % rainbow-delimiters-depth-1-face
\definecolor{EFrdb}{HTML}{7388d6}
\newcommand{\EFrdb}[1]{\textcolor{EFrdb}{#1}} % rainbow-delimiters-depth-2-face
\definecolor{EFrdc}{HTML}{909183}
\newcommand{\EFrdc}[1]{\textcolor{EFrdc}{#1}} % rainbow-delimiters-depth-3-face
\definecolor{EFrdd}{HTML}{709870}
\newcommand{\EFrdd}[1]{\textcolor{EFrdd}{#1}} % rainbow-delimiters-depth-4-face
\definecolor{EFrde}{HTML}{907373}
\newcommand{\EFrde}[1]{\textcolor{EFrde}{#1}} % rainbow-delimiters-depth-5-face
\definecolor{EFrdf}{HTML}{6276ba}
\newcommand{\EFrdf}[1]{\textcolor{EFrdf}{#1}} % rainbow-delimiters-depth-6-face
\definecolor{EFrdg}{HTML}{858580}
\newcommand{\EFrdg}[1]{\textcolor{EFrdg}{#1}} % rainbow-delimiters-depth-7-face
\definecolor{EFrdh}{HTML}{80a880}
\newcommand{\EFrdh}[1]{\textcolor{EFrdh}{#1}} % rainbow-delimiters-depth-8-face
\definecolor{EFrdi}{HTML}{887070}
\newcommand{\EFrdi}[1]{\textcolor{EFrdi}{#1}} % rainbow-delimiters-depth-9-face
\begin{document}

\maketitle
\begin{frame}{Outline}
\tableofcontents
\end{frame}

\section{Background}
\label{sec:org4a9fd49}
\begin{frame}[label={sec:orge93a328}]{Photovoltaic (PV) Power Generation}
\begin{columns}
\begin{column}{0.3\columnwidth}
\begin{center}
\includegraphics[width=100]{thin-film-PV.png}
\end{center}
\begin{center}
\includegraphics[width=90]{Si-HaP-tandem-PV.png}
\end{center}

\tiny{}\center{}Perovskite Applications\footnote{\tiny{}\href{https://www.lesker.com/newweb/ped/applications/perovskite-research.cfm }{www.lesker.com}}
\end{column}

\begin{column}{0.7\columnwidth}
\begin{block}{Band Gap Effects on Absorption Spectrum}
\begin{columns}
\begin{column}{0.5\columnwidth}
 
\begin{center}
\includesvg[width=130pt]{./.ob-jupyter/e7503af7c27d2b1ea6f3f80eb79c3f85ba9928b4}
\end{center}
\end{column}

\begin{column}{0.5\columnwidth}
\begin{align*}
E_\gamma = \hbar\omega &= bg\,\si{\joule} \\
\implies \frac{hc}{628\,\si{\nano\meter}} &= 3.16\times{}10^{-19}\,\si{\joule}\\
&= 1.97\,\si{\electronvolt}
\end{align*}
\end{column}
\end{columns}
\end{block}

\begin{block}{Perovskite Semiconductors in Solar Energy Harvesting}
\begin{itemize}
\item 2009 : \(\ce{MAPbI_3}\) dye-sensitized absorber with 3.8\% PCE
\autocite{kojima-2009-organ-halid}
\item 2011 : absorber efficiency doubles
\autocite{im-2011}
\item 2013 : \(\ce{MAPbX_3}\) found to transport holes+electrons
\autocite{saliba-2014-influen-therm}
\end{itemize}

HaP PVs become competitive in 2013
\end{block}
\end{column}
\end{columns}
\end{frame}
\begin{frame}[label={sec:orgf3e517e}]{Rapid Rise of Perovskite Photovoltaics}
\begin{columns}
\begin{column}{0.6\columnwidth}
\begin{center}
\includegraphics[width=250pt]{Perovskites/2023-05-15_18-16-32_screenshot.png}
\end{center}
\end{column}

\begin{column}{0.4\columnwidth}
\begin{itemize}
\item Cumulative maximum efficiency of PVs tested by accredited laboratories\autocite{research-2023-best-resear}
\item Perovskite+Si tandem monolithic cells second only to two-junction \(\ce{GaAs}\)
\item Other rising applications
\begin{itemize}
\item LEDs and photodiodes
\item infrared sensors
\item superconductors
\item quantum bits
\end{itemize}
\end{itemize}
\end{column}
\end{columns}
\end{frame}

\begin{frame}[label={sec:orgd39e928}]{What's a Perovskite?}
Notable oxide perovskites and properties circa 2006
\autocite{jiang-2006-predic-lattic}
\begin{columns}
\begin{column}{0.5\columnwidth}
\small
\begin{center}
\begin{tabular}{ll}
<K,Ba>BiO3 & super-conduction\\[0pt]
Pb<Zr,Ti>O3 & piezoelectric action\\[0pt]
Pb<Mb,Mg>O3 & relaxor ferroelectric\\[0pt]
BaTiO3 & dielectric properties\\[0pt]
\end{tabular}
\end{center}
\end{column}
\begin{column}{0.5\columnwidth}
\small
\begin{center}
\begin{tabular}{ll}
Pb<Zr,Ti>O3 & electro-optic\\[0pt]
LaMnO3 & magneto-resistance\\[0pt]
LaCrO3 & catalytic\\[0pt]
BaCeO3 & photonic conductivity\\[0pt]
\end{tabular}
\end{center}
\end{column}
\end{columns}
\begin{figure}[htbp]
\centering
\includegraphics[width=170]{Introduction_To_Perovskites/2022-07-14_15-32-57_screenshot.png}
\caption{Barium Titanate functional ceramic}
\end{figure}
\end{frame}

\begin{frame}[label={sec:org1032f49}]{Halide Perovskites}
\begin{columns}
\begin{column}{0.6\columnwidth}
\begin{itemize}
\item purely inorganic (e.g., CsPbI3)
\item hybrid organic-inorganic (e.g., MAPbBr3)
\begin{description}
\item[{MA}] methylammonium
\item[{FA}] formamidium
\end{description}
\item tailor stability and electronic/optical properties of \(\ce{ABX3}\)
compounds by changing
\begin{itemize}
\item component at each site
\item mix of components
\item atom arrangement
\end{itemize}
\end{itemize}

\begin{table}[htbp]
\caption{\(\ce{ABX3}\) Site Candidates}
\centering
\begin{tabular}{l|llllll}
A-site & MA & FA & Cs & Rb & K & \\[0pt]
B-site & Pb & Sn & Ge & Ba & Sr & Ca\\[0pt]
X-site & I & Br & Cl &  &  & \\[0pt]
\end{tabular}
\end{table}
\end{column}

\begin{column}{0.4\columnwidth}
\begin{center}
\includegraphics[width=140]{hybrid-HaP.png}
\end{center}

\center{}2x1x2 supercell
\end{column}
\end{columns}
\end{frame}

\begin{frame}[label={sec:org2a2c27e}]{Design Goals and Challenges of Combinatorial Mixing}
\begin{columns}
\begin{column}{0.5\columnwidth}
\begin{block}{Search for Improved Compounds}
\alert{37,785} out of \alert{207+ Million} compounds possible in a 2x2x2 supercell
will be examined for solutions to these deficiencies using
\begin{itemize}
\item high-throughput density functional theory
\item data collected from literature
\item multi-fidelity machine learning
\end{itemize}
\end{block}

 
\begin{center}
\includesvg[inkscapeformat=png, inkscapedpi=300,width=150pt]{./.ob-jupyter/3ed5c0e44a1db77d1bd67911c688251d563c2ef0}
\end{center}
\end{column}

\begin{column}{0.5\columnwidth}
 
\begin{center}
\includesvg[inkscapeformat=png, inkscapedpi=300,width=150pt]{./.ob-jupyter/e5dedf8aa3764c0a663c1bbbd2d70eef512cfa96}
\end{center}

\begin{block}{Deficiencies in Leading Perovskite PVs}
inherent instability/short life-cycle
\begin{itemize}
\item degradation in sunlight
\item soluble in water
\end{itemize}
environmental impact
\begin{itemize}
\item toxicity of Lead (Pb)
\end{itemize}
\end{block}
\end{column}
\end{columns}
\end{frame}

\section{Computational Methods In Materials Science}
\label{sec:orgbdff140}
\begin{frame}[label={sec:org934949b}]{Density Functional Theory (DFT)}
Very impractical to explore combinatorial search space with laboratory experiments

VASP \emph{in silico} simulation gives more direct, more systematic, less accurate sampling
\autocite{kresse-1996-effic-iterat,kresse-1996-effic-ab}
\end{frame}

\begin{frame}[label={sec:org0edc18a}]{Machine Learning for Property Prediction}
\begin{columns}
\begin{column}{0.7\columnwidth}
\begin{center}
\includegraphics[width=300]{Computational_Methods_In_Materials_Science/2023-05-24_04-42-21_screenshot.png}
\end{center}
\end{column}


\begin{column}{0.3\columnwidth}
\begin{block}{Applications}
predict anything/make decisions\autocite{pablo-2019-new-front}
\begin{itemize}
\item interatomic force fields
\item predict properties
\item classify observations
\end{itemize}
\end{block}
\end{column}
\end{columns}
\end{frame}

\begin{frame}[label={sec:org238cf05}]{Relevant Property Prediction from Literature}
\begin{center}
\includegraphics[width=.9\linewidth]{Computational_Methods_In_Materials_Science/2023-05-24_05-33-49_screenshot.png}
\end{center}

HSE06 improve band gap predictions compared to PBE\autocite{chan-2010-effic-band}
\end{frame}

\begin{frame}[label={sec:org10d41ce}]{Data Driven Materials Design}
\begin{center}
\includegraphics[width=390]{DDD-flowchart.png}
\end{center}
\center{}\vspace{-0.5cm}Workflow for accelerating testing and development of new PV materials
\autocite{yang-2023-high-throug,pablo-2019-new-front}
\end{frame}

\begin{frame}[label={sec:org4aebd18}]{Random Forest Regression (RFR)}
\begin{columns}
\begin{column}{0.6\columnwidth}
\begin{center}
\includegraphics[width=260]{Computational_Methods_In_Materials_Science/2023-05-23_20-23-36_screenshot.png}
\end{center}
\end{column}

\begin{column}{0.4\columnwidth}
\begin{itemize}
\item trees capture interactions between feature
\item trees are flexible and may be highly biased to data
\item forest explains variance in data
\item forest reduces the variance of error
\end{itemize}
\end{column}
\end{columns}
\end{frame}

\begin{frame}[label={sec:orgcd7bfaa}]{Gaussian Process Regression (GPR)}
\begin{columns}
\begin{column}{0.6\columnwidth}
Universal approximator based on Bayesian analysis.
\begin{block}{Kernel Method}
the similarity function \(k(x, x')\)
\begin{itemize}
\item defines a "universe" of functions
\item defines a density of functions prior to any data
\item \(x\) is any descriptor describable as self-similar
\end{itemize}
\end{block}
\begin{block}{Drawbacks}
\begin{itemize}
\item \(\mathcal{O}(N^3)\) training time complexity
\item kernels require engineering to accommodate prior expectations
\item break down in sparse spaces/high-dimensional spaces
\end{itemize}
\end{block}
\end{column}

\begin{column}{0.4\columnwidth}
\begin{block}{Sample Prior Functions}
 
\begin{center}
\includesvg[width=170pt]{./.ob-jupyter/ec4d60a2109843449bb674d71a67240d13c1611a}
\end{center}
\end{block}

\begin{block}{Sample Posteriors}
 
\begin{center}
\includesvg[width=170pt]{./.ob-jupyter/eeae18f97963b2846566f847fcd1a8bb5dd88175}
\end{center}
\end{block}
\end{column}
\end{columns}
\end{frame}

\begin{frame}[label={sec:org647d55c}]{Sure-Independence Screening and Sparsifying Operator (SISSO)}
\begin{columns}
\begin{column}{0.4\columnwidth}
generalizes "Greedy Pursuit" algorithms
\begin{itemize}
\item Orthogonal Matching Pursuit
\item Basis Pursuit (aka LASSO)
\end{itemize}

SIS+correlation analysis is a form of orthogonalization
\end{column}

\begin{column}{0.6\columnwidth}
\begin{center}
\includegraphics[width=250]{Computational_Methods_In_Materials_Science/2022-12-20_09-09-33_screenshot.png}
\end{center}
Notice SIS re-iterations explain the residual\autocite{ghiringhelli-2017-learn-physic}
\end{column}
\end{columns}
\end{frame}

\section{Data}
\label{sec:org68165c9}
\begin{frame}[label={sec:org40221df}]{Effect of DFT Functionals Levels of Theory}
\begin{center}
\includegraphics[width=.9\linewidth]{Data/2023-05-15_20-04-18_screenshot.png}
\end{center}
\end{frame}

\begin{frame}[label={sec:orgcbff2f4}]{Band Gap Relates to Efficiency}
\begin{columns}
\begin{column}{0.5\columnwidth}
 
\begin{center}
\includesvg[width=210pt]{./.ob-jupyter/2b677464785e9d1d299f0f955a3af16e9c3b5f7e}
\end{center}
\end{column}

\begin{column}{0.5\columnwidth}
 
\begin{center}
\includesvg[width=210pt]{./.ob-jupyter/0c4832ea502cef117df59890aa3341839c46147b}
\end{center}
\end{column}
\end{columns}

SLME (proxy for PCE\autocite{almora-2020-devic-perfor}) accounts for more energetic processes than
Shockley-Queisser criterion (\(bg \approx 1.3\)) allowing selection on range
of bandgaps determined according to level of theory\autocite[p.1]{yu-2012-ident-poten}
\end{frame}

\begin{frame}[label={sec:orgd83d6fd}]{Collecting Data is Hard}
\begin{columns}
\begin{column}{0.7\columnwidth}
experiments take years and, until recently, much labor to collect
\autocite{almora-2020-devic-perfor}
\begin{center}
\includegraphics[width=200]{Data/2023-05-24_08-01-24_screenshot.png}
\end{center}

DFT calculations take multiple
\begin{itemize}
\item hours to converge at PBE level
\item days to converge at HSE level
\item days more to debug
\end{itemize}
accumulating the HaP dataset took years
\autocite{yang-2023-high-throug}
\end{column}

\begin{column}{0.3\columnwidth}
 
\begin{center}
\includesvg[width=120pt]{./.ob-jupyter/55e458fa81a19544b854f8eb0141c77a857bf06b}
\end{center}
\end{column}
\end{columns}
\end{frame}

\begin{frame}[label={sec:org7efbd36},fragile]{Describing a Halide Perovskite}
 \begin{columns}
\begin{column}{0.4\columnwidth}
\begin{block}{14 Dimensional Chemical Vector}
 
\begin{center}
\includesvg[width=170pt]{./.ob-jupyter/3f9092c825b142b1da90f67f4b672476f485574b}
\end{center}
\end{block}
\end{column}

\begin{column}{0.6\columnwidth}
\begin{block}{Parsing compositions using \texttt{cmcl}}
\begin{Code}
\begin{Verbatim}
\color{EFD}\EFk{import} cmcl
\EFv{Y} = load\_codomain\_subset()
\EFv{df} = Y.Formula.to\_frame().ft.comp()
df.\EFv{index} = Y.Formula
\EFb{print}(df)
\end{Verbatim}
\end{Code}

\begin{small}
\begin{verbatim}
                    FA   Pb   Sn    I   MA   Br
Formula                                        
FAPb_0.7Sn_0.3I_3  1.0  0.7  0.3  3.0  NaN  NaN
MAPb(I0.9Br0.1)3   NaN  1.0  NaN  2.7  1.0  0.3
\end{verbatim}
\end{small}
\end{block}
\end{column}
\end{columns}
\end{frame}

\begin{frame}[label={sec:org7736450}]{Sample Space}
\begin{columns}
\begin{column}{0.4\columnwidth}
\begin{itemize}
\item t-SNE projection projects 14D composition vectors to 2D
\item subset of domain has been previously sampled and is available to
Mannodi reaserch group
\autocite{yang-2023-high-throug}
\end{itemize}
\end{column}

\begin{column}{0.6\columnwidth}
\begin{center}
\includegraphics[width=.9\linewidth]{Data/2023-05-19_13-44-52_screenshot.png}
\end{center}
\end{column}
\end{columns}
\end{frame}

\begin{frame}[label={sec:orgfc3d793}]{Constituent Properties}
\begin{block}{Twelve Properties per \(\ce{ABX_3}\) constituent}
\begin{columns}
\begin{column}{0.5\columnwidth}
\begin{itemize}
\item Ionic Radius
\item Boiling Temperature
\item Melting Temperature
\item Density
\item Atomic Weight
\item Electron Affinity
\end{itemize}
\end{column}
\begin{column}{0.5\columnwidth}
\begin{itemize}
\item Ionization Energy
\item Heat of Fusion
\item Heat of Vaporization
\item Electronegativity
\item Atomic Number
\item Period
\end{itemize}
\end{column}
\end{columns}
\end{block}

compute by averaging elemental properties obtained from Mendeleev database
\autocite{mentel-2014}
by atomic weight of constituent in compount
\end{frame}

\begin{frame}[label={sec:org2e43c1e}]{Bivariate Analysis of Functional Effects}
 
\begin{center}
\includesvg[width=400pt]{./.ob-jupyter/936ea994119ed4e82dc02a3d1b1fd48bf89083da}
\end{center}

band gap at higher levels of theory correlate with composition differently\autocite{yang-2023-high-throug}
\end{frame}

\section{Multi-Fidelity Learning}
\label{sec:org3ad1eb6}
\begin{frame}[label={sec:orgd56814f}]{Multi-fidelity Machine Learning}
\[
\hat{y}_i = y_i + \epsilon_i \mbox{ for observations } i=1,…,N
\]

\begin{description}
\item[{high fidelity}] prior on variance is narrow
\item[{low fidelity}] prior on variance is wide
\end{description}

\begin{block}{Training Strategy}
\begin{itemize}
\item oversampling
\item weight scores to reward models faithful to higher fidelities
\item sequential learning/delta learning
\item co-Kriging
\end{itemize}
\end{block}
\end{frame}

\begin{frame}[label={sec:orge7bf990}]{Multi-Fidelity Model Development Workflow}
\begin{center}
\includegraphics[width=300]{/home/panos/Pictures/flowcharts/ML_pipe.png}
\end{center}

Weightable Machine Learning pipeline implemented in Python
\autocite{manganaris-2022-mrs-comput}
\end{frame}

\section{Models}
\label{sec:org2928124}
\begin{frame}[label={sec:org9b452ea}]{Compare Three Methods}
 
\begin{center}
\includesvg[inkscapeformat=png, inkscapedpi=300,width=450pt]{./.ob-jupyter/f135c78f3459b792a3436307ab575dd3e1669074}
\end{center}
\end{frame}

\begin{frame}[label={sec:org4969afb}]{SHAP Analyze RFR}
\begin{center}
\includegraphics[width=.9\linewidth]{Models/2023-05-19_13-42-16_screenshot.png}
\end{center}
\end{frame}

\section{Results}
\label{sec:orgf9002d7}
\begin{frame}[label={sec:org8f9edb9}]{Make Predictions on Domain Subset}
\begin{columns}
\begin{column}{0.4\columnwidth}
\begin{itemize}
\item desired range \(bg \in [1, 2.5]\) eV
\end{itemize}
\end{column}
\begin{column}{0.6\columnwidth}
\begin{center}
\includegraphics[width=200]{Results/2023-05-19_13-23-47_screenshot.png}
\end{center}
\end{column}
\end{columns}
\end{frame}

\begin{frame}[label={sec:org177a289}]{Tolerance Factor}
\begin{block}{Perovskite Stability}
Stability of high symmetry \(\alpha\) phase depends on the ABX member's
ionic radii satisfying Goldschmidt's tolerance:
\autocite{yin-2015-halid-perov}

\[
1 \approx t = \frac{R_A+R_X}{\sqrt{2}*(R_B+R_X)}
\]

Empirically, it is \alert{necessary} that A must be much larger than B
for Perovskite formation. B is usually a large atom (e.g. Pb, Sn) in
Halide Perovskites. So A is suitably occupied by a Molecule -- an
amine -- with an effective ionic radius.
\autocite{kieslich-2015-exten-toler}
\end{block}
\end{frame}

\begin{frame}[label={sec:org232b1fa}]{Other Tolerance Factors in Literature}
\begin{block}{Octahedral Factor}
\[
o=\frac{r_B}{r_X}
\]
\end{block}
\begin{block}{Bartel Tolerance}
\[
t_{Bartel}=\frac{r_X}{r_B}-[1-\frac{\frac{r_A}{r_B}}{ln(\frac{r_A}{r_B})}]
\]
proposed recently
\autocite{bartel-2019-new-toler}
\end{block}
\begin{block}{Ionic Radii}
\begin{itemize}
\item Elemental Radii are well documented in public databases
\item Molecular Radii are defined in various ways, mostly for small
molecules.
\end{itemize}
\end{block}
\end{frame}

\begin{frame}[label={sec:orgb21ad6c}]{Screening Yields}
\begin{columns}
\begin{column}{0.3\columnwidth}
\begin{center}
\includegraphics[width=120]{screening_ops.png}
\end{center}
\end{column}
\begin{column}{0.7\columnwidth}
\begin{center}
\includegraphics[width=260]{Results/2023-05-19_13-20-09_screenshot.png}
\end{center}
\end{column}
\end{columns}
\end{frame}

\begin{frame}[label={sec:org299c7b5}]{Screened Constituent Frequencies}
\begin{center}
\includegraphics[width=400]{Results/2023-05-19_13-07-38_screenshot.png}
\end{center}
\begin{columns}
\begin{column}{0.3\columnwidth}
\begin{itemize}
\item prefer occupying small fraction mixing in most constituents
\item \(\ce{K}\) and \(\ce{Rb}\) also prefer full occupancy
\end{itemize}
\end{column}

\begin{column}{0.3\columnwidth}
\begin{itemize}
\item favor full occupation a rate of 5-8\%
\item showed some preference approaching doping
\end{itemize}
\end{column}

\begin{column}{0.3\columnwidth}
\begin{itemize}
\item strongly prefer full occupation
\end{itemize}
\end{column}
\end{columns}
\end{frame}

\begin{frame}[label={sec:orgd74661b}]{Interpretation}
\begin{columns}
\begin{column}{0.5\columnwidth}
\begin{center}
\includegraphics[width=200]{Results/2023-05-19_13-21-13_screenshot.png}
\end{center}
\end{column}

\begin{column}{0.5\columnwidth}
\begin{center}
\includegraphics[width=200]{Results/2023-05-19_13-20-09_screenshot.png}
\end{center}
\end{column}
\end{columns}
\end{frame}

\section{Summary and Conclusions}
\label{sec:org51ca0ed}
\begin{frame}[label={sec:org5405d17},fragile]{Review of Work}
 \begin{itemize}
\item preexisting data set of \textasciitilde{}500 samples from domain of \textasciitilde{}40000 single-site mixed compositions
\item extract features using \texttt{cmcl}
\item train models of band gaps measured at multiple fidelities
\item ascertain best model
\item analyze model accuracy, interpretability of bias and variance
\item make predictions on whole sample domain
\item Screen for viable candidate compositions
\end{itemize}
\end{frame}

\begin{frame}[label={sec:org45dd8db}]{Conclusions}
\begin{itemize}
\item RFR yields the best model because
\begin{itemize}
\item it captures relevant feature interactions as corroborated by Pearson correlations
\item it flexibly represents nonlinear relationships between feature interactions and band gap
\item as an ensemble model, it reduces the variance of error
\end{itemize}
\end{itemize}
\end{frame}

\section{Credit}
\label{sec:org290ce3a}
\begin{frame}[label={sec:org6037eec}]{Acknowledgements}
\begin{columns}
\begin{column}{0.6\columnwidth}
\begin{block}{Research Group}
I am grateful to
\begin{itemize}
\item Professor Arun Mannodi-Kanakkithodi for his mentorship, support, and knowledge.
\item Professor Erk for her guidance and council
\item Professor Strachan for his instruction
\item Jiaqi Yang for his collaboration
\item Habibur Rahman for his friendship and curiosity
\end{itemize}
\end{block}
\begin{block}{Funding}
\begin{itemize}
\item Ross Fellowship
\item Professor Arun's startup research grant
\end{itemize}
\end{block}
\begin{block}{Resources}
RCAC Bell Computing Cluster
\end{block}
\end{column}
\begin{column}{0.4\columnwidth}
\hspace*{-1cm}
\includegraphics[width=200]{/home/panos/Pictures/mrg-photoshoot/41_Kanakkithodi.jpg}
\end{column}
\end{columns}
\end{frame}
\section{Bib}
\label{sec:org7c4125c}
\begin{frame}[allowframebreaks]{References}
\AtNextBibliography{\tiny}
\printbibliography
\end{frame}
\appendix
\begin{frame}[label={sec:orgf5ebe74}]{Data Pre-Processing}
\begin{center}
\includegraphics[width=300]{/home/panos/Pictures/flowcharts/data_proc.png}
\end{center}
Data pre-processing Workflow to Implement with Python Pandas
\end{frame}
\begin{frame}[label={sec:orgdfdab8e}]{DFT Functionals Effect Band Structure Calculation in HaPs}
\begin{itemize}
\item D3
\item SOC
\item PBE semi-local vs HSE hybrid
\begin{itemize}
\item (effects on electronic structure)
\item computational heft
\end{itemize}
\end{itemize}
\end{frame}
\end{document}