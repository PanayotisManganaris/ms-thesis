% Created 2023-05-15 Mon 20:25
% Intended LaTeX compiler: pdflatex
\documentclass[10pt, aspectratio=169, presentation]{beamer}
\usepackage[utf8]{inputenc}
\usepackage[T1]{fontenc}
\usepackage{graphicx}
\usepackage{longtable}
\usepackage{wrapfig}
\usepackage{rotating}
\usepackage[normalem]{ulem}
\usepackage{amsmath}
\usepackage{amssymb}
\usepackage{capt-of}
\usepackage{hyperref}
\usepackage{ptm}

%\setbeamertemplate{caption}[numbered]
\beamertemplatenavigationsymbolsempty
%
\usepackage{mathtools}%% bugfixing and additional tools for amsmath
\usepackage{amsfonts}%% formal math fonts
\usepackage{physics}%% powerful mathematics shorthand
\usepackage{bm}%% bold math
\usepackage{gensymb}%% inter-environment consistent measurment unit symbols
\usepackage{upgreek}%% easy lower and uppercase nonitalicized greek letters
\usepackage[version=4]{mhchem}%% for writing chemical formulae
\usepackage{siunitx}[=v2]%% version 2 workaround first-frame artifact
%
\usepackage[%
%citestyle=authoryear-icomp,
%bibstyle=authoryear-icomp,
style=authortitle-tcomp,
%style=verbose,
autocite=footnote,
hyperref=true,
backref=true,
maxcitenames=3,
url=true,
backend=biber,
natbib=true,
]{biblatex}
\addbibresource{/home/panos/Documents/masters/local-bib.bib}
\addbibresource{~/org/bibliotex/bibliotex.bib}
%
\usepackage{svg}
\svgsetup{%
inkscapelatex=false,
inkscapeformat=png,
inkscapedpi=300,
}
\svgpath{{/home/panos/Pictures/figs}}
%
\setbeamertemplate{footline}{%
\hfill
\vspace*{0.1cm}
\insertframenumber\slash\inserttotalframenumber
\hspace*{0.2cm}
}
\institute[Mannodi Group]{
\inst{1} Purdue Materials Engineering\\Professor Arun Mannodi-Kanakkithodi
}
\usetheme{default}
\author{Panayotis Manganaris\inst{1}}
\date{\today}
\title{Machine Learning Perovskite Bandgaps for Improved Photovoltaics}
\hypersetup{
 pdfauthor={Panayotis Manganaris\inst{1}},
 pdftitle={Machine Learning Perovskite Bandgaps for Improved Photovoltaics},
 pdfkeywords={},
 pdfsubject={},
 pdfcreator={Emacs 28.2 (Org mode 9.5.5)}, 
 pdflang={English}}
\begin{document}

\maketitle
\begin{frame}{Outline}
\tableofcontents
\end{frame}

\section{Background}
\label{sec:orgb2331fb}
\begin{frame}[label={sec:orgd7f75f9}]{Photovoltaic Power Generation}
\begin{itemize}
\item visible spectrum solar absorbers
\item semiconductors
\item efficiency
\end{itemize}
\end{frame}
\begin{frame}[label={sec:orgf8abbd6}]{Rapid Rise of Perovskite PVs}
\begin{columns}
\begin{column}{0.6\columnwidth}
\begin{center}
\includegraphics[width=250pt]{Perovskites/2023-05-15_18-16-32_screenshot.png}
\end{center}
\end{column}

\begin{column}{0.4\columnwidth}
\begin{itemize}
\item Cumulative maximum efficiency of PVs tested by accredited laboratories\autocite{research-2023-best-resear}
\item Perovskite+Si tandem monolithic cells second only to two-junction \(\ce{GaAs}\)
\end{itemize}
\end{column}
\end{columns}
\end{frame}

\begin{frame}[label={sec:orged427e8}]{What's a Perovskite?}
Notable oxide perovskites and properties circa 2006
\autocite{jiang-2006-predic-lattic}
\begin{columns}
\begin{column}{0.5\columnwidth}
\small
\begin{center}
\begin{tabular}{ll}
<K,Ba>BiO3 & super-conduction\\[0pt]
Pb<Zr,Ti>O3 & piezoelectric action\\[0pt]
Pb<Mb,Mg>O3 & relaxor ferroelectric\\[0pt]
BaTiO3 & dielectric properties\\[0pt]
\end{tabular}
\end{center}
\end{column}
\begin{column}{0.5\columnwidth}
\small
\begin{center}
\begin{tabular}{ll}
Pb<Zr,Ti>O3 & electro-optic\\[0pt]
LaMnO3 & magneto-resistance\\[0pt]
LaCrO3 & catalytic\\[0pt]
BaCeO3 & photonic conductivity\\[0pt]
\end{tabular}
\end{center}
\end{column}
\end{columns}
\begin{figure}[htbp]
\centering
\includegraphics[width=170]{Introduction_To_Perovskites/2022-07-14_15-32-57_screenshot.png}
\caption{Barium Titanate functional ceramic}
\end{figure}
\end{frame}

\begin{frame}[label={sec:org122a40a}]{Halide Perovskites}
\begin{columns}
\begin{column}{0.5\columnwidth}
\begin{block}{Solar Energy Harvesting}
\begin{itemize}
\item 2009 : \(\ce{MAPbI_3}\) dye-sensitized absorber with 3.8\% PCE
\autocite{kojima-2009-organ-halid}
\item 2011 : absorber efficiency doubles
\autocite{im-2011}
\item 2013 : \(\ce{MAPbX_3}\) found to transport holes+electrons
\autocite{saliba-2014-influen-therm}
\end{itemize}

HaP PVs become competitive in 2013
\end{block}
\end{column}
\begin{column}{0.5\columnwidth}
\begin{center}
\includegraphics[width=.9\linewidth]{hybrid-HaP.png}
\end{center}
\end{column}
\end{columns}
\end{frame}

\section{Machine Learning In Materials Science}
\label{sec:org0931430}
\begin{frame}[label={sec:org964a9a9}]{Data Driven Materials Design}
\begin{figure}[htbp]
\centering
\includegraphics[width=390]{DDD-flowchart.png}
\caption{Workflow for accelerating testing and development of new PV materials\autocite{yang-2023-high-throug,pablo-2019-new-front}}
\end{figure}
\end{frame}

\begin{frame}[label={sec:org2e28c9b}]{Machine Learning In Materials Science}
An increasingly popular technique for accelerating discovery and
development\autocite{pablo-2019-new-front}
\begin{itemize}
\item force fields
\item property-prediction
\item building on dft
\end{itemize}
\end{frame}
\begin{frame}[label={sec:org8e08a8c}]{Regression}
\end{frame}

\begin{frame}[label={sec:orgcab7d15}]{Learning Algorithms}
\begin{itemize}
\item Decision Tree
\item Random Forest
\item Linear
\item SISSO
\item Gaussian Process
\end{itemize}
\end{frame}

\section{Data}
\label{sec:orgf869c1b}
\begin{frame}[label={sec:org49e7695}]{Band Gap Relates to Efficiency}
\begin{columns}
\begin{column}{0.5\columnwidth}
 
\begin{center}
\includesvg[width=210pt]{./.ob-jupyter/2b677464785e9d1d299f0f955a3af16e9c3b5f7e}
\end{center}
\end{column}

\begin{column}{0.5\columnwidth}
 
\begin{center}
\includesvg[width=210pt]{./.ob-jupyter/0c4832ea502cef117df59890aa3341839c46147b}
\end{center}
\end{column}
\end{columns}

SLME metric accounts for more energetic processes than
Shockley-Queisser criterion (\(bg \approx 1.3\)) allowing selection on range
of bandgaps determined according to level of theory\autocite[p.1]{yu-2012-ident-poten}
\end{frame}

\begin{frame}[label={sec:org6dc9425}]{Effect of DFT Functionals Levels of Theory}
\begin{center}
\includegraphics[width=.9\linewidth]{Data/2023-05-15_20-04-18_screenshot.png}
\end{center}
\end{frame}

\begin{frame}[label={sec:orgfdb3599}]{Variation in Band Gap changes with Level of Theory}
 
\begin{center}
\includesvg[width=300pt]{./.ob-jupyter/e044073b053593cbfe267508875c51401afb43a3}
\end{center}
\end{frame}

\begin{frame}[label={sec:orgabe6329}]{Features of Perovskites Descriptive of Band Gap}
\begin{block}{composition vectors}
\end{block}
\end{frame}

\begin{frame}[label={sec:org991223b}]{Sample Space}
\end{frame}

\begin{frame}[label={sec:org2e55e52}]{Constituent Properties}
\begin{block}{Twelve Properties per \(\ce{ABX_3}\) constituent}
\begin{columns}
\begin{column}{0.5\columnwidth}
\begin{itemize}
\item Ionic Radius
\item Boiling Temperature
\item Melting Temperature
\item Density
\item Atomic Weight
\item Electron Affinity
\end{itemize}
\end{column}
\begin{column}{0.5\columnwidth}
\begin{itemize}
\item Ionization Energy
\item Heat of Fusion
\item Heat of Vaporization
\item Electronegativity
\item Atomic Number
\item Period
\end{itemize}
\end{column}
\end{columns}
\end{block}
\end{frame}

\begin{frame}[label={sec:org3f2baf5}]{bivariate analysis}
\begin{itemize}
\item correlations can be surprising
\item some confirm physical expectations
\end{itemize}
\end{frame}

\section{Multi-Fidelity Learning}
\label{sec:org6599713}
\begin{frame}[label={sec:org19c5349}]{Challenges of Learning from}
\end{frame}
\begin{frame}[label={sec:org8196b23}]{multi-fidelity model development workflow}
\alert{flowchart}

simultaneously utilize multiple measurements made of the same
composition in training models.\autocite{manganaris-2022-mrs-comput}
\end{frame}


\section{Summary}
\label{sec:org69935b8}
\begin{frame}[label={sec:org4e5e961}]{review everything}
\end{frame}
\section{Credit}
\label{sec:org8af9a23}
\begin{frame}[label={sec:orgc2cf4d0}]{Acknowledgements}
\begin{columns}
\begin{column}{0.6\columnwidth}
\begin{block}{Research Group}
I am greatful to
\begin{itemize}
\item Professor Arun Mannodi-Kanakkithodi for his mentorship and knowledge.
\item Jiaqi Yang for his collaboration
\item Habibur Rahman for his friendship and curiosity
\end{itemize}
\end{block}
\begin{block}{Funding}
\begin{itemize}
\item Ross Fellowship
\item Professor Arun's startup research grant F.10023800.05.002
\end{itemize}
\end{block}
\begin{block}{Resources}
RCAC Bell Computing Cluster
\end{block}
\end{column}
\begin{column}{0.4\columnwidth}
\hspace*{-1cm}
\includegraphics[width=200]{/home/panos/Pictures/mrg-photoshoot/41_Kanakkithodi.jpg}
\end{column}
\end{columns}
\end{frame}
\section{Bib}
\label{sec:orgc074de6}
\begin{frame}[allowframebreaks]{References}
\AtNextBibliography{\tiny}
\printbibliography
\end{frame}
\appendix
\begin{frame}[label={sec:orga9e2c3c}]{Data Pre-Processing}
\begin{center}
\includegraphics[width=300]{/home/panos/Pictures/flowcharts/data_proc.png}
\end{center}
Data pre-processing Workflow to Implement with Python Pandas
\end{frame}
\begin{frame}[label={sec:org70807f8}]{Machine Learning Pipeline}
\begin{center}
\includegraphics[width=300]{/home/panos/Pictures/flowcharts/ML_pipe.png}
\end{center}
Machine Learning Pipeline to Implement with Python SciKit-Learn
\end{frame}
\begin{frame}[label={sec:org075e190}]{DFT Functionals Effect Band Structure Calculation in HaPs}
\begin{itemize}
\item D3
\item SOC
\item PBE semi-local vs HSE hybrid
\begin{itemize}
\item (effects on electronic structure)
\item computational heft
\end{itemize}
\end{itemize}
\end{frame}
\end{document}