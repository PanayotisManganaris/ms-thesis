% Created 2023-06-07 Wed 19:27
% Intended LaTeX compiler: pdflatex
\documentclass[10pt, aspectratio=169, presentation]{beamer}
\usepackage[utf8]{inputenc}
\usepackage[T1]{fontenc}
\usepackage{graphicx}
\usepackage{longtable}
\usepackage{wrapfig}
\usepackage{rotating}
\usepackage[normalem]{ulem}
\usepackage{amsmath}
\usepackage{amssymb}
\usepackage{capt-of}
\usepackage{hyperref}
\usepackage{ptm}
\usepackage{braket}
%\setbeamertemplate{caption}[numbered]
\beamertemplatenavigationsymbolsempty
%
\usepackage{mathtools}%% bugfixing and additional tools for amsmath
\usepackage{amsfonts}%% formal math fonts
\usepackage{physics}%% powerful mathematics shorthand
\usepackage{bm}%% bold math
\usepackage{gensymb}%% inter-environment consistent measurment unit symbols
\usepackage{upgreek}%% easy lower and uppercase nonitalicized greek letters
\usepackage[version=4]{mhchem}%% for writing chemical formulae
\usepackage{siunitx}[=v2]%% version 2 workaround first-frame artifact
%
\usepackage[%
citestyle=authoryear-icomp,
bibstyle=verbose,
ibidpage=true,
maxcitenames=3,
maxbibnames=99,
autocite=footnote,
hyperref=true,
backref=true,
backend=biber,
natbib=true,
]{biblatex}
\addbibresource{/home/panos/Documents/masters/local-bib.bib}
\addbibresource{~/org/bibliotex/bibliotex.bib}
%
\renewbibmacro*{cite}{%
\iffieldundef{shorthand}
{\ifthenelse{\ifnameundef{labelname}\OR\iffieldundef{labelyear}}
{\usebibmacro{cite:label}%
\setunit{\printdelim{nonameyeardelim}}}
{\printnames{labelname}%
\setunit{\printdelim{nameyeardelim}}}%
\usebibmacro{cite:labeldate+extradate}%
\setunit{\addcomma\space}%
\usebibmacro{journal}
\setunit{\addcomma\space}%
\printfield{note}}
{\usebibmacro{cite:shorthand}}}
% make footnotes smaller
\renewcommand{\footnotesize}{\fontfamily{phv}\selectfont\scriptsize}
%
\usepackage{svg}
\svgsetup{%
inkscapelatex=false,
inkscapeformat=png,
inkscapedpi=300,
}
\svgpath{{/home/panos/Pictures/figs}}
%
\setbeamertemplate{footline}{%
\hfill
\vspace*{0.1cm}
\insertframenumber\slash\inserttotalframenumber
\hspace*{0.2cm}
}
\institute[Mannodi Group]{\large{}
\inst{1} Purdue Materials Engineering\\Advisor Arun Mannodi-Kanakkithodi
}
\AtBeginSection[]
{
\begin{frame}
\frametitle{Table of Contents}
\tableofcontents[currentsection]
\end{frame}
}
\usetheme{default}
\author{\large{}Panayotis Manganaris\inst{1}}
\date{\today}
\title{\Huge{}Machine Learning Perovskite Bandgaps for Improved Photovoltaics}
\hypersetup{
 pdfauthor={\large{}Panayotis Manganaris\inst{1}},
 pdftitle={\Huge{}Machine Learning Perovskite Bandgaps for Improved Photovoltaics},
 pdfkeywords={},
 pdfsubject={},
 pdfcreator={Emacs 28.2 (Org mode 9.5.5)}, 
 pdflang={English}}

% Setup for code blocks [1/2]

\usepackage{fvextra}

\fvset{%
  commandchars=\\\{\},
  highlightcolor=white!95!black!80!blue,
  breaklines=true,
  breaksymbol=\color{white!60!black}\tiny\ensuremath{\hookrightarrow}}

% Make line numbers smaller and grey.
\renewcommand\theFancyVerbLine{\footnotesize\color{black!40!white}\arabic{FancyVerbLine}}

\usepackage{xcolor}

% In case engrave-faces-latex-gen-preamble has not been run.
\providecolor{EfD}{HTML}{f7f7f7}
\providecolor{EFD}{HTML}{28292e}

% Define a Code environment to prettily wrap the fontified code.
\usepackage[breakable,xparse]{tcolorbox}
\DeclareTColorBox[]{Code}{o}%
{colback=EfD!98!EFD, colframe=EfD!95!EFD,
  fontupper=\footnotesize\setlength{\fboxsep}{0pt},
  colupper=EFD,
  IfNoValueTF={#1}%
  {boxsep=2pt, arc=2.5pt, outer arc=2.5pt,
    boxrule=0.5pt, left=2pt}%
  {boxsep=2.5pt, arc=0pt, outer arc=0pt,
    boxrule=0pt, leftrule=1.5pt, left=0.5pt},
  right=2pt, top=1pt, bottom=0.5pt,
  breakable}

% Support listings with captions
\usepackage{float}
\floatstyle{plain}
\newfloat{listing}{htbp}{lst}
\newcommand{\listingsname}{Listing}
\floatname{listing}{\listingsname}
\newcommand{\listoflistingsname}{List of Listings}
\providecommand{\listoflistings}{\listof{listing}{\listoflistingsname}}


% Setup for code blocks [2/2]: syntax highlighting colors

\newcommand\efstrut{\vrule height 2.1ex depth 0.8ex width 0pt}
\definecolor{EFD}{HTML}{000000}
\definecolor{EfD}{HTML}{ffffff}
\newcommand{\EFD}[1]{\textcolor{EFD}{#1}} % default
\definecolor{EFh}{HTML}{7f7f7f}
\newcommand{\EFh}[1]{\textcolor{EFh}{#1}} % shadow
\definecolor{EFsc}{HTML}{228b22}
\newcommand{\EFsc}[1]{\textcolor{EFsc}{\textbf{#1}}} % success
\definecolor{EFw}{HTML}{ff8e00}
\newcommand{\EFw}[1]{\textcolor{EFw}{\textbf{#1}}} % warning
\definecolor{EFe}{HTML}{ff0000}
\newcommand{\EFe}[1]{\textcolor{EFe}{\textbf{#1}}} % error
\definecolor{EFc}{HTML}{b22222}
\newcommand{\EFc}[1]{\textcolor{EFc}{#1}} % font-lock-comment-face
\definecolor{EFcd}{HTML}{b22222}
\newcommand{\EFcd}[1]{\textcolor{EFcd}{#1}} % font-lock-comment-delimiter-face
\definecolor{EFs}{HTML}{8b2252}
\newcommand{\EFs}[1]{\textcolor{EFs}{#1}} % font-lock-string-face
\definecolor{EFd}{HTML}{8b2252}
\newcommand{\EFd}[1]{\textcolor{EFd}{#1}} % font-lock-doc-face
\definecolor{EFm}{HTML}{008b8b}
\newcommand{\EFm}[1]{\textcolor{EFm}{#1}} % font-lock-doc-markup-face
\definecolor{EFk}{HTML}{9370db}
\newcommand{\EFk}[1]{\textcolor{EFk}{#1}} % font-lock-keyword-face
\definecolor{EFb}{HTML}{483d8b}
\newcommand{\EFb}[1]{\textcolor{EFb}{#1}} % font-lock-builtin-face
\definecolor{EFf}{HTML}{0000ff}
\newcommand{\EFf}[1]{\textcolor{EFf}{#1}} % font-lock-function-name-face
\definecolor{EFv}{HTML}{a0522d}
\newcommand{\EFv}[1]{\textcolor{EFv}{#1}} % font-lock-variable-name-face
\definecolor{EFt}{HTML}{228b22}
\newcommand{\EFt}[1]{\textcolor{EFt}{#1}} % font-lock-type-face
\definecolor{EFo}{HTML}{008b8b}
\newcommand{\EFo}[1]{\textcolor{EFo}{#1}} % font-lock-constant-face
\definecolor{EFwr}{HTML}{ff0000}
\newcommand{\EFwr}[1]{\textcolor{EFwr}{\textbf{#1}}} % font-lock-warning-face
\newcommand{\EFnc}[1]{#1} % font-lock-negation-char-face
\definecolor{EFpp}{HTML}{483d8b}
\newcommand{\EFpp}[1]{\textcolor{EFpp}{#1}} % font-lock-preprocessor-face
\newcommand{\EFrc}[1]{\textbf{#1}} % font-lock-regexp-grouping-construct
\newcommand{\EFrb}[1]{\textbf{#1}} % font-lock-regexp-grouping-backslash
\newcommand{\EFob}[1]{#1} % org-block
\definecolor{EFhn}{HTML}{008b8b}
\newcommand{\EFhn}[1]{\textcolor{EFhn}{#1}} % highlight-numbers-number
\definecolor{EFhq}{HTML}{9370db}
\newcommand{\EFhq}[1]{\textcolor{EFhq}{#1}} % highlight-quoted-quote
\definecolor{EFhs}{HTML}{008b8b}
\newcommand{\EFhs}[1]{\textcolor{EFhs}{#1}} % highlight-quoted-symbol
\definecolor{EFrda}{HTML}{707183}
\newcommand{\EFrda}[1]{\textcolor{EFrda}{#1}} % rainbow-delimiters-depth-1-face
\definecolor{EFrdb}{HTML}{7388d6}
\newcommand{\EFrdb}[1]{\textcolor{EFrdb}{#1}} % rainbow-delimiters-depth-2-face
\definecolor{EFrdc}{HTML}{909183}
\newcommand{\EFrdc}[1]{\textcolor{EFrdc}{#1}} % rainbow-delimiters-depth-3-face
\definecolor{EFrdd}{HTML}{709870}
\newcommand{\EFrdd}[1]{\textcolor{EFrdd}{#1}} % rainbow-delimiters-depth-4-face
\definecolor{EFrde}{HTML}{907373}
\newcommand{\EFrde}[1]{\textcolor{EFrde}{#1}} % rainbow-delimiters-depth-5-face
\definecolor{EFrdf}{HTML}{6276ba}
\newcommand{\EFrdf}[1]{\textcolor{EFrdf}{#1}} % rainbow-delimiters-depth-6-face
\definecolor{EFrdg}{HTML}{858580}
\newcommand{\EFrdg}[1]{\textcolor{EFrdg}{#1}} % rainbow-delimiters-depth-7-face
\definecolor{EFrdh}{HTML}{80a880}
\newcommand{\EFrdh}[1]{\textcolor{EFrdh}{#1}} % rainbow-delimiters-depth-8-face
\definecolor{EFrdi}{HTML}{887070}
\newcommand{\EFrdi}[1]{\textcolor{EFrdi}{#1}} % rainbow-delimiters-depth-9-face
\begin{document}

\maketitle
\begin{frame}{Outline}
\tableofcontents
\end{frame}

\section{Devices and Perovskites Background}
\label{sec:org2b7a124}
\begin{frame}[label={sec:orgfd91270}]{Photovoltaic (PV) Power Generation}
\begin{columns}
\begin{column}{0.3\columnwidth}
\begin{center}
\includegraphics[width=90]{thin-film-PV.png}
\end{center}
\begin{center}
\includegraphics[width=90]{Si-HaP-tandem-PV.png}
\end{center}

\tiny{}\center{}Perovskite Applications\footnote{\tiny{}\href{https://www.lesker.com/newweb/ped/applications/perovskite-research.cfm }{www.lesker.com}}
\end{column}

\begin{column}{0.7\columnwidth}
\begin{block}{Band Gap (bg) Effects on Absorption Spectrum}
\begin{columns}
\begin{column}{0.5\columnwidth}
 
\begin{center}
\includesvg[width=130pt]{./.ob-jupyter/e44c0e6b821bb2eea4fc6ed321c93e99352b8192}
\end{center}
\end{column}

\begin{column}{0.5\columnwidth}
\begin{align*}
E_\gamma = \hbar\omega &= bg\,\si{\joule} \\
\implies \frac{hc}{628\,\si{\nano\meter}} &= 3.16\times{}10^{-19}\,\si{\joule}\\
&= 1.97\,\si{\electronvolt}
\end{align*}
\end{column}
\end{columns}
\end{block}

\begin{block}{Perovskite Semiconductors in Solar Energy Harvesting}
\begin{itemize}
\item 2009 : \(\ce{MAPbI_3}\) dye-sensitized absorber with 3.8\% PCE
\autocite{kojima-2009-organ-halid}
\item 2011 : absorber efficiency doubles
\autocite{im-2011}
\item 2013 : \(\ce{MAPbX_3}\) found to transport holes+electrons
\autocite{saliba-2014-influen-therm}
\end{itemize}
\end{block}
\end{column}
\end{columns}
\end{frame}
\begin{frame}[label={sec:org799d2cc}]{Rapid Rise of Perovskite Photovoltaics}
\begin{columns}
\begin{column}{0.6\columnwidth}
\begin{center}
\includegraphics[width=250pt]{Perovskites/2023-05-15_18-16-32_screenshot.png}
\end{center}
\end{column}

\begin{column}{0.4\columnwidth}
\begin{itemize}
\item Cumulative maximum efficiency of PVs tested by accredited laboratories
\autocite{research-2023-best-resear}
\item Perovskite+Si tandem monolithic cells second only to two-junction \(\ce{GaAs}\)
\item Other rising applications
\begin{itemize}
\item LEDs and photodiodes
\item infrared sensors
\item superconductors
\item quantum bits
\end{itemize}
\end{itemize}
\end{column}
\end{columns}
\end{frame}

\begin{frame}[label={sec:org25984b2}]{Halide Perovskites}
\begin{columns}
\begin{column}{0.6\columnwidth}
\begin{itemize}
\item purely inorganic (e.g., CsPbI3)
\item hybrid organic-inorganic (e.g., MAPbBr3)
\begin{description}
\item[{MA}] methylammonium \(\ce{CH3NH3+}\)
\item[{FA}] formamidium \(\ce{CH(NH2)2+}\)
\end{description}
\item tailor stability and electronic/optical properties of \(\ce{ABX3}\)
compounds by changing
\begin{itemize}
\item component at each site
\item mix of components
\item atom arrangement
\end{itemize}
\end{itemize}

\begin{table}[htbp]
\caption{\(\ce{ABX3}\) Site Candidates}
\centering
\begin{tabular}{l|llllll}
A-site & MA & FA & Cs & Rb & K & \\[0pt]
B-site & Pb & Sn & Ge & Ba & Sr & Ca\\[0pt]
X-site & I & Br & Cl &  &  & \\[0pt]
\end{tabular}
\end{table}
\end{column}

\begin{column}{0.4\columnwidth}
\begin{center}
\includegraphics[width=140]{hybrid-HaP.png}
\end{center}

\center{}2x2x2 \(\alpha\)-phase supercell\break \alert{with MA at A-site}
\end{column}
\end{columns}
\end{frame}

\begin{frame}[label={sec:org88fcc44}]{Perovskite Formability}
\begin{itemize}
\item A must be much larger than B for Perovskite formation
\item Various tolerance factors have been designed to describe this constraint
\item B usually large (e.g. Pb, Sn) in Halide Perovskites
\item A accommodates \alert{molecules} with large effective ionic radii\autocite{kieslich-2015-exten-toler}
\end{itemize}

The following definitions approximate \(\alpha\) phase stability
\autocite{yin-2015-halid-perov,bartel-2019-new-toler}
\begin{columns}
\begin{column}{0.33\columnwidth}
\begin{block}{Goldschmidt Tolerance}
\[
t = \frac{R_A+R_X}{\sqrt{2}*(R_B+R_X)}
\]

\[
0.813 < t < 1.107
\]
\end{block}
\end{column}

\begin{column}{0.33\columnwidth}
\begin{block}{Octahedral Factor}
\[
o=\frac{r_B}{r_X}
\]

\[
0.442 < o < 0.895
\]
\end{block}
\end{column}

\begin{column}{0.33\columnwidth}
\begin{block}{Bartel Tolerance}
\[
\tau = \frac{r_X}{r_B}-\left[ 1-\frac{\frac{r_A}{r_B}}{ln(\frac{r_A}{r_B})} \right]
\]

\[
\tau < 4.18
\]
\end{block}
\end{column}
\end{columns}
\end{frame}

\begin{frame}[label={sec:orgb696d80}]{Spectroscopic Limited Maximum Efficiency (SLME)}
SLME accounts for more energetic processes than Shockley-Queisser
criterion (\(bg \approx 1.3\)) allowing selection on range of bandgaps
determined according to level of theory\autocite[p.1]{yu-2012-ident-poten}

\[
a(E)=1-e^{-2\alpha(E)L}
\]

Here, \(\alpha(E)\) is the simulated optical absorption coefficient
as a function of incident photon energy and \(L\) is the thickness of
the absorber.

\[
J=e\int_{0}^{\infty} a(E)I_{sun}(E)dE - J_{0}(1-e^{\frac{eV}{kT}})
\]

\[
\eta = \frac{P_{m}}{P_{in}}=\frac{\max(J \times V)}{P_{in}}
\]

To calculate SLME efficiency the current density \(J\), the light
spectrum intensity of sunlight \(I_{sun}\), and the power \(P\) are
all that is needed.
\end{frame}

\begin{frame}[label={sec:orgc0e6297}]{Absorbance Relates to Efficiency via Band Gap in Our Data}
 
\begin{center}
\includesvg[width=370pt]{./.ob-jupyter/64174f234626288aaaca8eb9d0ba24f1cfdbe000}
\end{center}

Inform screening criterion reported by these results\autocite{yang-2023-high-throug} 
\begin{itemize}
\item SLME used as computational proxy for PCE
\item Experimental data\autocite{almora-2020-devic-perfor} broadly agrees with PBE simulation
\end{itemize}
\end{frame}

\begin{frame}[label={sec:org2a430f1}]{My Thesis Research Overview}
The work we did in \citetitle{yang-2023-high-throug}\autocite{yang-2023-high-throug}
\begin{itemize}
\item created the multi-fidelity dataset covering the chemical domain
\item decided the screening criterion used to judge viability of compositions
\end{itemize}

\begin{table}[htbp]
\caption{\(\ce{ABX3}\) Site Candidates}
\centering
\begin{tabular}{l|llllll}
A-site & MA & FA & Cs & Rb & K & \\[0pt]
B-site & Pb & Sn & Ge & Ba & Sr & Ca\\[0pt]
X-site & I & Br & Cl &  &  & \\[0pt]
\end{tabular}
\end{table}

The work I did in \citetitle{manganaris-2023-multi-fidel}\autocite{manganaris-2023-multi-fidel}
\begin{itemize}
\item methods for predicting band gaps at experimental fidelity
\begin{itemize}
\item open-source tools developed for this work
\item made available to the broader materials science community
\end{itemize}
\item use predictions + screening to search sample space for promising compounds
\end{itemize}
\end{frame}

\begin{frame}[label={sec:orgc68b8e9}]{Design Goals in \alert{14-Dimensional} Composition Space}
\begin{columns}
\begin{column}{0.5\columnwidth}
\begin{block}{Deficiencies in Leading Perovskite PVs}
inherent instability/short life-cycle
\begin{itemize}
\item degradation in sunlight
\item soluble in water
\end{itemize}
environmental impact
\begin{itemize}
\item toxicity of Lead (Pb)
\end{itemize}
\end{block}

 
\begin{center}
\includesvg[inkscapeformat=png, inkscapedpi=300,width=150pt]{./.ob-jupyter/7b552d6e27bb59ea631cff90bf7278de09b04ca3}
\end{center}
\end{column}

\begin{column}{0.5\columnwidth}
 
\begin{center}
\includesvg[inkscapeformat=png, inkscapedpi=300,width=150pt]{./.ob-jupyter/4a7937400ef9f31b84cb014534685b0ec1de85c5}
\end{center}

\begin{block}{Search for Improved Compounds}
\alert{37,785} out of \alert{207+ million} compounds possible in a 2x2x2 supercell
will be examined for solutions to these deficiencies using
\begin{itemize}
\item high-throughput density functional theory
\item data collected from literature
\item multi-fidelity machine learning
\end{itemize}
\end{block}
\end{column}
\end{columns}
\end{frame}

\section{Computational Methods In Materials Science}
\label{sec:org0c138b5}
\begin{frame}[label={sec:org1ee19d0}]{Density Functional Theory and Simulation Fidelities}
Vienna \emph{ab initio} Simulation Program (VASP)\autocite{kresse-1996-effic-iterat} computes
\begin{enumerate}
\item ground state atomic structure of perovskite
\item energy levels and occupation of crystal structure's orbitals
\end{enumerate}

\begin{block}{Find Ground State Energy using Levels of Theory (LoT)}
\[
\min_n(E(n)) = \min_n(\hat{T}(n) + \hat{U}(n) + \hat{V}(n))
\]

\(\hat{U}(n)\) approximations account for varying levels of quantum theory 

\begin{description}
\item[{Perdew-Burke-Erzenhof (PBE)}] modern Generalized Gradient Approximation
\item[{Heyd-Scuzeria-Erzenhof (HSE)}] hybrid GGA and Hartree-Folk Theory
\item[{HSE with Spin Orbit Coupling}] corrects for electronic effect of heavy elements (e.g. \(\ce{Pb}\))
\item[{HSE(SOC) on PBE relaxed structure}] compromise, using PBE only for structure relaxation
\end{description}

Spin Orbit Coupling (SOC) Accounts for relativistic effects near heavy nuclei
\end{block}
\end{frame}

\begin{frame}[label={sec:orgebb9776}]{DFT Calculation of Perovskite Band Gaps in Literature}
\begin{center}
\includegraphics[width=.9\linewidth]{Computational_Methods_In_Materials_Science/2023-05-24_05-33-49_screenshot.png}
\end{center}

HSE06 improves band gap predictions compared to PBE\autocite{chan-2010-effic-band}
\end{frame}

\begin{frame}[label={sec:org91eade9}]{Machine Learning for Property Prediction}
\begin{columns}
\begin{column}{0.7\columnwidth}
\begin{center}
\includegraphics[width=300]{Computational_Methods_In_Materials_Science/2023-05-24_04-42-21_screenshot.png}
\end{center}
\end{column}

\begin{column}{0.3\columnwidth}
\begin{block}{Applications}
predict anything/make decisions\autocite{pablo-2019-new-front}
\begin{itemize}
\item interatomic force fields
\item predict properties
\item classify observations
\end{itemize}
\end{block}
\end{column}
\end{columns}
\end{frame}

\begin{frame}[label={sec:org7c180a5}]{Data Driven Framework for Perovskite Discovery}
\begin{center}
\includegraphics[width=390]{DDD-flowchart.png}
\end{center}
\center{}\vspace{-0.75cm}Workflow for accelerating development of new PV materials
\autocite{yang-2023-high-throug}
via ongoing inverse design work
\autocite{yang-2023-discov-novel}
\end{frame}

\section{Perovskites Band Gaps Data and Descriptors}
\label{sec:org8595ee1}
\begin{frame}[label={sec:orgbf08fa9}]{Sample of 14-Dimensional Composition Space}
\begin{columns}
\begin{column}{0.5\columnwidth}
\begin{itemize}
\item 500 unique compositions
\item 5 measurement fidelities
\item 1300 simulated band gaps
\item 90 experimentally measured band gaps
\end{itemize}

 
\begin{center}
\includesvg[width=200pt]{./.ob-jupyter/90faa4a4ad85bb6db363729920bb2c449385e86c}
\end{center}
\end{column}

\begin{column}{0.5\columnwidth}
 
\begin{center}
\includesvg[width=190pt]{./.ob-jupyter/95b93277e9406ce0d36f9f13562ea1f865cc5ead}
\end{center}
\end{column}
\end{columns}
\end{frame}

\begin{frame}[label={sec:org4e1033f}]{Collecting Data is Hard}
\begin{columns}
\begin{column}{0.7\columnwidth}
experiments take years to perform and much labor to organize
\autocite{almora-2020-devic-perfor}

\begin{center}
\includegraphics[width=190]{Data/2023-05-24_08-01-24_screenshot.png}
\end{center}

\vspace{-0.5cm}DFT calculations take multiple
\begin{itemize}
\item hours to converge at PBE level
\item days to converge at HSE level
\item days more to debug
\end{itemize}
Years of work by Professor Arun and Jiaqi have generated
the largest cubic HaP computational database.\autocite{yang-2023-high-throug}
\end{column}

\begin{column}{0.3\columnwidth}
 
\begin{center}
\includesvg[width=120pt]{./.ob-jupyter/3822d7499fa7752e34ad5c4f42ca00509c26cbc3}
\end{center}
\end{column}
\end{columns}
\end{frame}

\begin{frame}[label={sec:org45dd923}]{Effect of DFT Functionals on Our Perovskites Data}
 
\begin{center}
\includesvg[width=400pt]{./.ob-jupyter/618bfd7f0900f6415f0a4a0a8c242c95ccd16e7e}
\end{center}
\end{frame}

\begin{frame}[label={sec:org0e9f8a0},fragile]{Describing a Halide Perovskite}
 \begin{columns}
\begin{column}{0.4\columnwidth}
\begin{block}{14 Dimensional Chemical Vector}
 
\begin{center}
\includesvg[width=170pt]{./.ob-jupyter/ca72ee259641cdd0bcbb7685d0bd737c8163e268}
\end{center}
\end{block}
\end{column}

\begin{column}{0.6\columnwidth}
\begin{block}{Parsing compositions using \texttt{cmcl}}
\begin{Code}
\begin{Verbatim}
\color{EFD}\EFk{import} cmcl
\EFv{Y} = load\_codomain\_subset()
\EFv{df} = Y.Formula.to\_frame().ft.comp()
df.\EFv{index} = Y.Formula
\EFb{print}(df)
\end{Verbatim}
\end{Code}

\begin{small}
\begin{verbatim}
                    FA   Pb   Sn    I   MA   Br
Formula                                        
FAPb_0.7Sn_0.3I_3  1.0  0.7  0.3  3.0  NaN  NaN
MAPb(I0.9Br0.1)3   NaN  1.0  NaN  2.7  1.0  0.3
\end{verbatim}
\end{small}
\end{block}
\end{column}
\end{columns}
\end{frame}

\begin{frame}[label={sec:org3c59447}]{40000 Composition Vectors projected to 2D}
\begin{columns}
\begin{column}{0.4\columnwidth}
\begin{itemize}
\item t-SNE projection projects 14D composition vectors to 2D
\item forms clusters of similar points
\item \textasciitilde{}500 compositions sampled by our research group
\autocite{yang-2023-high-throug}
\item \textasciitilde{}90 compositions sampled in the literature
\autocite{almora-2020-devic-perfor}
\end{itemize}
\end{column}

\begin{column}{0.6\columnwidth}
 
\begin{center}
\includesvg[width=200pt]{./.ob-jupyter/af817dadfec86424566f849862ac137f443e9223}
\end{center}
\end{column}
\end{columns}
\end{frame}

\begin{frame}[label={sec:org8a3778e}]{Constituent Properties}
\begin{block}{Twelve Properties per \(\ce{ABX_3}\) constituent}
\begin{columns}
\begin{column}{0.5\columnwidth}
\begin{itemize}
\item Ionic Radius
\item Boiling Temperature
\item Melting Temperature
\item Density
\item Atomic Weight
\item Electron Affinity
\end{itemize}
\end{column}
\begin{column}{0.5\columnwidth}
\begin{itemize}
\item Ionization Energy
\item Heat of Fusion
\item Heat of Vaporization
\item Electronegativity
\item Atomic Number
\item Period
\end{itemize}
\end{column}
\end{columns}
\end{block}

compute by averaging elemental properties obtained from Mendeleev database
\autocite{mentel-2014}
weighted by stoichiometry of elements at A/B/X-site.
\end{frame}

\begin{frame}[label={sec:orgba4745d}]{Correlating Band Gap with 36 Dimensional Property vector}
band gap at higher levels of theory correlate with composition differently\autocite{yang-2023-high-throug}

 
\begin{center}
\includesvg[width=390pt]{./.ob-jupyter/4aff497f1c6138caa314121c64efb73b46e75d71}
\end{center}
\end{frame}

\section{Multi-Fidelity Learning}
\label{sec:orgacc14e3}
\begin{frame}[label={sec:org4baaeb2}]{Learning from Multi-fidelity dataset}
\begin{block}{Training Strategy}
\begin{itemize}
\item[{$\square$}] oversampling
\item[{$\boxtimes$}] weight scores on higher fidelities to reward target accuracy
\item[{$\boxminus$}] sequential learning/delta learning
\item[{$\boxminus$}] co-Kriging
\end{itemize}
\end{block}

\begin{block}{One-Hot Encoding Levels of Theory}
\begin{columns}
\begin{column}{0.3\columnwidth}
\scriptsize
\begin{center}
\begin{tabular}{rl}
index & LoT\\[0pt]
\hline
1373 & HSE-PBErel(SOC)\\[0pt]
1420 & HSE-PBErel(SOC)\\[0pt]
887 & EXP\\[0pt]
1361 & HSE-PBErel(SOC)\\[0pt]
541 & HSE\\[0pt]
388 & PBE\\[0pt]
1252 & HSE-PBErel(SOC)\\[0pt]
89 & PBE\\[0pt]
297 & PBE\\[0pt]
546 & HSE\\[0pt]
\end{tabular}
\end{center}
\end{column}

\begin{column}{0.7\columnwidth}
\scriptsize
\begin{center}
\begin{tabular}{rrrrrr}
index & EXP & HSE & HSE(SOC) & HSE-PBErel(SOC) & PBE\\[0pt]
\hline
1373 & 0 & 0 & 0 & 1 & 0\\[0pt]
1420 & 0 & 0 & 0 & 1 & 0\\[0pt]
887 & 1 & 0 & 0 & 0 & 0\\[0pt]
1361 & 0 & 0 & 0 & 1 & 0\\[0pt]
541 & 0 & 1 & 0 & 0 & 0\\[0pt]
388 & 0 & 0 & 0 & 0 & 1\\[0pt]
1252 & 0 & 0 & 0 & 1 & 0\\[0pt]
89 & 0 & 0 & 0 & 0 & 1\\[0pt]
297 & 0 & 0 & 0 & 0 & 1\\[0pt]
546 & 0 & 1 & 0 & 0 & 0\\[0pt]
\end{tabular}
\end{center}
\end{column}
\end{columns}
\end{block}
\end{frame}

\begin{frame}[label={sec:org8dceefc}]{Random Forest Regression (RFR)}
\begin{columns}
\begin{column}{0.6\columnwidth}
\begin{center}
\includegraphics[width=260]{Computational_Methods_In_Materials_Science/2023-05-23_20-23-36_screenshot.png}
\end{center}
\end{column}

\begin{column}{0.4\columnwidth}
Decision Tree algorithm memorizes an algorithmic path to the target
\begin{itemize}
\item trees capture interactions between features
\item trees are flexible and may be highly biased to data
\end{itemize}
The forest of random trees averages many paths
\begin{itemize}
\item forest explains variance in data
\item forest reduces the variance of error
\end{itemize}
\end{column}
\end{columns}
\end{frame}

\begin{frame}[label={sec:org00bed4f}]{Gaussian Process Regression (GPR)}
\begin{columns}
\begin{column}{0.6\columnwidth}
Informally, GPRs "remember" the training examples and judge unlabeled
data by its similarity to that aggregate memory.
\begin{block}{Kernel Method}
the similarity function \(k(x, x')\)
\begin{itemize}
\item defines a "universe" of functions
\item defines a density of functions prior to any data
\item works for any two quantifiably similar \(x\)
\begin{itemize}
\item vectors, text, graphs, etc.
\end{itemize}
\end{itemize}
\end{block}

\begin{block}{Drawbacks}
\begin{itemize}
\item \(\mathcal{O}(N^3)\) training time complexity
\item kernels require engineering to accommodate prior expectations
\item break down in sparse spaces/high-dimensional spaces
\end{itemize}
\end{block}
\end{column}

\begin{column}{0.4\columnwidth}
\begin{block}{Sample Prior Functions}
 
\begin{center}
\includesvg[width=170pt]{./.ob-jupyter/ec4d60a2109843449bb674d71a67240d13c1611a}
\end{center}
\end{block}

\begin{block}{Sample Posteriors}
 
\begin{center}
\includesvg[width=170pt]{./.ob-jupyter/eeae18f97963b2846566f847fcd1a8bb5dd88175}
\end{center}
\end{block}
\end{column}
\end{columns}
\end{frame}

\begin{frame}[label={sec:org9836bfe}]{Sure-Independence Screening and Sparsifying Operator (SISSO)}
Resulting in parsimonious models, e.g. for fatigue strength\autocite{he-2021-learn-inter}
\begin{columns}
\begin{column}{0.45\columnwidth}
\begin{itemize}
\item SIS operates on input features creating space of engineered features
\item SO creates sparse model
\begin{itemize}
\item rank engineered features by explanatory power
\item pose linear combination from best features
\item solve for coefficients
\end{itemize}
\end{itemize}

\begin{align*}
FS = 50.58 &\sqrt{\mbox{Diffusion T} \times (\ce{Cr} + \ce{Ni})}\\
 + 2.24 \ce{C} &|\mbox{Tempering T}\\
 - &|\mbox{Diffusion T} - \mbox{Through-Harden T}||\\
 + 158.71
\end{align*}
\end{column}

\begin{column}{0.55\columnwidth}
\begin{center}
\includegraphics[width=200]{Computational_Methods_In_Materials_Science/2022-12-20_09-09-33_screenshot.png}
\end{center}
Notice SIS re-iterations explain the residual\autocite{ghiringhelli-2017-learn-physic}
\end{column}
\end{columns}
\end{frame}

\begin{frame}[label={sec:org09b4d28},fragile]{Tuning Multi-Fidelity Model Performance by Selecting Hyper-parameters}
 \begin{columns}
\begin{column}{0.7\columnwidth}
\begin{center}
\includegraphics[width=280]{ML_pipe_annot.png}
\end{center}
\end{column}

\begin{column}{0.3\columnwidth}
\begin{block}{scores employed}
HP scoring via \texttt{Yogi} tools and SciKit-Learn\autocite{manganaris-2022-mrs-comput}.
\begin{itemize}
\item (\(R^2\))
\item explained variance score
\item maximum error score
\item RMSE
\begin{itemize}
\item all data points
\item EXP
\item PBE
\item HSE
\item HSE(SOC)
\item HSE-PBErel(SOC)
\end{itemize}
\end{itemize}
\end{block}
\end{column}
\end{columns}
\end{frame}

\section{Model Training, Testing, and Interpretation}
\label{sec:org8e1df55}
\begin{frame}[label={sec:org2f549a8}]{Model Training Method}
\begin{columns}
\begin{column}{0.5\columnwidth}
\begin{enumerate}
\item partition dataset by 80/20 train/test split
\begin{itemize}
\item hold out 282 test set
\item proceed to following steps with 1123 train set
\end{itemize}
\item instantiate estimator pipeline with default parameters
\item Set up K-folds validation strategy
\begin{enumerate}
\item Generate learning curves with 10-fold validation
\item Determine adequate size of validation set (the knee)
\item the K accordingly
\end{enumerate}
\item Perform grid search Hyper-Parameter Optimization (HPO)
\begin{itemize}
\item score on validation set
\item assess scores wholistically to determine best parameters
\end{itemize}
\end{enumerate}
\end{column}

\begin{column}{0.5\columnwidth}
\begin{block}{RFR Learning Curves}
\begin{center}
\includesvg[width=190pt]{/home/panos/Documents/manuscripts/DFT+ML+feature_engineering/RFR/.ob-jupyter/4b5d77be3d415aa5e9faadb59147a7c2f560e136}
\end{center}
\end{block}
\end{column}
\end{columns}
\end{frame}

\begin{frame}[label={sec:orga987518}]{Test 282 Band Gap Predictions}
 
\begin{center}
\includesvg[width=400pt]{./.ob-jupyter/3133b81a112f13e387878a1d018b864f057d9f03}
\end{center}

\begin{columns}
\begin{column}{0.3\columnwidth}
Best RFR selected from >30k differently parametrized models
\end{column}

\begin{column}{0.3\columnwidth}
Best GPR utilizes non-stationary Matern kernel with \(\nu=3\slash{}2\)
\end{column}

\begin{column}{0.3\columnwidth}
Best SISSO model using only 11 features outperforms OLS on 55 features
\end{column}
\end{columns}
\end{frame}

\begin{frame}[label={sec:orgfdc5018}]{Evaluating Models for Use in Screening}
\begin{columns}
\begin{column}{0.5\columnwidth}
\begin{center}
\begin{tabular}{lrrr}
Score Categories & GPR & RFR & SISSO\\[0pt]
\hline
rmse & 0.15 & 0.12 & 0.47\\[0pt]
rmse EXP & 0.12 & 0.15 & 0.33\\[0pt]
rmse PBE & 0.12 & 0.10 & 0.39\\[0pt]
rmse HSE & 0.21 & 0.15 & 0.51\\[0pt]
rmse HSE(SOC) & 0.15 & 0.10 & 0.57\\[0pt]
rmse HSE-PBE(SOC) & 0.13 & 0.13 & 0.47\\[0pt]
\end{tabular}
\end{center}
\end{column}

\begin{column}{0.5\columnwidth}
\begin{block}{Choosing a Model}
\begin{itemize}
\item RMSE per fidelity subset at final HPO iteration
\item EXP RMSE motivated parameter selection
\item EXP RMSE is important
\item However, total RMSE is critical indicator of extrapolative ability
\end{itemize}

GPR outperforms RFR in EXP subset, but RFR shows better extrapolative
ability. RFR is best.
\end{block}
\end{column}
\end{columns}
\end{frame}

\begin{frame}[label={sec:org84588af}]{SISSO Expression}
\begin{align*}
bg\mbox{ [eV]} = 1.752 &((X;\mbox{electronegativity}*A;\mbox{heat of fusion})\\
                       &-(B;\mbox{electron affinity}+B;\mbox{ionization energy}))\\
                -0.586 &((B;Sn-\mbox{HSE})+(\mbox{PBE}-X;\mbox{electronegativity}))\\
                +1.064 &((A;\mbox{electronegativity}-B;Ca)*(B;\mbox{heat of vap}-X;\mbox{electron affinity}))\\
                +4.657
\end{align*}
\end{frame}

\begin{frame}[label={sec:orgc301ee3}]{Shapely Additive Explanation (SHAP)}
SHAP values quantify the contribution of a feature \(x_i \in X\) to a prediction
\autocite{lundberg-2017-unified-approac}

\begin{center}
\includegraphics[width=400]{SHAP-crop.png}
\end{center}
\end{frame}

\begin{frame}[label={sec:org9fce750}]{SHAP Analyze RFR}
\begin{center}
\includegraphics[width=350]{Models/2023-05-24_09-40-46_screenshot.png}
\end{center}
\end{frame}

\section{Discovery of Suitable Compositions}
\label{sec:org8c40697}
\begin{frame}[label={sec:org57313ed}]{Make Predictions on \alert{40000} Hypothetical Compounds}
\begin{columns}
\begin{column}{0.4\columnwidth}
\begin{itemize}
\item Make predictions on sample space using RFR
\item Predict at experimental data fidelity (LoT = EXP)
\item Predictions have expected error of 0.15 eV
\item Visualize predictions on t-SNE projection
\item No obvious clustering of \(bg \in [1, 2]\,\si{\electronvolt}\)
\end{itemize}
\end{column}

\begin{column}{0.6\columnwidth}
 
\begin{center}
\includesvg[width=200pt]{./.ob-jupyter/787ace24d92492507b8e898ff8b26e0c00b39084}
\end{center}
\end{column}
\end{columns}
\end{frame}

\begin{frame}[label={sec:org0a19ef5}]{Screening Yields}
\begin{columns}
\begin{column}{0.3\columnwidth}
\begin{center}
\includegraphics[width=120]{screening_ops.png}
\end{center}
\end{column}

\begin{column}{0.7\columnwidth}
\scriptsize
\begin{center}
\begin{tabular}{rlr}
 & Formula & band gap [eV]\\[0pt]
\hline
19290 & FA0.375Rb0.625Sn1.000I1.000 & 1.98\\[0pt]
19309 & FA0.375MA0.125Rb0.500Sn1.000Cl1.000 & 1.99\\[0pt]
19310 & FA0.375MA0.125Rb0.500Sn1.000Br1.000 & 1.95\\[0pt]
19306 & FA0.375MA0.125Rb0.500Sr1.000Cl1.000 & 1.95\\[0pt]
19308 & FA0.375MA0.125Rb0.500Sn1.000I1.000 & 1.70\\[0pt]
19307 & FA0.375MA0.125Rb0.500Sr1.000Br1.000 & 1.96\\[0pt]
19304 & FA0.375Rb0.625Ba1.000Br1.000 & 1.93\\[0pt]
19303 & FA0.375Rb0.625Ba1.000Cl1.000 & 1.89\\[0pt]
19305 & FA0.375MA0.125Rb0.500Sr1.000I1.000 & 1.74\\[0pt]
19302 & FA0.375Rb0.625Ba1.000I1.000 & 1.68\\[0pt]
\end{tabular}
\end{center}

 
\begin{center}
\includesvg[width=260pt]{./.ob-jupyter/4d64b98962d2908713d9543efb9c28952a992b32}
\end{center}
\end{column}
\end{columns}
\end{frame}

\begin{frame}[label={sec:orge219a05}]{Screened Constituent Frequencies}
 
\begin{center}
\includesvg[width=400pt]{./.ob-jupyter/6319952505071b95f6f19f67c253faafb5f46d15}
\end{center}

\begin{columns}
\begin{column}{0.3\columnwidth}
\begin{itemize}
\item prefer occupying small fraction mixing in most constituents
\item \(\ce{K}\) and \(\ce{Rb}\) also prefer full occupancy.
\end{itemize}
\end{column}

\begin{column}{0.3\columnwidth}
\begin{itemize}
\item favor full occupation a rate of 5-8\%
\item showed some preference approaching doping
\end{itemize}
\end{column}

\begin{column}{0.3\columnwidth}
\begin{itemize}
\item strongly prefer full occupation
\end{itemize}
\end{column}
\end{columns}
\end{frame}

\section{Summary and Conclusions}
\label{sec:orgd8888fb}
\begin{frame}[label={sec:org9d17fc3},fragile]{Seven Step Semiconductor Composition Optimization}
 \begin{enumerate}
\item preexisting data set of \textasciitilde{}500 samples from domain of \textasciitilde{}40000 single-site mixed compositions
\item extract features using \texttt{cmcl}
\item train models of band gaps measured at multiple fidelities \alert{using nine scoring criterion}
\item optimize model parameters while selecting for high scores in high-fidelity subsets using \texttt{yogi}
\item test best model, analyze accuracy, interpret predictions.
\item make predictions on \alert{entire sample space}
\item screen for viable candidate compositions
\end{enumerate}
\end{frame}

\begin{frame}[label={sec:orgbe8b3aa}]{Conclusions}
\begin{itemize}
\item RFR yields the best model because
\begin{itemize}
\item it captures relevant feature interactions as corroborated by Pearson correlations
\item it flexibly represents nonlinear relationships between feature interactions and band gap
\item as an ensemble model, it reduces the variance of error
\end{itemize}
\item SISSO is remarkably accurate for its simplicity but it is limited
for high fidelity predictions due to the absence of some fidelity
variables
\item GPR has potential for further study alongside graph neural networks
\item \alert{834/1251 candidates are lead free}
\item Top 10 most-stable lead-free compounds A-site mixed and mostly Iodides
\end{itemize}
\end{frame}

\begin{frame}[label={sec:org46a7e04},fragile]{Publications Presentations and Other Contributions}
 \begin{columns}
\begin{column}{0.66\columnwidth}
\begin{block}{Articles}
\footnotesize
\begin{itemize}
\item Yang, J., Manganaris, P. T., \& Mannodi Kanakkithodi, A. K. (2023). A high-throughput computational dataset of halide perovskite alloys. Digital Discovery, \url{http://dx.doi.org/10.1039/d3dd00015j}
\item Manganaris, P., Yang, J., \& Arun Mannodi Kanakkithodi (2023). Multi-fidelity machine learning pervoskite composition vs band gap. IN PREPARATION.
\item Edlabadkar, R., Yang, J., Rahman, H., Manganaris, P., Korimilli, E. P., \& Arun Mannodi-Kanakkithodi (TBD). Driving Halide Perovskite Discovery Using Graph Neural Networks. IN PREPARATION.
\item Gollapalli, P., Manganaris, P., \& Arun Mannodi-Kanakkithodi (TBD). Graph neural network predictions for formation energy of native defects in zinc blende semiconductors. IN PREPARATION.
\item Yang, J., Manganaris, P., \& Arun Mannodi-Kanakkithodi (TBD). Discovering novel halide perovskite alloys using multi-fidelity machine learning and genetic algorithm. IN PREPARATION.
\end{itemize}
\end{block}
\end{column}
\begin{column}{0.33\columnwidth}
\begin{block}{Presentations}
\footnotesize
\begin{itemize}
\item Poster for DS02 symposium MRS fall 2022
\item Talk for Purdue Soft Materials symposium
\item developed MRS spring 2022 tutorial hosted on nanoHUB
\end{itemize}
\end{block}
\begin{block}{Software}
\footnotesize
\url{https://github.com/PanayotisManganaris/}
\begin{itemize}
\item \texttt{cmcl}
\item \texttt{yogi}
\item \texttt{spyglass}
\item \texttt{pysisso}
\end{itemize}
\end{block}
\end{column}
\end{columns}
\end{frame}

\section{Credit}
\label{sec:orgc86571d}
\begin{frame}[label={sec:org0875c50}]{Acknowledgements}
\begin{columns}
\begin{column}{0.6\columnwidth}
\begin{block}{Research Group}
I am grateful to
\begin{itemize}
\item Professor Arun Mannodi-Kanakkithodi for his mentorship, support, and knowledge.
\item Professor Erk for her guidance and council
\item Professor Strachan for his instruction
\item Jiaqi Yang for his collaboration
\item Habibur Rahman for his friendship and curiosity
\end{itemize}
\end{block}
\begin{block}{Funding}
\begin{itemize}
\item Ross Fellowship
\item Professor Arun's startup research grant
\end{itemize}
\end{block}
\begin{block}{Resources}
RCAC Bell Computing Cluster
\end{block}
\end{column}
\begin{column}{0.4\columnwidth}
\hspace*{-1cm}
\includegraphics[width=200]{/home/panos/Pictures/mrg-photoshoot/41_Kanakkithodi.jpg}
\end{column}
\end{columns}
\end{frame}
\section{Bib}
\label{sec:org8d1a04b}
\begin{frame}[allowframebreaks]{References}
\AtNextBibliography{\tiny}
\printbibliography
\end{frame}
\appendix
\begin{frame}[label={sec:orgdff3e15}]{Interpreting t-SNE clusters}
\begin{columns}
\begin{column}{0.5\columnwidth}
 
\begin{center}
\includesvg[width=200pt]{./.ob-jupyter/c7002c894b6ade243f089aabdefcb2c2b4d07a89}
\end{center}
\end{column}

\begin{column}{0.5\columnwidth}
 
\begin{center}
\includesvg[width=200pt]{./.ob-jupyter/bd7813d2d1d9cd1108cdd457bd977e8fd1a6aa99}
\end{center}
\end{column}
\end{columns}
\end{frame}

\begin{frame}[label={sec:org2d1c8fe}]{Future Work}
\[
\hat{y}_i = y_i + \epsilon_i \mbox{ for observations } i=1,…,N
\]

multi-fidelity learning \alert{is} regression on heteroscedastic data

\begin{description}
\item[{high fidelity}] prior on \(\epsilon_i\) is narrow
\item[{low fidelity}] prior on \(\epsilon_i\) is wide
\end{description}

e.g \(\epsilon_i \sim N(0,\sigma_i^2)\), with \(\sigma_i^2\) being a function of fidelity

\begin{itemize}
\item find a way to bias RFR training using prior information
\item avoid OHE which adds additional sparse features which increases training expense.
\end{itemize}
\end{frame}

\begin{frame}[label={sec:orgfb798ca}]{Data Pre-Processing}
\begin{center}
\includegraphics[width=300]{/home/panos/Pictures/flowcharts/data_proc.png}
\end{center}
\center{}Data pre-processing workflow implemented with Python Pandas
\end{frame}

\begin{frame}[label={sec:org5369737}]{SHAP Analyze GPR}
\begin{center}
\includegraphics[width=350]{Models/2023-05-24_09-42-08_screenshot.png}
\end{center}
\end{frame}

\begin{frame}[label={sec:org59e0f47}]{SHAP Analyze SIS Features in RFR}
\begin{center}
\includegraphics[width=400pt]{/home/panos/Documents/manuscripts/DFT+ML+feature_engineering/RFR/.ob-jupyter/d0b6ba16e4913fe81324e8170b2b5b241c1053c8.png}
\end{center}
\end{frame}

\begin{frame}[label={sec:org250affb}]{Density Functional Theory (DFT)}
\begin{itemize}
\item Higher levels of theory (LoT) approximations of \(\hat{U}(n)\) account for more physics.
\item VASP is used to perform these calculations at the appropriate LoT.
\autocite{kresse-1996-effic-iterat,kresse-1996-effic-ab}
\end{itemize}
\begin{columns}
\begin{column}{0.5\columnwidth}
Recast \(3n\)-D Schrödinger equation as \(n\) \(3\)-D Kohn-Sham equations
\[
\left(-\frac{\hbar^2}{2m}\nabla+\hat{V}_s\right)\psi(r_k) = \epsilon_k\psi(r_k)
\]

Solutions yield the local electron densities
\[
n(r) = \sum_{i=1}^n|\psi_i(r)|^2
\]

which can be used to minimize \(E(n)\).
\end{column}

\begin{column}{0.5\columnwidth}
Express potential energy due to the nuclei using density \(n(r)\)
given at \(r_k\) by marginalizing the wave function over all \(r_{\neg k}\)

\[
\hat{V}(n) = \int \hat{V}_{nuc}(r_k)n(r_k)d^3r
\].

So only electron-electron interaction term \(\hat{U}(n)\) requires approximation

\[
E = \bra{\psi}\hat{H}\ket{\psi} \implies E(n) &= \hat{T}(n) + \hat{U}(n) + \hat{V}(n)
\]
\end{column}
\end{columns}
\end{frame}
\end{document}