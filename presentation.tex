% Created 2023-05-10 Wed 18:29
% Intended LaTeX compiler: pdflatex
\documentclass[10pt, compress, aspectratio=169]{beamer}
\usepackage[utf8]{inputenc}
\usepackage[T1]{fontenc}
\usepackage{graphicx}
\usepackage{longtable}
\usepackage{wrapfig}
\usepackage{rotating}
\usepackage[normalem]{ulem}
\usepackage{amsmath}
\usepackage{amssymb}
\usepackage{capt-of}
\usepackage{hyperref}
\usepackage{ptm}

\institute[Mannodi Group]{\inst{1} Purdue Materials Engineering\\Professor Arun Mannodi-Kanakkithodi}
%\setbeamertemplate{caption}[numbered]
\beamertemplatenavigationsymbolsempty
%
\usepackage{mathtools}%% bugfixing and additional tools for amsmath
\usepackage{amsfonts}%% formal math fonts
\usepackage{physics}%% powerful mathematics shorthand
\usepackage{bm}%% bold math
\usepackage{gensymb}%% inter-environment consistent measurment unit symbols
\usepackage{upgreek}%% easy lower and uppercase nonitalicized greek letters
\usepackage[version=3]{mhchem}%% for writing chemical formulae
%
\usepackage[%
%citestyle=authoryear-icomp,
%bibstyle=authoryear-icomp,
style=authortitle-tcomp,
%style=verbose,
autocite=footnote,
hyperref=true,
backref=true,
maxcitenames=3,
url=true,
backend=biber,
natbib=true,
]{biblatex}
\addbibresource{~/org/bibliotex/bibliotex.bib}
%
%% \renewbibmacro*{cite:title}{%
%%   \printtext[bibhyperref]{%
%%     \printfield[citetitle]{labeltitle}%
%%     \setunit{\space}%
%%     \printtext[parens]{\printdate}%
%%   }%
%% }
%
\usepackage{svg}
\svgsetup{%
inkscapelatex=false,
inkscapeformat=png,
inkscapedpi=300,
}
\svgpath{{/home/panos/Pictures/figs}}
\setbeamertemplate{footline}{%
\hfill
\vspace*{0.1cm}
\insertframenumber\slash\inserttotalframenumber
\hspace*{0.2cm}
}
\usetheme{default}
\author{Panayotis Manganaris}
\date{\today}
\title{presentation}
\hypersetup{
 pdfauthor={Panayotis Manganaris},
 pdftitle={presentation},
 pdfkeywords={},
 pdfsubject={},
 pdfcreator={Emacs 28.2 (Org mode 9.5.5)}, 
 pdflang={English}}
\begin{document}

\maketitle
\begin{frame}{Outline}
\tableofcontents
\end{frame}

\section{Perovskites}
\label{sec:orgcf5e07c}
\begin{frame}[label={sec:org7a49dd3}]{Perovskites Recent History}
Notable oxide perovskites and properties circa 2006
\autocite{jiang-2006-predic-lattic}

\begin{columns}
\begin{column}{0.5\columnwidth}
\small
\begin{center}
\begin{tabular}{ll}
<K,Ba>BiO3 & super-conduction\\[0pt]
Pb<Zr,Ti>O3 & piezoelectric action\\[0pt]
Pb<Mb,Mg>O3 & relaxor ferroelectric\\[0pt]
BaTiO3 & dielectric properties\\[0pt]
\end{tabular}
\end{center}
\end{column}
\begin{column}{0.5\columnwidth}
\small
\begin{center}
\begin{tabular}{ll}
Pb<Zr,Ti>O3 & electro-optic\\[0pt]
LaMnO3 & magneto-resistance\\[0pt]
LaCrO3 & catalytic\\[0pt]
BaCeO3 & photonic conductivity\\[0pt]
\end{tabular}
\end{center}
\end{column}
\end{columns}
\begin{figure}[htbp]
\centering
\includegraphics[width=170]{Introduction_To_Perovskites/2022-07-14_15-32-57_screenshot.png}
\caption{Barium Titanate functional ceramic}
\end{figure}
\end{frame}

\begin{frame}[label={sec:org92eea70}]{Halide Perovskites}
\begin{columns}
\begin{column}{0.4\columnwidth}
\begin{block}{Solar Energy Harvesting}
\begin{itemize}
\item 2009 : \(\ch{MAPbI_3}\) dye-sensitized absorber with 3.8\% PCE
\autocite{kojima-2009-organ-halid}
\item 2011 : absorber efficiency doubles
\autocite{im-2011}
\item 2013 : \(\ch{MAPbX_3}\) found to transport holes+electrons
\autocite{saliba-2014-influen-therm}
\end{itemize}

\alert{Data-Driven Design} accelerates development
\end{block}
\end{column}
\begin{column}{0.6\columnwidth}
\begin{center}
\includegraphics[width=115]{Introduction_To_Perovskites/2022-07-15_23-57-31_screenshot.png}
\end{center}
Perovskite Solar Absorber Layer\footnote{\tiny{}Photo by Dennis Schroeder\newline{}National Renewable Energy Laboratory}
\end{column}
\end{columns}
\end{frame}
\begin{frame}[allowframebreaks]{Hybrid Organic-Inorganic Perovskites}
\begin{columns}
\begin{column}{0.5\columnwidth}
\begin{center}
\includegraphics[height=90]{Introduction_To_Perovskites/2022-07-18_13-42-36_screenshot.png}
\end{center}

\(\alpha\)-Phase \(Pm\bar{3}m\) Hybrid Perovskite Crystal
\autocite{yan-2016-defec-physic}

\ch{ABX_3} with A=\ch{CH3NH3+} (MA)
\end{column}
\begin{column}{0.5\columnwidth}
\begin{center}
\includegraphics[angle=90,height=90]{Introduction_To_Perovskites/2022-07-13_19-55-58_screenshot.png}
\end{center}
2-D Ruddlesden-Popper \(Pbca\) Hybrid Perovskite Crystal
\autocite{hong-2021-layer-edge}

\(\ch{L_2BX_4}\) with \(L=\ch{C4H12N+}\) (BA)

Generalizes to LA\textsubscript{m-1}B\textsubscript{m}X\textsubscript{3m+1} perovskite with m>1
\end{column}
\end{columns}
\end{frame}
\begin{frame}[label={sec:org21597d7}]{Attractive Properties of Perovskites}
\begin{columns}
\begin{column}{0.5\columnwidth}
\begin{center}
\includegraphics[width=200]{Introduction_To_Perovskites/2022-07-16_01-34-42_screenshot.png}
\end{center}
Typical severity of gap states
\autocite{mannodi-kanakkithodi-2020-defec-energ}
\end{column}
\begin{column}{0.5\columnwidth}
\begin{itemize}
\item robust against carrier scattering
\autocite{yan-2016-defec-physic}
\item spontaneously forming
\begin{itemize}
\item high energy defects unlikely
\end{itemize}
\item edge-states
\item finite internal dipole moments
\autocite{hong-2021-layer-edge}
\begin{itemize}
\item ferroellectrically aligned ligands
\item long luminescence lifetimes
\end{itemize}
\item hydro-stability
\autocite{fu-2021-two-dimen}
\begin{itemize}
\item hydrophobic ligands
\end{itemize}
\end{itemize}
\end{column}
\end{columns}
\end{frame}

\section{Bib}
\label{sec:orgcacb5ff}
\begin{frame}[label={sec:orge63e173}]{References}
\AtNextBibliography{\tiny}
\printbibliography
\end{frame}
\section{SI}
\label{sec:org2778532}
\begin{frame}[label={sec:orge0924f0}]{Data Pre-Processing}
\begin{center}
\includegraphics[width=300]{/home/panos/Pictures/flowcharts/data_proc.png}
\end{center}
Data pre-processing Workflow to Implement with Python Pandas
\end{frame}
\begin{frame}[label={sec:orga7b1a97}]{Machine Learning Pipeline}
\begin{center}
\includegraphics[width=300]{/home/panos/Pictures/flowcharts/ML_pipe.png}
\end{center}
Machine Learning Pipeline to Implement with Python SciKit-Learn
\end{frame}
\end{document}