% Created 2023-05-16 Tue 22:46
% Intended LaTeX compiler: pdflatex
\documentclass[10pt, aspectratio=169, presentation]{beamer}
\usepackage[utf8]{inputenc}
\usepackage[T1]{fontenc}
\usepackage{graphicx}
\usepackage{longtable}
\usepackage{wrapfig}
\usepackage{rotating}
\usepackage[normalem]{ulem}
\usepackage{amsmath}
\usepackage{amssymb}
\usepackage{capt-of}
\usepackage{hyperref}
\usepackage{ptm}

%\setbeamertemplate{caption}[numbered]
\beamertemplatenavigationsymbolsempty
%
\usepackage{mathtools}%% bugfixing and additional tools for amsmath
\usepackage{amsfonts}%% formal math fonts
\usepackage{physics}%% powerful mathematics shorthand
\usepackage{bm}%% bold math
\usepackage{gensymb}%% inter-environment consistent measurment unit symbols
\usepackage{upgreek}%% easy lower and uppercase nonitalicized greek letters
\usepackage[version=4]{mhchem}%% for writing chemical formulae
\usepackage{siunitx}[=v2]%% version 2 workaround first-frame artifact
%
\usepackage[%
%citestyle=authoryear-icomp,
%bibstyle=authoryear-icomp,
style=authortitle-tcomp,
%style=verbose,
autocite=footnote,
hyperref=true,
backref=true,
maxcitenames=3,
url=true,
backend=biber,
natbib=true,
]{biblatex}
\addbibresource{/home/panos/Documents/masters/local-bib.bib}
%
\usepackage{svg}
\svgsetup{%
inkscapelatex=false,
inkscapeformat=png,
inkscapedpi=300,
}
\svgpath{{/home/panos/Pictures/figs}}
%
\setbeamertemplate{footline}{%
\hfill
\vspace*{0.1cm}
\insertframenumber\slash\inserttotalframenumber
\hspace*{0.2cm}
}
\institute[Mannodi Group]{
\inst{1} Purdue Materials Engineering\\Professor Arun Mannodi-Kanakkithodi
}
\usetheme{default}
\author{Panayotis Manganaris\inst{1}}
\date{\today}
\title{Machine Learning Perovskite Bandgaps for Improved Photovoltaics}
\hypersetup{
 pdfauthor={Panayotis Manganaris\inst{1}},
 pdftitle={Machine Learning Perovskite Bandgaps for Improved Photovoltaics},
 pdfkeywords={},
 pdfsubject={},
 pdfcreator={Emacs 28.2 (Org mode 9.5.5)}, 
 pdflang={English}}
\begin{document}

\maketitle
\begin{frame}{Outline}
\tableofcontents
\end{frame}

\section{Background}
\label{sec:orgce13dd3}
\begin{frame}[label={sec:org90aa522}]{Photovoltaic Power Generation}
\begin{columns}
\begin{column}{0.3\columnwidth}
\begin{center}
\includegraphics[width=100]{thin-film-PV.png}
\end{center}
\begin{center}
\includegraphics[width=90]{Si-HaP-tandem-PV.png}
\end{center}

\tiny{}\center{}Perovskite Applications\footnote{\tiny{}\href{https://www.lesker.com/newweb/ped/applications/perovskite-research.cfm }{www.lesker.com}}
\end{column}
\begin{column}{0.7\columnwidth}
\begin{block}{Solar Energy Harvesting}
\begin{itemize}
\item 2009 : \(\ce{MAPbI_3}\) dye-sensitized absorber with 3.8\% PCE
\autocite{kojima-2009-organ-halid}
\item 2011 : absorber efficiency doubles
\autocite{im-2011}
\item 2013 : \(\ce{MAPbX_3}\) found to transport holes+electrons
\autocite{saliba-2014-influen-therm}
\end{itemize}


HaP PVs become competitive in 2013
\end{block}
\end{column}
\end{columns}
\end{frame}
\begin{frame}[label={sec:orgcdf8b37}]{Rapid Rise of Perovskite PVs}
\begin{columns}
\begin{column}{0.6\columnwidth}
\begin{center}
\includegraphics[width=250pt]{Perovskites/2023-05-15_18-16-32_screenshot.png}
\end{center}
\end{column}

\begin{column}{0.4\columnwidth}
\begin{itemize}
\item Cumulative maximum efficiency of PVs tested by accredited laboratories\autocite{research-2023-best-resear}
\item Perovskite+Si tandem monolithic cells second only to two-junction \(\ce{GaAs}\)
\end{itemize}
\end{column}
\end{columns}
\end{frame}

\begin{frame}[label={sec:orgaa0f698}]{What's a Perovskite?}
Notable oxide perovskites and properties circa 2006
\autocite{jiang-2006-predic-lattic}
\begin{columns}
\begin{column}{0.5\columnwidth}
\small
\begin{center}
\begin{tabular}{ll}
<K,Ba>BiO3 & super-conduction\\[0pt]
Pb<Zr,Ti>O3 & piezoelectric action\\[0pt]
Pb<Mb,Mg>O3 & relaxor ferroelectric\\[0pt]
BaTiO3 & dielectric properties\\[0pt]
\end{tabular}
\end{center}
\end{column}
\begin{column}{0.5\columnwidth}
\small
\begin{center}
\begin{tabular}{ll}
Pb<Zr,Ti>O3 & electro-optic\\[0pt]
LaMnO3 & magneto-resistance\\[0pt]
LaCrO3 & catalytic\\[0pt]
BaCeO3 & photonic conductivity\\[0pt]
\end{tabular}
\end{center}
\end{column}
\end{columns}
\begin{figure}[htbp]
\centering
\includegraphics[width=170]{Introduction_To_Perovskites/2022-07-14_15-32-57_screenshot.png}
\caption{Barium Titanate functional ceramic}
\end{figure}
\end{frame}

\begin{frame}[label={sec:org5650a47}]{Halide Perovskites}
\begin{columns}
\begin{column}{0.5\columnwidth}
\begin{table}[htbp]
\caption{\(\ce{ABX3}\) Chemical Domain}
\centering
\begin{tabular}{l|llllll}
A-site & \ACRshort{ma} & \ACRshort{fa} & Cs & Rb & K & \\[0pt]
B-site & Pb & Sn & Ge & Ba & Sr & Ca\\[0pt]
X-site & I & Br & Cl &  &  & \\[0pt]
\end{tabular}
\end{table}
\end{column}

\begin{column}{0.5\columnwidth}
\begin{center}
\includegraphics[width=.9\linewidth]{hybrid-HaP.png}
\end{center}
\end{column}
\end{columns}
\end{frame}

\section{Machine Learning In Materials Science}
\label{sec:orgf62a8f0}
\begin{frame}[label={sec:org6a2f5ad}]{Data Driven Materials Design}
\begin{center}
\includegraphics[width=390]{DDD-flowchart.png}
\end{center}
\center{}\vspace{-0.5cm}Workflow for accelerating testing and development of new PV materials
\autocite{yang-2023-high-throug,pablo-2019-new-front}
\end{frame}

\begin{frame}[label={sec:orgd1306af}]{Machine Learning In Materials Science}
An increasingly popular technique for accelerating discovery and
development\autocite{pablo-2019-new-front}
\begin{itemize}
\item force fields
\item property-prediction
\item building on dft
\end{itemize}
\end{frame}
\begin{frame}[label={sec:org0b7c33a}]{Regression}
\end{frame}

\begin{frame}[label={sec:org6e7e229}]{Learning Algorithms}
\begin{itemize}
\item Decision Tree
\item Random Forest
\item Linear
\item SISSO
\item Gaussian Process
\end{itemize}
\end{frame}

\section{Data}
\label{sec:orgfac9444}
\begin{frame}[label={sec:org6498f75}]{Band Gap Relates to Efficiency}
\begin{columns}
\begin{column}{0.5\columnwidth}
 
\begin{center}
\includesvg[width=210pt]{./.ob-jupyter/2b677464785e9d1d299f0f955a3af16e9c3b5f7e}
\end{center}
\end{column}

\begin{column}{0.5\columnwidth}
 
\begin{center}
\includesvg[width=210pt]{./.ob-jupyter/0c4832ea502cef117df59890aa3341839c46147b}
\end{center}
\end{column}
\end{columns}

SLME metric accounts for more energetic processes than
Shockley-Queisser criterion (\(bg \approx 1.3\)) allowing selection on range
of bandgaps determined according to level of theory\autocite[p.1]{yu-2012-ident-poten}
\end{frame}

\begin{frame}[label={sec:org6f1475a}]{Effect of DFT Functionals Levels of Theory}
\begin{center}
\includegraphics[width=.9\linewidth]{Data/2023-05-15_20-04-18_screenshot.png}
\end{center}
\end{frame}

\begin{frame}[label={sec:org5670c66}]{Variation in Band Gap changes with Level of Theory}
 
\begin{center}
\includesvg[width=300pt]{./.ob-jupyter/e044073b053593cbfe267508875c51401afb43a3}
\end{center}
\end{frame}

\begin{frame}[label={sec:orge51580d}]{Features of Perovskites Descriptive of Band Gap}
\begin{block}{composition vectors}
\end{block}
\end{frame}

\begin{frame}[label={sec:org8725de9}]{Sample Space}
\end{frame}

\begin{frame}[label={sec:org7167c35}]{Constituent Properties}
\begin{block}{Twelve Properties per \(\ce{ABX_3}\) constituent}
\begin{columns}
\begin{column}{0.5\columnwidth}
\begin{itemize}
\item Ionic Radius
\item Boiling Temperature
\item Melting Temperature
\item Density
\item Atomic Weight
\item Electron Affinity
\end{itemize}
\end{column}
\begin{column}{0.5\columnwidth}
\begin{itemize}
\item Ionization Energy
\item Heat of Fusion
\item Heat of Vaporization
\item Electronegativity
\item Atomic Number
\item Period
\end{itemize}
\end{column}
\end{columns}
\end{block}
\end{frame}

\begin{frame}[label={sec:org02c7a5f}]{Bivariate Analysis}
\begin{itemize}
\item correlations can be surprising
\item some confirm physical expectations
\end{itemize}
\end{frame}

\section{Multi-Fidelity Learning}
\label{sec:orgea80fb5}
\begin{frame}[label={sec:org89ec1aa}]{Challenges of Learning from Overlapping datasets}
\begin{description}
\item[{high fidelity}] prior on variance is narrow
\item[{low fidelity}] prior on variance is wide
\end{description}
\end{frame}

\begin{frame}[label={sec:orga754722}]{Multi-Fidelity Model Development Workflow}
\alert{flowchart}

simultaneously utilize multiple measurements made of the same
composition in training models.\autocite{manganaris-2022-mrs-comput}
\end{frame}

\section{Models}
\label{sec:org98f5768}
\begin{frame}[label={sec:orgbd012ff}]{Compare Three Methods}
\end{frame}
\begin{frame}[label={sec:orgf43f51d}]{Interpreting Models}
\end{frame}
\begin{frame}[label={sec:org4d6b75b}]{Shapely Additive Explanation (SHAP)}
\end{frame}
\section{Results}
\label{sec:orgdebc8bf}
\begin{frame}[label={sec:orgcfbe984}]{Make Predictions on Domain Subset}
\end{frame}

\begin{frame}[label={sec:orgb71d26e}]{Tolerance Factor}
\begin{block}{Perovskite Stability}
Stability of high symmetry \(\alpha\) phase depends on the ABX member's
ionic radii satisfying Goldschmidt's tolerance:
\autocite{yin-2015-halid-perov}

\[
1 \approx t = \frac{R_A+R_X}{\sqrt{2}*(R_B+R_X)}
\]

Empirically, it is \alert{necessary} that A must be much larger than B
for Perovskite formation. B is usually a large atom (e.g. Pb, Sn) in
Halide Perovskites. So A is suitably occupied by a Molecule -- an
amine -- with an effective ionic radius.
\autocite{kieslich-2015-exten-toler}
\end{block}
\end{frame}
\begin{frame}[label={sec:org638cbf4}]{Other Tolerance Factors in Literature}
\begin{block}{Octahedral Factor}
\[
o=\frac{r_B}{r_X}
\]
\end{block}
\begin{block}{Bartel Tolerance}
\[
t_{Bartel}=\frac{r_X}{r_B}-[1-\frac{\frac{r_A}{r_B}}{ln(\frac{r_A}{r_B})}]
\]
proposed recently
\autocite{bartel-2019-new-toler}
\end{block}
\begin{block}{Ionic Radii}
\begin{itemize}
\item Elemental Radii are well documented in public databases
\item Molecular Radii are defined in various ways, mostly for small
molecules.
\end{itemize}
\end{block}
\end{frame}

\begin{frame}[label={sec:org99453c5}]{Screening Yields}
\end{frame}
\section{Summary}
\label{sec:org17925d3}
\begin{frame}[label={sec:orgb42e787},fragile]{Review of Work}
 \begin{itemize}
\item preexisting data set of \textasciitilde{}500 samples from domain of \textasciitilde{}40000 single-site mixed compositions
\item extract features using \texttt{cmcl}
\item train models of band gaps measured at multiple fidelities
\item ascertain best model
\item analyze model accuracy, interpretability of bias and variance
\item make predictions on whole sample domain
\item Screen for viable candidate compositions
\end{itemize}
\end{frame}

\section{Credit}
\label{sec:orgdaee5ac}
\begin{frame}[label={sec:org41869d9}]{Acknowledgements}
\begin{columns}
\begin{column}{0.6\columnwidth}
\begin{block}{Research Group}
I am grateful to
\begin{itemize}
\item Professor Arun Mannodi-Kanakkithodi for his mentorship, support, and knowledge.
\item Professor Erk for her guidance and council
\item Professor Strachan for his instruction
\item Jiaqi Yang for his collaboration
\item Habibur Rahman for his friendship and curiosity
\end{itemize}
\end{block}
\begin{block}{Funding}
\begin{itemize}
\item Ross Fellowship
\item Professor Arun's startup research grant
\end{itemize}
\end{block}
\begin{block}{Resources}
RCAC Bell Computing Cluster
\end{block}
\end{column}
\begin{column}{0.4\columnwidth}
\hspace*{-1cm}
\includegraphics[width=200]{/home/panos/Pictures/mrg-photoshoot/41_Kanakkithodi.jpg}
\end{column}
\end{columns}
\end{frame}
\section{Bib}
\label{sec:orgca11aa3}
\begin{frame}[allowframebreaks]{References}
\AtNextBibliography{\tiny}
\printbibliography
\end{frame}
\appendix
\begin{frame}[label={sec:orgba6575d}]{Data Pre-Processing}
\begin{center}
\includegraphics[width=300]{/home/panos/Pictures/flowcharts/data_proc.png}
\end{center}
Data pre-processing Workflow to Implement with Python Pandas
\end{frame}
\begin{frame}[label={sec:org1e77e44}]{Machine Learning Pipeline}
\begin{center}
\includegraphics[width=300]{/home/panos/Pictures/flowcharts/ML_pipe.png}
\end{center}
Machine Learning Pipeline to Implement with Python SciKit-Learn
\end{frame}
\begin{frame}[label={sec:orgba2335f}]{DFT Functionals Effect Band Structure Calculation in HaPs}
\begin{itemize}
\item D3
\item SOC
\item PBE semi-local vs HSE hybrid
\begin{itemize}
\item (effects on electronic structure)
\item computational heft
\end{itemize}
\end{itemize}
\end{frame}
\end{document}