\chapter*{ADDITIONAL FIGURES}
\label{sec:org46e8685}
\section*{Learning Curves}
\label{sec:orgcb8f5b4}
Cross-validation within the training set is the only way of checking the generality of models during the grid search.
Identifying the validation split size is necessary to obtain an understanding of how much data is needed to train a model that can generalize.

Learning curves are computed for each scorer.
Notice that the error metrics are negated for consistency with the R\textsuperscript{2} and ev scores; the greater the number, the better the model performs.

More data offers better chances.
However, the smaller the split, the longer and more expensive the loop training becomes, e.g.
10-fold splits makes for 10 sample scores at each partition size.
Meaning, 90\% of the training set is used for actual training and the remaining 10\% is used for validation and this is repeated 10 times.

Shuffling is performed prior to generating each fold.
The shuffle is seeded with a deterministic random state to ensure scores are comparable across partition size 

\section*{Feature Distributions}
\label{sec:org23eca9b}
 
\begin{figure}[htbp]
\centering
\includesvg[width=450pt]{./.ob-jupyter/f724dca219c63ad7eaee9862431820dfe50fee59}
\caption{Normalized Distribution of A-site Constituents}
\end{figure}

 
\begin{figure}[htbp]
\centering
\includesvg[width=450pt]{./.ob-jupyter/a9b04c2294b41e26ac120135615a3356cf9d5bc1}
\caption{Normalized Distribution of B-site Constituents}
\end{figure}

 
\begin{figure}[htbp]
\centering
\includesvg[width=450pt]{./.ob-jupyter/bc4f79d7b43f705748b87f8b8bcc5111a778bed2}
\caption{Normalized Distribution of X-site Constituents}
\end{figure}

 
\begin{figure}[htbp]
\centering
\includesvg[width=450pt]{./.ob-jupyter/67142da5b3dcf51d03af83b5fe37bd3a4672b176}
\caption{Distributions of Mean A-Site Properties}
\end{figure}

 
\begin{figure}[htbp]
\centering
\includesvg[width=450pt]{./.ob-jupyter/3b53272412facd812fa6c52483d898ab67c2c602}
\caption{Distributions of Mean B-Site Properties}
\end{figure}

 
\begin{figure}[htbp]
\centering
\includesvg[width=450pt]{./.ob-jupyter/1473d3cf1b8588ff107fa23f1a91440e2716ebbd}
\caption{Distributions of Mean X-Site Properties}
\end{figure}

\section*{GPR SHAP analysis}
\label{sec:org3e59bec}
\begin{figure}[htbp]
\centering
\includegraphics[width=450pt]{/home/panos/Documents/manuscripts/DFT+ML+feature_engineering/GPR/.ob-jupyter/31552cf23170d5409f0d7f373221bd1784e2b209.png}
\caption{\label{fig:gprSHAP} Gaussian Process Regression Band Gap SHAP Values}
\end{figure}

\section*{SIS+RFR SHAP analysis}
\label{sec:org1d16891}

\begin{figure}[htbp]
\centering
\includegraphics[width=450pt]{/home/panos/Documents/manuscripts/DFT+ML+feature_engineering/RFR/.ob-jupyter/d0b6ba16e4913fe81324e8170b2b5b241c1053c8.png}
\caption{\label{fig:rfrSHAPe} Random Forest Regression Band Gap on SIS domain SHAP Values}
\end{figure}

\section*{SIS+GPR SHAP analysis}
\label{sec:orgec3eee2}

\begin{figure}[htbp]
\centering
\includegraphics[width=450pt]{/home/panos/Documents/manuscripts/DFT+ML+feature_engineering/GPR/.ob-jupyter/718541cd89c707a3acb068057f8bf25626eefd61.png}
\caption{\label{fig:gprSHAPe} Gaussian Process Regression Band Gap on SIS domain SHAP Values}
\end{figure}

\section*{Known Clustering in t-SNE Projections}
\label{sec:orgb40c09b}
 
\begin{figure}[htbp]
\centering
\includesvg[width=450pt]{./.ob-jupyter/0b728e519df7ec4be07158baa922f385bcac7076}
\caption{\label{fig:alloys} Projection of sample space via t-SNE overlaid with labels indicating site of mixing}
\end{figure}

 
\begin{figure}[htbp]
\centering
\includesvg[width=450pt]{./.ob-jupyter/c2d304e546a074ce9370ae449d18c778716a2de4}
\caption{\label{fig:chosen} Projection of sample space via t-SNE overlaid with labels indicating presense of data points in screened subset}
\end{figure}
