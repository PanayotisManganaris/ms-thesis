\ProvidesFile{thesis.tex}[2022-10-05 PurdueThesis thesis.tex file]

%
%  The home page for the PurdueThesis software is
%      https://engineering.purdue.edu/~mark/PurdueThesis/
%
%  Be sure to sign up for the PurdueThesis mailing list at
%      https://engineering.purdue.edu/ECN/mailman/listinfo/purduethesis-list
%  so you learn of new versions of this software.  You must be on that
%  mailing list to receive help with this software.
%
%  This is the template root file for an example thesis (for master's
%  degree) or dissertation (for a Ph.D.).  From now on "thesis" will
%  refer to both of these unless stated otherwise.
%
%  LaTeX systems include auxiliary programs to do bibliographies,
%  indexes, etc.  The latexmk program runs the fewest programs needed
%  to update your thesis.  latexmk runs automatically on Overleaf.  If
%  you're LaTeXing your document on a non-Overleaf system you may need
%  to run latexmk manually.
%
%  This thesis contains Feynman diagrams in the ap-physics.tex file.
%  For these to be processed correctly you must use the lualatex
%  program:
%      latexmk -lualatex thesis
%  (If your thesis doesn't have Feynman diagrams---the
%      \include{ap-physics}
%  command may be commented out by prefixing it with a
%  '%') use pdflatex instead of lualatex:
%      latexmk thesis
%
%  To make a final PDF file before you turn in your thesis do
%      latexmk -g thesis
%  This makes sure than everything is done for your final version.
%
%  References cited below:
%
%  TM2017 is short for Thesis Manual 2017:
%    A Manual for the Preparation of Graduate Theses,
%    eighth revised edition,
%    Thesis and Dissertation Office,
%    Purdue University,
%    2017,
%    revised August 30, 2017,
%    http://www.purdue.edu/gradschool/documents/thesis/graduate-thesis-manual.pdf,
%    last retrieved on May, 8, 2021.
%
%  In this file, change the example information to your information.
%

% institution
% Choose an institution name from the following list:
%     VALUE                   COMMENT
%     Purdue University
%     University of Hawaii    don't include the 'okina character here,
%                             it will get printed automatically
\def\ZZinstitution{Purdue University}


% campus
% Choose a campus from the following list:
%     VALUE               COMMENT
%     Bloomington
%     Fort\space Wayne
%     Hammond
%     Indianapolis
%     Manoa               don't include the macron character here,
%                         it will get printed automatically
%     Reno
%     Westville
%     West\space Lafayette
\def\ZZcampus{West\space Lafayette}


% program
% Choose a program from the following list:
%     VALUE
%     Aeronautics and Astronautics
%     Agricultural and Biological Engineering
%     Agricultural Economics
%     Agronomy
%     American Studies
%     Animal Sciences
%     Anthropology
%     Art and Design
%     Aviation and Transportation Technology
%     Basic Medical Sciences
%     Biochemistry
%     Biological Sciences
%     Biology
%     Biomedical Engineering
%     Botany and Plant Pathology
%     Chemical Engineering
%     Chemistry
%     Chemistry and Chemical Biology
%     Child Development and Family Studies
%     Civil and Mechanical Engineering
%     Civil Engineering
%     Communication
%     Communications
%     Comparitive Pathobiology
%     Computer and Information Science
%     Computer and Information Technology
%     Computer Graphics Technology
%     Computer Science
%     Construction Management Technology
%     Consumer Science
%     Curriculum and Instruction
%     Earth, Atmospheric, and Planetary Sciences
%     Economics
%     Educational Studies
%     Electrical and Computer Engineering
%     Engineering Education
%     English
%     Entomology
%     Food Science
%     Forensic and Investigative Sciences
%     Forestry and Natural Resources
%     Health and Kinesiology
%     Health Sciences
%     History
%     History and Philosphy
%     Horticulture
%     Hospitality and Tourism Management
%     Human Development and Family Studies
%     Industrial and Physical Pharmacy
%     Industrial Engineering
%     Interdisciplinary Biomedical Studies Program
%     Interdisciplinary Studies (Comparitive Literature)
%     Languages and Cultures
%     Linguistics
%     Management
%     Materials Engineering
%     Mathematical Sciences
%     Mathematics
%     Mechanical and Energy Engineering
%     Mechanical Engineering
%     Mechnical and Civil Engineering
%     Medicinal Chemistry and Molecular Pharmacology
%     Nuclear Engineering
%     Nursing
%     Nutrition Science
%     Organizational Behavior and Human Resource Management
%     Pharmacy Practice
%     Philosophy
%     Philosophy and Literature
%     Physics
%     Physics and Astronomy
%     Political Science
%     Psychological Sciences
%     Sociology
%     Speech, Language, and Hearing Sciences
%     Statistics
%     Technology
%     Technology Leadership and Innovation
%     Theatre
%     Veterinary Clinical Sciences
%     Youth Development and Agricultural Education
\def\ZZprogram{Electrical and Computer Engineering}

% degree
% Choose a degree from the following list:
%     Doctor of Audiology
%     Doctor of Nursing Practice
%     Doctor of Philosophy
%     Doctor of Technology
%     Master of Arts
%     Master of Fine Arts
%     Master of Library Science
%     Master of Science
%     Master of Science in Agriculture
%     Master of Science in Aviation and Aerospace Management
%     Master of Science in Aeronautics and Astronautics
%     Master of Science in Agricultural and Biological Engineering
%     Master of Science in Building Construction Management
%     Master of Science in Biomedical Engineering
%     Master of Science in Chemical Engineering
%     Master of Science in Civil Engineering
%     Master of Science in Electrical and Computer Engineering
%     Master of Science in Engineering
%     Master of Science in Education
%     Master of Science in Human Resources Management
%     Master of Science in Industrial Engineering
%     Master of Science in Industrial Technology
%     Master of Science in Mechanical Engineering
%     Master of Science in Materials Engineering
%     Master of Science in Nuclear Engineering
\def\ZZdegree{Doctor of Philosophy}

% author
% Put your name here.
\def\ZZauthor{Ima Student}

% document
% Choose a document from the following list:
%     A Dissertation
%     A Master's Bypass Report
%     A Preliminary Report
%     A Thesis
\def\ZZdocument{A Dissertation}

% graduation
% Chose a month from
%     May
%     August
%     December
% followed by a space
% then choose a year from 2020 to 2030.
\def\ZZgraduation{December 2022}

% title
% If you need to manually split the title,
% over several lines do, for example,
%     \def\ZZtitle{%
%       This is the First Line\\[-6pt]
%       and this is the Second Line%
%     }
\def\ZZtitle{This is the Title}


% showcolophon
% Print the ap-colophon.tex file at the end of the document?
% THE SUBMITTED COPY OF YOUR THESIS MUST BE RUN WITH ZZshowcolophon = {false}.
\def\ZZshowcolophon{false}

% showdiagonalline
% Show a diagonal line from lower left to center
% of main printed part of page?
% THE SUBMITTED COPY OF YOUR THESIS MUST BE RUN WITH ZZshowdiagonalline = {false}.
\def\ZZshowdiagonalline{false}

% showgridlines
% Show grid lines on main printed part of page
% Vertical and horizontal grid lines are put
% in the normal printed part of the page---this
% includes lines where the margins are.
% THE SUBMITTED COPY OF YOUR THESIS MUST BE RUN WITH ZZshowgridlines = {false}.
\def\ZZshowgridlines{false}

% showmarginlines
% Show margin lines on the edge of the normal printed part of the page?
% Margin lines show where the margins are.
% THE SUBMITTED COPY OF YOUR THESIS MUST BE RUN WITH ZZshowmarginlines = {false}.
%     VALUE    MEANING
%     false    don't show marginlines
%     true     show marginlines
\def\ZZshowmarginlines{false}

% showtimestamp
% Show, for example, a "compiled on  2021-03-02  Tuesday  17:16:24"
% timestamp in the upper right corner of page?
%     VALUE    MEANING
%     false    don't show timestamp
%     true     show timestamp
% THE SUBMITTED COPY OF YOUR THESIS MUST BE RUN WITH ZZshowtimestamp = {false}.
\def\ZZshowtimestamp{false}

% todonotes
% Set things up for todonotes.
%     VALUE    MEANING
%     false    don't put todo notes in PDF file
%     true     put 0.8 inch wide todo notes in PDF file
%     wide     put 3.8 inch wide todo notes in PDF file, do not send
%              todonotes = wide output to a printer
% THE SUBMITTED COPY OF YOUR THESIS MUST BE RUN WITH todonotes = {false}.
\def\ZZtodonotes{false}

% Mark Senn uses an `optional-debugging-code.tex file' but does not
% distribute it.  The following line won't do anything if you don't
% have an optional-debugging-code.tex file so you can leave it the
% way it is.
\InputIfFileExists{optional-debugging-code.tex}{}{}

% The \includeonly command can be used to only include some
% files that have \include commands below.  This is handy
% to only include some files so your document will LaTeX
% faster or if you're trying to narrow down where an error
% occurs.  You can use
%     \includeonly{ch-introduction}
% to only include ch-introduction.tex, or
%     \includeonly{ch-introduction,ap-about-appendices}
% to include ch-introduction.tex and ap-about-appendices.tex.
% More files can be added---just put ',' between the names.
% Comment out the following line before submitting the
% final copy of your thesis.
% \includeonly{ch-introduction,ap-about-appendices}

\documentclass{PurdueThesis}


%%%% \ExplSyntaxOn                         %%%% changed 2021-07-27 by mark
%%%% \bool_set_true:N \ZZCenterCaptionB    %%%% changed 2021-07-27 by mark
%%%% \ExplSyntaxOff                        %%%% changed 2021-07-27 by mark

\def\ZZatinformation{}
% If you are at the Hammond or Westville campus
% remove the "%" from the begining of the next line.
%\def\ZZatinformation{~at~Purdue~Northwest}

% If the title contains commas, do, for example,
% \def\ZZtitle{WIRELESS POWER TRANSFER:
% EFFICIENCY, FAR FIELD, DIRECTIVITY, AND PHASED ARRAY ANTENNAS}



% PurdueThesis.cls loads the rotating package which loads the graphicx
% package.  From page 12 of "Packages in the `graphics' bundle", 2021-03-05,
% retrieved 2021-06-16, at https://texdoc.org/serve/grfguide.pdf/0
%     \graphicspath{<dir-list>}
%
%         This optional declaration may be used to specify a list of
%         directories in which to search for graphics files.  The
%         format is the same as for the LaTeX 2e primitive \input@path.
%         A list of directories, each in a {} group (even if there is
%         only one in the list).  For example:
%             \graphicspath{{eps/}{tiff/}}
%         would cause the system to look in the subdirectories eps and
%         tiff of the current directory.  (All modern TeX systems use /
%         as the directory separator, even on Windows.)
%
%         The default setting of this path is \input@path that is:
%         graphics files will be found whereever TeX files are found.
%
% Look in the "graphics" subfolder for graphics files.
% This is done to reduce the number of files in the main thesis folder
% so the ones in there are easier to find.
\graphicspath{{graphics/}}

% Look in the "packages" subfolder for packages.
% This is done to reduce the number of files in the main thesis folder
% so the ones in there are easier to find.
\makeatletter
  \def\input@path{{packages/}}
\makeatother

%
% Configure bibliography.
%
% Automatically configure the bibliography.  Based on the
% institution, campus, and program listed in the \documentclass
% command \ZZBibProcessor is set to "BibLaTeX" or "BibTeX".
% For BibLaTeX, a
%    \usepackage[...]{biblatex}
% is done.  Put your bibliography entries in all-biblatex.bib.
% For BibTeX, a
%     \bibliographystyle{...}
% command is done.  Put your bibliography entries in all-bibtex.bib.
%
% All combinations of institution, campus, and program use BibLaTeX.
% Exceptions that use BibTeX:
%     o  "Purdue University", "West Lafayette", "Earth, Atmospheric,
%        and Planetary Sciences" uses the ametsoc2014 bibliography style.
%     o  "Purdue University", "West Lafayette", "Veterinary Clinical
%        Sciences" uses the ama bibliography style.
%
% To override the default choices picked by \ConfigureBibliography, change,
% for example,
%     \ConfigureBibliography
% to
    % \ConfigureBibliography
    \newcommand{\ZZBibProcessor}{BibLaTeX}
    \usepackage[backend=biber, citestyle=apa, dashed=false, sortcites=true, style=apa]{biblatex}
    \addbibresource{all-biblatex.bib}

%
% This is only done if you are using BibLaTeX.
%
%
% If you don't want to ignore urldate fields,
% comment out (put "%" before) the next ten lines.
% \DeclareSourcemap
%   {
%     \maps[datatype=bibtex]
%     {
%       % Ignore "urldate = {...}" in .bib files.
%       % See the first complete example on page 201 of
%       %     https://mirrors.rit.edu/CTAN/macros/latex/contrib/biblatex/doc/biblatex.pdf
%       \map
%         {
%           \step[fieldset=urldate, null]
%         }
%         % Enter approximate (circa) dates using, for example,
%         % "year = c2020"  See
%         %     https://tex.stackexchange.com/questions/224617/what-is-the-correct-way-to-handle-approximate-dates-in-biblatex
%       \map[overwrite=false]
%         {
%           \step[fieldsource=year]
%           \step[fieldset=sortyear, origfieldval, final]
%           \step[fieldsource=sortyear, match={c}, replace={}]
%         }
%     }
%   }

% To let {\bfseries\scshape text} work as expected.
% See
%     https://tex.stackexchange.com/questions/27411/small-caps-and-bold-face
\usepackage{bold-extra}

% For chemical figures.
\usepackage{chemfig}

% For typesetting cryptography pseudocode, algorithms, and protocols.
% See
%     https://mirror.las.iastate.edu/tex-archive/macros/latex/contrib/cryptocode/cryptocode.pdf
\usepackage
[
  n,            % or lambda
  advantage,
  operators,
  sets,
  adversary,
  landau,
  probability,
  notions,
  logic,
  ff,
  mm,
  primitives,
  events,
  complexity,
  oracles,
  asymptotics,
  keys,
]
{cryptocode}

% Define
%    \VerbatimInput[options]{filename}
%    \begin{VerbatimOut}{filename} ... \end{VerbatimOut}.
\usepackage{fancyvrb}
  \DefineShortVerb{\|}  % so "|verbatim|" will be verbatim

% For \InpuutIfFileExists.
\usepackage{filehook}

% So "_" will work in URLs when using BibTeX.
\usepackage[T1]{fontenc}

% For nlui testing.
\usepackage{listings}

% For chemical equations.
% See
%     https://ctan.org/pkg/mhchem?lang=en
% From the "Package documentation" linked-to document
%     mhchem needs a couple of other packages.
%     For instance, expl3, amsmath and calc.
\usepackage[version=4]{mhchem}
  % If I'm loading the package to just define a few new commands I'll indent
  % two spaces right after loading the package and define the few new
  % commands here.  If I'm defining more than a few commands I usually do it
  % after loading all the packages.
  % Define "\nitrate" to be the chemical symbol for nitrate.
  \newcommand{\nitrate}{\ce{NO3{-}}}
  % Define "\pnitrate" (short for "parenthesized nitrate") to be the chemical
  % symbol for nitrate surrounded by parentheses.
  \newcommand{\pnitrate}{(\nitrate)}
  % "Define \vpnitrate" (short for "verbose parenthesized nitrate") to be
  % the word "nitrate" followed by a space followed by the chemical symbol
  % for nitrate with parentheses around it.
  \newcommand{\vpnitrate}{nitrate (\nitrate)}

% For
%     \cancel
%     \highlight
% See
%     http://ftp.math.purdue.edu/mirrors/ctan.org/macros/latex/contrib/siunitx/siunitx.pdf
% pages 11--12.
\usepackage{cancel}


% Redefine description, enumerate, and itemize lists.
% See
%     https://mirrors.concertpass.com/tex-archive/macros/latex/contrib/enumitem/enumitem.pdf
% \usepackage{enumitem}
% \setlist[itemize]{leftmargin=7pt,rightmargin=24pt}



% This gets rid of
%     [5] (./thesis.toc
%     ! Undefined control sequence.
%     \vbox_set:Nn ...box:D {\color_group_begin: #2\par
%                                                       \color_group_end: }
%     l.32 ...}Basic Circuit Components}{31}{section.67}
%                                                       %
%     ?
% and
%     [6]
%     ! Undefined control sequence.
%     \vbox_set:Nn ...box:D {\color_group_begin: #2\par
%                                                       \color_group_end: }
%     l.61 ...rline {P.1}Frenchspacing}{67}{section.445}
%                                                       %
%     ?
% errors.
% See
%     https://github.com/latex3/latex2e/issues/73
\usepackage{etoc}

% Define \setmaxprintline{number_of_columns}.
% \usepackage{hardwrap}

% For indexing.  Making an index is optional.
% Make these commands available:
%     COMMAND           DESCRIPTION
%     \index{string}    put "string" in index information
%     \makeindex        save information to make the index
%     \printindex       print the index
% See
%     https://ctan.org/pkg/makeidx?lang=en
% for more information.
\usepackage{makeidx}
  % By default \index ignores its argument.
  % This activates indexing.
  \makeindex
  % The "chapter name" for the index.
  \renewcommand{\indexname}{INDEX}

% The mathtools package
% (see http://mirror.utexas.edu/ctan/macros/latex/required/amsmath/amsmath.pdf)
% loads the amsmath package which defines the
%     align
%     align*
%     alignat
%     alignat*
%     equation
%     equation*
%     flalign
%     flalign*
%     gather
%     gather*
%     multitaper
%     multitaper*
%     split
% environments and extends amsmath by defining many other commands.
% See
%     https://ctan.org/pkg/amsmath
% for information about amsmath and
%     http://ctan.math.washington.edu/tex-archive/macros/latex/contrib/mathtools/mathtools.pdf
% for information about mathtools.
\usepackage{mathtools}

% Define \includemedia.
\usepackage{media9}

% Define \begin{multicols}{number_of_columns} ... \end{multicolumns}.
% Used in ap-text.tex.
\usepackage{multicol}

% Define \ditto.
\usepackage{pa-ditto}

% Define \FigureDash.
% \FigureDash is a dash the width of a digit in the current font.
\usepackage{pa-figure-dash}

% For PurdueThesis, PuTh, TeX, LaTeX, METAFONT, METAPOST, etc. related logos.
\usepackage{pa-logos}

% (Or maybe use isomath instead?  -mark  2021-06-20)
% Follow ISO 80000-2:2019
%     o   put e, i, j, and pi in upright font automatically
%     o   use, for example, "\di x" to get "\,mathrm{d}\/x"
% This loads
%     o   amsmath.sty (which is already loaded above)
%     o   mathtools.sty
%     o   upgreek.sty
% Load the package.
\usepackage{pa-mismath}
  % Tell mismath to put e, i, j, and pi in upright font automatically.
  \enumber
  \inumber
  \jnumber
  \pinumber
  % To typeset math italic e, i, j, and pi use
  %     \mathit e
  %     \mathit i
  %     \mathit j
  %     \itpi

% Define \MyRepeat{what}{repeat}.
% Do "what" "repeat" number of times.
\usepackage{pa-repeat}

% Define \FloatBarrier.
% \FloatBarrier process all unproccesed floats (tables, figures, etc.).
\usepackage{placeins}

% Define \hl.
% Undefine \st so soul will load without an error.
% I hope \st wasn't used for something important!
\let\st\relax
\usepackage{soul}

% Define \textcent.
\usepackage{textcomp}

% !!! This doesn't work yet, figure it out later.
% For \textprimstress.
% \usepackage{tipa}

% Needed for chapter "Graphics", section "TikZ and PGF".
\usepackage{tikz}
  % Needed to customize arrows.
  \usetikzlibrary{arrows.meta}
  % For electrical diagrams.
  % Uses the TikZ package.
  % The circuitikz name is short for "circuit TikZ".
  \usepackage{circuitikz}
  %
  \usepackage{menukeys}
  %
  % Needed for 3D TikZ stuff.
  \usetikzlibrary{3d}
  %
  % Needed for pa-typographic-conventions package.
  \usetikzlibrary{calc,shadows,shapes.misc,shapes.symbols}
  %
  % Needed for putting text along a path.
  \usetikzlibrary{decorations.text}
  %
  % Draw TikZ decorations.
  % Needed for at least the Kalman filter system model graphic.
  \usetikzlibrary{decorations.pathmorphing} % noisy shapes
  %
  % Fit shapes to coordinates.
  % Needed for at least the Kalman filter system model graphic.
  \usetikzlibrary{fit}
  %
  % Draw the background after the foreground.
  \usetikzlibrary{backgrounds}	% drawing the background after the foreground

% Needed for the Feynman diagram in ap-physics.tex.
% Tikz-feynman requires LuaLaTeX instead of pdflatex be run.
% LuaLaTeX screws up spacing in the list of figures so this
% is not loaded and LuaLaTeX should not be used.
\usepackage[compat=1.1.0]{tikz-feynman}

% The vertical space between a table heading and the table contents
% in a tabular environment.
\newcommand{\tabularspace}{\noalign{\vspace*{2pt}}}

% For \sfrac, used to do slanted fractions, similar to, e.g., 1/2,
% but 1 is small and high and 2 is small and low.
\usepackage{xfrac}


% Define \I.
% \I1 does \indent once, \I2 does \indent twice, etc.
\newcommand{\I}[1]{\MyRepeat{\indent}{#1}}

% Define \MyI.
% Typeset my input.
\long\def\MyI#1%
  {%
    {%
      \fontsize{8}{10}\tt
      \VerbatimInput
        [
          firstnumber = 1,
          numbers     = left,
          xleftmargin = 0.33in,
        ]%
        {#1}
    }%
  }

% Define \MyIO.
% Typeset my input and output.
% The input will all be on the same page.
% The output may be split over multiple pages.
\newcommand{\MyIO}
  {%
    \input{z.out}

    {%
      \fontsize{8}{10}\tt
      \VerbatimInput
        [
          firstnumber = 1,
          numbers     = left,
          xleftmargin = 0.33in,
        ]
        {z.out}
    }
    \FloatBarrier
  }

% Define \MyIOS.
% Typeset my input and output.
% The input may be split over multiple pages.
% The output may be split over multiple pages.
% This doesn't work right:
%     o  Putting a \vbox around the input and output
%        does not allow todoindex entries to be listed.
%     o  Using \vfilneg at beginning and end of definition
%        screws up vertical spacing.
% \newcommand{\MyIOS}
% {%
%   \input{z.out}
%
%   {%
%     \fontsize{8}{10}\tt
%     \VerbatimInput
%     [
%       firstnumber = 1,
%       numbers     = left,
%       xleftmargin = 0.33in,
%     ]{z.out}%
%   }
% }

% Define \MyIOT.
% Typset my input and output together on the same page.
% This doesn't work right:
%     o  Putting a \vbox around the input and output
%        does not allow todoindex entries to be listed.
%     o  Using \vfilneg at beginning and end of definition
%        screws up vertical spacing.
% \def\MyIOT
% {%
%   \vfilneg
%   % \vbox
%   {%
%     \input{z.out}%
%     \fontsize{8}{10}\tt
%     \VerbatimInput[
%       firstnumber = 1,
%       numbers     = left,
%       xleftmargin = 0.33in,
%     ]{z.out}%
%   }%
%   \FloatBarrier
%   \vfilneg
% }

% Define \NL (newline) so LaTeX goes to the next output line.
% Just doing \\ complains
%     ! LaTeX Error: There's no line here to end.
% \mbox{} is an empty math box.
\newcommand{\NL}{\mbox{}\\}

% Print a list of files used and their version numbers in the log file.
\listfiles


% \def\bibindent{0em}
% Customize the bibliography.
% \DefineBibliographyStrings{english}{
%   urlfrom = {URLFROM},
%   urlseen = {URLSEEN}
% }

% For typographical conventions stuff including
%     \Emph{...}
%     \First{...}
%     \Keys{...}
%     \Literal{...}
%     \Menu{...}
%     \Place{...}
%     \Shell{...}
% This must be after
%     \usepackage{tikz}
\usepackage{pa-typographic-conventions}


% For the \begin{example} ... \end{example} environment
% used in ap-linguistics.tex.
\usepackage{covington}
\usepackage{slgloss}

% "CTAN---Comprehensive" did not get hyphenated and extended
% into the right margin when using BibLaTeX and the apa style.
% These did not change it:
%     \hyphenation{Com-pre-hen-sive}
%     \hyphenation{CTAN---Com-pre-hen-sive}
% I changed    publisher = {CTAN---Comprehensive TeX Archive Network},
% to           publisher = {CTAN---Com\-pre\-hen\-sive TeX Archive Network},
% in my all-biblatex.bib file and it worked as expeceted.
% If you need to change the hyphenation points of a word in the text
% you can do, for example,
%     \hyphenation{ve-ry-od-dly-hy-phen-at-ed}


\begin{document}

\setcounter{tocdepth}{3}

\maketitle

% Define front matter
%     dedication
%     acknowledgments
%     preface
%     table of contents
%     list of tables
%     list of figures
%     list of symbols
%     abbreviations
%     nomenclature
%     glossary
%     abstract
\ProvidesFile{ch-front.tex}[2022-10-05 front matter chapter]
%
%  This is the ``front matter'' for the thesis.
%
%  REFERENCES
%
%    TCMOS17
%      The Chicago Manual of Style Online, 17th edition.
%      https://www.chicagomanualofstyle.org/home.html
%      retrieved on 2020-02-29
%
%    TEMPL
%      Thesis and Disertation Office Templates.
%      https://www.purdue.edu/gradschool/research/thesis/templates.html
%      retrieved on 2020-02-29
%
%    WNNCD
%    Webster's Ninth New Collegiate Dictionary.
%

%
%   Only Purdue University uses this page
%
%   Comment out \begin{statement} through \end{statement}
%   if you are not at Purdue University.
%
% Statement of Thesis/Dissertation Approval Page
% This page is REQUIRED.  The page should be numbered "2"
% and should NOT be listed in your TABLE OF CONTENTS.
\begin{statement}
  % Delete or add \entry commands as needed for all committe members.
  \entry{Dr.~Arun Mannodi-Kanakkithodi, Chair}{School of Materials Engineering}
  \entry{Dr.~Alejandro Strachan}{School of Materials Engineering}
  \entry{Dr.~Kendra Erk}{School of Materials Engineering}
  % There should be one \approvedby command containing the
  % "FORM 9 THESIS FORM HEAD NAME HERE" (from TEMPL, retrieved on 2020-03-01).
  \approvedby{Pending FORM 9 Approval}
\end{statement}

% Dedication page is optional.
% A name and often a message in tribute to a person or cause.
% References: WEB9 332.
\begin{dedication}
  To my family
\end{dedication}

% Acknowledgements page is optional but most theses include
% a brief statement of appreciation or recognition of special
% assistance.
\begin{acknowledgments}
  I am grateful for the guidance of my advisor, Dr.~Arun
  Mannodi-Kanakkithodi.
  The Ross Fellowship awarded to me by Purdue University Graduate
  School helped fund my work.
  The Rosen Center of Advanced Computing provided the computational
  resources needed to conduct simulations, store data, and train models.
\end{acknowledgments}

% The preface is optional.
% References: TCMOS17 1.49, WEB9 927.
\begin{preface}
  This is the preface.
\end{preface}

% The Table of Contents is required.
% The Table of Contents will be automatically created for you
% using information you supply in
%     \chapter
%     \section
%     \subsection
%     \subsubsection
%     commands.
\pdfbookmark{TABLE OF CONTENTS}{Contents}
\tableofcontents

% If your thesis has tables, a list of tables is required.
% The List of Tables will be automatically created for you using
% information you supply in
%     \begin{table} ... \end{table}
% environments.
\listoftables

% If your thesis has figures, a list of figures is required.
% The List of Figures will be automatically created for you using
% information you supply in
%     \begin{figure} ... \end{figure}
% environments.
\listoffigures

% If your thesis has listings, a list of listings is required.
% The List of Listings will be automatically created for you using
% information you supply in
%     \begin{ZZlisting} ... \end{ZZlisting}
% environments.
\ZZlistoflistings

% If your thesis has protocols, you may want to do a list of protocols.
% The List of Protocols will be automatically created for you using
% information you supply in
%     \begin{protocol} ... \end{protocol}
% environments.
\listofprotocols

% If your thesis has schemes, you may want to do a list of schemes.
% The List of Schemes will be automatically created for you using
% information you supply in
%     \begin{scheme} ... \end{scheme}
% environments.
\listofschemes

% List of Symbols is optional.
\begin{symbols}
  $m$& mass\cr
  $v$& velocity\cr
\end{symbols}

% List of Abbreviations is optional.
\begin{abbreviations}
  abbr& abbreviation\cr
  bcf& billion cubic feet\cr
  BMOC& big man on campus\cr
\end{abbreviations}

% Nomenclature is optional.
\begin{nomenclature}
  alanine& 2-Aminopropanoic acid\\
\end{nomenclature}

% Glossary is optional.
\begin{glossary}
  philtrum& the groove between the nose and upper lip\\
\end{glossary}

\begin{abstract}%
  A wide range of optoelectronic applications demand semiconductors optimized for purpose.
  My research focused on data-driven identification of Halide perovskite compositions for optimum photovoltaic absorption in solar cells.
  I identified mixtures of candidate constituents at the A, B or X sites of the perovskite unit cell which leveraged how mixed perovskite band gaps ``bow'' away from the linear interpolations predicted by Vegard's law of mixing to obtain a selection of stable perovskites with band gaps in the ideal range of 1 to 2.5 eV for visible light spectrum absorption.

  I trained machine learning models on previously reported datasets of halide perovskite band gaps based on first principles computations performed at different fidelities.
  The primary objective of these models was to predict the perovskite band gap using the composition and inherent elemental properties as descriptors, eventually leading to accurate prediction and screening across the much larger chemical space from which the data samples were drawn.

  I utilized a recently published density functional theory (DFT) dataset of more than 1300 perovskite band gaps from 4 different levels of theory, added to an experimental perovskite band gap dataset of \textasciitilde{}100 points, to train random forest regression (RFR), Gaussian process regression (GPR), and Sure Independence Screening and Sparsifying Operator (SISSO) regression models, with data fidelity added as one-hot encoded features.
  I found that RFR yields the best model with a band gap root mean square error of 0.12 on the total dataset and 0.15 on the experimental points.
  SISSO provided compound features and functions for direct prediction of band gap, but errors were larger than from RFR and GPR.
  Additional insights gained from Pearson correlation and Shapley additive explanation (SHAP) analysis of learned descriptors suggest the RFR models performed best because of (a) their focus on identifying and capturing relevant feature interactions and (b) their flexibility to represent nonlinear relationships between such interactions and the band gap.
  % and (c) their benefit as an ensemble model to improve predictions by reducing the variance of errors
  The best model was deployed for predicting experimental band gap of \textasciitilde{}40,000 hypothetical compounds, based on which we screened \textasciitilde{}3000 stable compounds with band gap predicted to be between 1 and 2.5 eV at experimental accuracy.
\end{abstract}


%
% Put chapter \include commands here.
%

% Introductions may precede the first chapters or major divisions of theses.
% Reference: TM2017, page 31.
\chapter{INTRODUCTION}
\label{sec:org2924b89}

Perovskites have historically been materials of great interest for a variety of optoelectronic applications with special interest in the past ten years (see figure \ref{fig:nrel}) in their potential as photovoltaic absorbers.
As absorbers for solar cells, they offer opportunities to reduce cost and environmental impact as well as increase performance.
\autocite{ansari-2018-front-oppor,yin-2015-halid-perov,manser-2016-intrig-optoel,brenner-2016-hybrid-organ}
A cubic phase perovskite unit cell with general formula \(\ce{ABX3}\) contains two cations A and B at the corners and body center, and an anion X at each of the face centers.
The symbolic 3D perovskite structure is a network of \(\ce{BX6}\) octahedra robustly held together by large A-site cations.
This unique structure means that perovskite properties are highly tunable by changing the size and number of A/B/X species, by manipulating relative octahedral arrangements, and by creating non-cubic and metastable phases.
Halide perovskites (HaP), as opposed to the oxide perovskites that have been well researched over the past century are so characterized because their X-site anions are halogens.
Their B-site cations may be divalent elements, and the A-site is occupied by large monovalent cations that are either inorganic elements or organic molecules.

The most commonly studied hybrid organic-inorganic HaPs, \(\ce{MAPbI3}\) and \(\ce{FAPbI3}\), have demonstrated large power conversion efficiency (PCE) values between 20\% and 25\% when used as absorbers in single- or multi-junction solar cells.
\autocite{cui-2019-planar-p,jeong-2020-stabl-perov}
This is a five-fold improvement over the efficiencies of the same compositions first reported in 2009 and demonstrates the most attractive feature of HaPs, their unique tunability.
A theoretical cubic perovskite structure is considered stable if the ionic radii of A, B, and X-site species satisfy the well-known tolerance (\(t\)) and octahedral (\(o\)) factors.
\autocite{bartel-2019-new-toler}
Even accounting for stability constraints, the chemical space of perovskites experiences combinatorial scaling with the number of candidate elements which could be incorporated at each site.
This poses a multidimensional optimization problem for which determining the optimal atomic fractions for a particular performance target requires acceleration.
My research focused on accelerating this search through the development and application of data-driven design methods in the composition space of halide perovskites.
The most recent work presented in this dissertation is being submitted for publication as \fullcite{manganaris-2023-multi-fidel}. 
Publications from earlier work leading to this dissertation include
\fullcite{yang-2023-high-throug},
and \fullcite{manganaris-2022-mrs-comput}.

 
\begin{figure}[htbp]
\centering
\includesvg[width=400pt]{./.ob-jupyter/f34ddd159d9c661cb7098eb9bfd43dcfa5ab8286}
\caption{\label{fig:nrel} Rapid rise in cumulative maximum of HaP PCEs}
\end{figure}

\section{Design Goals in Perovskite Absorbers}
\label{sec:orgbd6587f}
Perovskite properties may be tuned in various ways.
The introduction of dopants and defects \autocite{kim-2020-upper-limit,dahliah-2021-high-throug} and the mutation of their cubic structure \autocite{kar-2018-comput-screen,kim-2017-hybrid-organ} are each promising areas of design.
However, the work presented here focuses specifically on the identification of a reasonable number of candidate compositions for future laboratory trials.
The most promising HaP compositions for PV absorption explored to date usually contain a mix of MA, FA, and Cs at the A-site, primarily Pb at the B-site with minor fractions of other divalent cations such as Sn and Ge, and I or Br at the X-site often with little Cl.
Discovery of novel HaP compositions with attractive properties is on the rise as researchers expand the search into more complex alloys, novel A-site organic molecules, and substitutes for Pb at the B-site from Group IV, Group II, or transition elements.
\autocite{zhu-2019-struc-elect,banerjee-2019-rashb-trigg,ding-2019-cesium-decreas,greenland-2020-correl-phase}
Mixing at the A-site has been shown to improve formability \autocite{zhang-2019-perov-photov}, while B-site and X-site mixing can tune and optimize band gap and therefore the maximum optical wavelength absorbable by the semiconductor.
Band gap is the energy required to promote an electron from the valence band to the conduction band.
Absorption of visible light from the plentiful green to red portion of the spectrum, by the simple relation \(E_{bg} = E_\gamma = \frac{hc}{\lambda}\), corresponds to band gaps of 1-2 \unit{\electronvolt}.
The allure of A/B/X-site mixing, even the creation of high entropy perovskite alloys, is in the promise to obtain dramatically different properties than those of pure compositions.
Perovskite properties have demonstrated highly nonlinear responses to changes in composition.
It is hoped that exploiting this could lead to possibly eliminating toxic lead, reducing degradation under light exposure, and even improving resistance to adverse environmental conditions while also targeting specific optoelectronic performance markers.

\begin{figure}[htbp]
\centering
\includegraphics[width=250pt]{hybrid-HaP.png}
\caption{\(2\times{}2\times{}2\) \(\alpha\)-phase supercell with Methylammonium at the A-site}
\end{figure}

The chemical design space of HaPs is intractably large to effectively screen by physical laboratory methods.
The halide perovskite chemical space covered by this dataset was based on fourteen species commonly appearing in study of these materials.
The five constituents making up the A-site occupants include three inorganic and two organic cations, namely \(\ce{CH3NH3+}\) Methylammonium (MA) and \(\ce{CH(NH2)2+}\) Formamidinium (FA).
\autocite{yan-2016-defec-physic,dimesso-2016-inves-formam}
Six divalent metals represent the possible B-site occupants and three halogen anions make up the possible X-site occupants.
See table \ref{tbl:site_tbl}.
The total number of distinct compositions possible in a \(2\times{}2\times{}2\) supercell is over 207 million.
Of these compositions, 37695 contain mixing at only one site and ninety are pure having no mixing at any site.
I refer to the combination of these subsets as the "cardinal mixing set" and it equally represents each of the constituent species of interest.
See figures \ref{fig:domainstats} and \ref{fig:domainmix}.

First principles density functional theory (DFT) simulations have been systematically performed to study the optoelectronic properties of HaPs as a function of structure, composition, and defects.
Recently, DFT simulations have been reliably used to predict structural information, band gaps, optical absorption spectra, and defect formation energies of a variety of HaPs with reasonable accuracy.
\autocite{mannodi-kanakkithodi-2022-data-driven,yin-2015-halid-perov}
An examination of the HaP-related computational literature reveals that there have been numerous medium (\textasciitilde{}10\textsuperscript{2} data points) to large (\textasciitilde{}10\textsuperscript{3} or more data points) DFT datasets reported for HaPs.
\autocite{castelli-2014-bandg-calcul,park-2019-explor-new,kar-2018-comput-screen,pu-2021-screen-perov}
These have been successfully screened to identify promising materials with desired stability and formability as well as PV-suitable band gaps, among other properties.

A clear limitation of High-Throughput (HT) DFT driven screening is the computational expense of applying a suitably advanced level of theory across a very large number of materials.
This problem is typically addressed by coupling DFT computations with machine learning (ML) techniques.
Within the area of perovskites, there are many examples in the literature where DFT datasets and suitable atomic/structural/compositional descriptors have been used to train a variety ML-based predictive and classification models, leading to accelerated prediction of lattice constants, formation energies, band gaps, and other important properties.
\autocite{park-2019-explor-new,stanley-2020-machin-learn,lee-2021-discov-lead}
Such DFT-ML models, once rigorously trained and tested, are deployed for high-throughput screening across massive sample spaces of unknown perovskites.
\autocite{yang-2022-high-throug}

\begin{table}[htbp]
\caption{\label{tbl:site_tbl} \(\ce{ABX3}\) candidate species per site}
\centering
\begin{tabular}{l|llllll}
A-site & MA & FA & Cs & Rb & K & \\[0pt]
B-site & Pb & Sn & Ge & Ba & Sr & Ca\\[0pt]
X-site & I & Br & Cl &  &  & \\[0pt]
\end{tabular}
\end{table}

 
\begin{figure}[htbp]
\centering
\includesvg[inkscapeformat=png, inkscapedpi=300,width=300pt]{./.ob-jupyter/835ba724468229103da0fd534aadf2acc7bb610c}
\caption{\label{fig:domainstats} The cardinal mixing sample space contains equal fractions of each element}
\end{figure}

 
\begin{figure}[htbp]
\centering
\includesvg[inkscapeformat=png, inkscapedpi=300,width=300pt]{./.ob-jupyter/9f2a6b3e88bd87edad16a8293ec22e2f0ce74780}
\caption{\label{fig:domainmix} The cardinal mixing sample space contains mostly B-site mixed compounds}
\end{figure}

\section{Multi-fidelity Dataset}
\label{sec:orgc33343b}
For my work I used a large DFT dataset collected over the past three years by my advisor and fellow student Jiaqi Yang.
This dataset consists of approximately 1300 calculations based on approximately 500 chemically distinct, pseudo-cubic, halide perovskite alloys reported in the Mannodi research group's prior work.
\autocite{mannodi-kanakkithodi-2022-data-driven,yang-2023-high-throug}
There are more calculations than there are distinct compositions because each composition is simulated multiple times to obtain results of varying fidelity.
Also, it is supplemented by an additional \textasciitilde{}100 points of data aggregated from reputable sources by \textcite{almora-2020-devic-perfor}.
In total it is one of the largest first principles cubic halide perovskite datasets and represents years of work.
It provided an excellent foundation for my work.

The relatively large size of this dataset samples the space of all possible single-site mixed compositions with good coverage.
This enables the training of interpolative models in the HaP composition space promising lower and less frequent error than if lesser coverage were used.
In this dataset, all perovskite structures are cubic or pseudo-cubic, which aids my focus on investigating effects of composition and alloying on photovoltaic performance.

\section{Perovskite Formability}
\label{sec:org0cdae8d}
In order to simplify the search for viable candidates, some standard empirical rating of perovskite formability are employed.
It has been observed that the A-site member must be much larger than its counterpart at the B-site for a Perovskite to be stable.
B-site elements are usually large (e.g. Pb, Sn) in Halide Perovskites, which incidentally motivates the use of organic cations at A.
\autocite{kieslich-2015-exten-toler}
Various tolerance factors have been proposed in the literature to describe this constraint.
Each tolerance factor is expressed as a function of the ionic radii of the species at each site \(r_A\), \(r_B\), and \(r_X\).
The Goldschmidt tolerance is defined as expression \ref{eq:t}.
\autocite{yin-2015-halid-perov}
The octahedral Factor is defined as the simple ratio \ref{eq:o}.
A recent tolerance factor proposed by \textcite{bartel-2019-new-toler} is defined as expression \ref{eq:b}.

\begin{equation}
\label{eq:t}
t = \frac{r_A+r_X}{\sqrt{2}*(r_B+r_X)}
\end{equation}

\begin{equation}
\label{eq:o}
o=\frac{r_B}{r_X}
\end{equation}

\begin{equation}
\label{eq:b}
b = \frac{r_X}{r_B}-\left[ 1-\frac{\frac{r_A}{r_B}}{\ln(\frac{r_A}{r_B})} \right]
\end{equation}

The approximate definitions \(t \approx 1\), \(o \approx 0.67\), and \(b \underset{\sim}{<} 4\) quantify perovskite \(\alpha\)-phase stability.
\autocite{yin-2015-halid-perov,bartel-2019-new-toler}
These criteria are useful for efficiently evaluating if a proposed perovskite can be formed.
They supplemented the cuts on band gap we developed for identifying high-performing Perovskite absorbers.

\section{Thesis Overview}
\label{sec:org3722640}
In chapter \ref{sec:org44ac04c}, I review the work we did in \citetitle{yang-2023-high-throug}\autocite{yang-2023-high-throug} to create a multi-fidelity dataset of perovskite band gaps.
I then detail the methods I used for predicting band gaps and the models I obtained in chapter \ref{sec:org37dcca4}.
Results and a discussion of their significance is given in chapter \ref{sec:orgbfd2f81}.
The open-source tools I developed for this work have been made available to the broader materials science community.
The essential qualities of these tools have been detailed in Appendix \ref{sec:orgc7f868a}.

\chapter{DENSITY FUNCTIONAL THEORY SIMULATION}
\label{sec:org44ac04c}
All material properties are fundamentally a function of the motion of electrons and atomic nuclei.
These fundamental building blocks of matter can be modeled as point particles in a many bodied system governed by an equation of motion.
Due to the quantum nature of these particles, position and momentum are defined in terms of a complex-valued wave function that describes the probability of finding a particle at a point in space.
Therefore, the equation of motion is the Schrödinger equation \eqref{eq:scheq}, the solution of which yields the energy of a single configuration of particles.
For electronic and optical properties, the electron configuration is most significant, allowing for some simplification by using the Born-Oppenheimer approximation (BOA).

\begin{equation}
\label{eq:scheq}
\imath\hbar\frac{\partial}{\partial t}\Psi(R_{3N}, r_{3n}, t) = \mathcal{H}(R_{3N}, r_{3n}, P_{3N}, p_{3n})\Psi(R_{3N}, r_{3n}, t) \stackrel{BOA}{\implies} \hat{H}\psi(r_{3n}) = E\psi(r_{3n})
\end{equation}

The solution to the Schrödinger equation is unfortunately intractable for even tens of electrons due to the huge expense of integrating the \(3n\) dimensional electron wave function \(\psi(r_{3n})\).
Density functional theory is the leading method for tractably computing the energy of many electron interactions in quantum systems.
DFT is founded on the Hohenberg Kohn relation proving an electron configuration fully determines the potential energy due to the nuclear configuration.
To illustrate, the electron density at a position is the expectation value of the position of all electrons not at that position.
This density emerges when computing the expectation value of the potential energy of a single electron \(r_k\)

\begin{equation}
\label{eq:V_ext}
\braket{\psi|\hat{V}_{nuc}(r_k)|\psi} = \int \di^3r_k \hat{V}_{nuc}(r_k)\int \Pi_{i=1}^{k-1}\di^3r_i\Pi_{i=k+1}^{n}\di^3r_i|\psi({r_i})|^2 = \int \hat{V}_{nuc}(r_k)n(r_k)\di^3r = \hat{V}(n)
\end{equation}

due to the nuclei.
This expression is reversible, showing that for a given density there must be a unique \(\psi(n(r))\) expressed as a functional of the density.
With the establishment of this density functional, any property of interest may be found entirely in terms of the density.
The major benefit of this re-framing is that the density functional lives in 3-D space, whereas the wave function lives in 3n-D space.
By circumventing the wave function, the many-electron problem solving for the configurational energy

\begin{equation}
\label{eq:funco}
E = \braket{\psi|\hat{H}|\psi} = \braket{\psi(n)|\hat{T}(n) + \hat{U}(n) + \hat{V}(n)|\psi(n)} \implies E(n) = \hat{T}(n) + \hat{U}(n) + \hat{V}(n)
\end{equation}

can be effectively cast to a sum of single-electron problems with theoretically no compromise on accuracy.
Minimizing \(E(n(r))\) yields the ground state energy and ground state electron configuration.
The electron-electron interaction \(\hat{U}(n(r))\) lacks a perfect solution and requires approximation.
The system potential energy \(\hat{V}(n(r))\), as discussed, is uniquely defined in 3D space by the nuclei in the system.
The kinetic energy operator \(\hat{T}(n(r))\) in combination with the system energy is used to derive the Kohn-Sham equations \eqref{eq:kseq}.

\begin{equation}
\label{eq:kseq}
\left(-\frac{\hbar^2}{2m}\nabla+\hat{V}_s\right)\psi(r_k) = \epsilon_k\psi(r_k)
\end{equation}

The solution for these single particle wave functions yield the electron density \(n(r) = \sum_{i=1}^n|\psi_i(r)|^2\) necessary to solve \eqref{eq:funco}.
The single-particle potential \(\hat{V}_s\) is the sum of the system energy, the electron-electron Coulomb repulsion, and the so called "exchange correlation."
This latter functional is what makes a solution for \(\hat{U}\) so elusive.
The more physics accounted for in the approximations used for exchange correlation, the better the accuracy but the greater the cost.
So these approximations are ranked by "level of theory."
My work is necessary because perfect simulations do not exist due to the conceptual difficulty in efficiently modeling the exchange correlation.
For the remained of this chapter I present how DFT was used to create the multi-fidelity dataset that enabled my work.

All DFT computations were performed using VASP version 6.2
\autocite{kresse-1996-effic-iterat,kresse-1993-ab-initio}
employing projector-augmented-wave (PAW) pseudo-potentials.
\autocite{kresse-1996-effic-ab,kresse-1999-from-ultras,kresse-1994-norm-conser}
Multiple levels of theory (LoT) were used in most computations.
Each simulation was conducted on the same set of compositions allowing \emph{at most} single-site mixing of our 14 constituent candidates for 3 sites (table \ref{tbl:site_tbl}).
Each HaP composition was simulated in a \(2\times{}2\times{}2\) supercell, which allowed A and B-site mixing to be performed in discrete 1/8\textsuperscript{th} fractions of the total site occupancy, and X-site mixing in 1/24\textsuperscript{th} fractions, though for simplicity, we restricted X-site mixing to fractions of 3/24.
For simulating mixed perovskites, the special quasi-random structures (SQS) method was applied to build periodic structures that made the first nearest-neighbor shells as similar to the target random alloy as possible.
\autocite{jiang-2016-special-quasir}
The final tally of successfully converged calculations is listed in table \ref{tbl:LoTs}.
The Perdew-Burke-Ernzerhof (PBE) functional within the generalized gradient approximation (GGA) as well as the hybrid Heyd-Scuseria-Ernzerhof functional with parameters (\(\alpha=0.25\)) and (\(\omega=0.2\)) (HSE06) are used for exchange-correlation energy.
\autocite{perdew-1996-gener-gradien,heyd-2003-hybrid-funct}
The energy cutoff for the plane-wave basis was set to 500 \unit{\electronvolt}.
For all PBE geometry optimization calculations, the Brillouin zone was sampled using a \(6\times{}6\times{}6\) Monkhorst-Pack mesh for unit cells and a \(3\times{}3\times{}3\) for supercells.
Using the PBE optimized structure as input, the electronic band structure was calculated along high-symmetry k-points to obtain accurate band gaps, and the optical absorption spectrum is further calculated using the LOPTICS tag, setting the number of energy bands to 1000 for each structure.
\autocite{hinuma-2016-band-struc,ganose-2018-sumo}
For HSE calculations, geometry optimization was performed using only the Gamma point, and subsequent computations used a reduced \(2\times{}2\times{}2\) Monkhorst-Pack mesh.
The force convergence threshold is set to be -0.05 eV/Å.
Spin-orbit coupling (SOC) is also applied to two flavors of HSE computations using the LORBIT tag and the non-collinear magnetic version of VASP 6.2.
\autocite{steiner-2016-calcul-magnet}
Optical absorption spectra from different HSE functionals were obtained by using the difference between the respective PBE and HSE band gap, and shifting the PBE-computed spectrum.

 
\begin{table}[htbp]
\caption{\label{tbl:LoTs} Sample counts by density functional represented in dataset}
\centering
\begin{tabular}{lr}
 & LoT\\[0pt]
\hline
PBE & 492\\[0pt]
HSE & 297\\[0pt]
HSE(SOC) & 282\\[0pt]
HSE-PBErel(SOC) & 244\\[0pt]
EXP & 90\\[0pt]
\hline
 & 1405\\[0pt]
\end{tabular}
\end{table}

\section{DFT Computed Band Gaps}
\label{sec:org7daa035}
Four types of electronic band gaps were computed by my advisor and group members using a \(2\times{}2\times{}2\) Monkhorst-Pack mesh.
These four measures E\textsubscript{bg}\textsuperscript{PBE}, E\textsubscript{bg}\textsuperscript{HSE}, E\textsubscript{bg}\textsuperscript{HSE(SOC)}, and E\textsubscript{bg}\textsuperscript{HSE-PBErel(SOC)} populated the multi-fidelity dataset I used for my work.
I aimed to accurately predict the band gaps of entirely hypothetical HaP compounds at the experimental fidelity using multi-fidelity models. 
This is motivated by the fact that the absorption spectrum of a perovskite determines its performance in a photovoltaic.
\autocite{mannodi-kanakkithodi-2019-compr-comput}
A well defined relationship therefore exists between the efficiency of a perovskite absorber and its true band gap.
\autocite{yu-2012-ident-poten}
The actual band gap (measured at the experimental fidelity) thus strongly predicts photovoltaic performance, and analysis of this relation informs the screening criterion.
My work aims to accelerate the design of high-performing photovoltaic devices by enabling more accurate rapid identification of candidate perovskites with desirable band gap properties.

\section{Spectroscopic Limited Maximum Efficiency (SLME)}
\label{sec:org67dad11}
Introduced by \textcite{yu-2012-ident-poten}, the SLME is a convenient metric for evaluating a semiconductor's suitability for single junction photovoltaic (PV) absorption.
In this work, SLME was calculated considering a 5\unit{\micro\meter} sample thickness for every perovskite using equations \ref{eq:absorption_alpha}, \ref{eq:slme_int}, and \ref{eq:slme_sum}, combining the original SL3ME.py code from \textcite{yu-2012-ident-poten} with our DFT computed absorption spectra and band gaps.

\begin{equation}
\label{eq:absorption_alpha}
a(E)=1-e^{-2\alpha(E)L}
\end{equation}

Here, \(\alpha(E)\) is the DFT computed optical absorption coefficient
as a function of incident photon energy and \(L\) is the thickness of
the absorber.

\begin{equation}
\label{eq:slme_int}
J=e\int_{0}^{\infty} a(E)I_{sun}(E)dE - J_{0}(1-e^{\frac{eV}{kT}})
\end{equation}

\begin{equation}
\label{eq:slme_sum}
\eta = \frac{P_{m}}{P_{in}}=\frac{\max(J \times V)}{P_{in}}
\end{equation}

To calculate SLME efficiency the current density \(J\), the light spectrum intensity of sunlight \(I_{sun}\), and the power \(P\) are all that is needed.
Using the DFT computed optical absorption spectrum as well as the magnitude and type (direct or indirect) of band gap as input, SLME is directly calculated using an open-source package.
\autocite{williams-2022-sl3me}
This calculation was performed using all four functionals and compliments the PCE measurements at the experimental fidelity.
SLME accounts for more energetic processes than the Shockley-Queisser criterion (\(bg \approx 1.3\)) allowing for a range of performant band gaps to be identified according to level of theory.
\autocite[p.1]{yu-2012-ident-poten}
Experimental data \autocite{almora-2020-devic-perfor} broadly agrees with PBE simulation, so the range of 1 to 2 \unit{\electronvolt} was justified (see figure \ref{fig:slme}).
Also, notice that even in just the sample dataset, there were candidates with potential to overtake the state of the art absorbers reported by NREL in figure \ref{fig:nrel}.
This is propitious for the screening I conducted on the 40000 point sample space.

 
\begin{figure}[htbp]
\centering
\includesvg[width=400pt]{./.ob-jupyter/1111b77f7e7ed1fa50791a404ec9652e766a082b}
\caption{\label{fig:slme} PBE SLME of sample compares to experimental PCE and cleanly demarcates competitive range of band gaps}
\end{figure}

\section{Improving Property Predictions using HSE06 and Spin-Orbit Coupling}
\label{sec:org417a4e0}
For a set of selected HaP compositions, while PBE-optimized lattice constants match well with experiments, PBE band gaps are underestimated, and HSE-PBE-SOC band gaps match better with measured values.
GGA-PBE computations reliably compute relaxed structures of both hybrid and purely inorganic HaPs.
However, advanced levels of theory such as the HSE06 functional with and without the inclusion of spin orbit coupling (SOC) to account for the relativistic effects of heavy atoms such as Pb, are of paramount importance when it comes to simulating electronic and optical properties.

The data set I used contains a series of \textasciitilde{}300 expensive HSE calculations across the 500 sampled compositions.
These are intended to yield insight into the effects of full geometry optimization at hybrid levels of theory to those of PBE-optimized structures.
Also, the effect of incorporating SOC in the calculation was examined.
In review, the sample of 500 band gaps available for training predictors was supplemented by 299 calculations conducted entirely at the HSE level of theory.
Furthermore, an additional 282 calculations were performed with HSE in addition to SOC, and 244 calculations were performed by running HSE(SOC) electronic structure calculations on PBE-relaxed structures.

The range of band gaps sampled by each simulation method are similar and are characterized by similar variance.
The descriptive statistics of each greatly exceeded those of the experimental subset (see figure \ref{fig:bg_dist}).
Nevertheless, the latter undoubtedly represented the smallest error from truth.
The types of mixing per level of theory were apportioned as in figure \ref{fig:lot_mix_org}.
This is the primary challenge I addressed with the multi-fidelity models discussed in chapter 3.

 
\begin{figure}[htbp]
\centering
\includesvg[width=300pt]{./.ob-jupyter/68775ba7ee8917dc6b42530cc8c59f2ed31c967d}
\caption{\label{fig:bg_dist} Variability in sampled band gaps at each fidelity}
\end{figure}

It is important to have a notion of which simulation is most accurate to the experimental measurements.
Figure \ref{fig:expqual} compares the band gaps obtained for a small subset of elements at all five levels of theory.
Theoretically, each functional may be more accurate for certain types of compositions.
For instance, organic-inorganic perovskites might benefit from greater account of Van der Waals forces and Pb-based compounds benefit from the use of spin orbit coupling as opposed to Pb-free compounds.
Note, phase information was not always available for certain experimental data points collected from the literature, and the inclusion of non-cubic phases in the tables may affect the evaluation of the functionals' accuracy.
Also, experimental data is tightly concentrated on the narrow range of performant band gaps likely due to selection bias.

The analysis is summarized in table \ref{tbl:expquant}.
HSE band gaps are heavily overestimated, but may be brought down by the addition of the SOC term.
Overall, HSE-PBErel(SOC) is the best approach for simulating band gaps with respect to computational cost and time.
PBE root mean square error (RMSE) is not significantly different from the HSE-PBErel(SOC) RMSE.
This is due to the accidental accuracy of semi-local functionals without SOC for hybrid organic-inorganic perovskites.
\autocite{mannodi-kanakkithodi-2019-compr-comput,mannodi-kanakkithodi-2022-data-driven}

 
\begin{figure}[htbp]
\centering
\includesvg[width=450pt]{./.ob-jupyter/087a658e6bb000a199d1fd4552bd9bde2db00444}
\caption{\label{fig:expqual} Effect of level of theory on band gap measurement}
\end{figure}

 
\begin{table}[htbp]
\caption{\label{tbl:expquant} RMSE values of band gaps computed from different functionals compared with experimental (Exp) values}
\centering
\begin{tabular}{lr}
 & RMSE vs EXP\\[0pt]
\hline
PBE & 0.55\\[0pt]
HSE & 0.87\\[0pt]
HSE(SOC) & 0.61\\[0pt]
HSE-PBErel(SOC) & 0.44\\[0pt]
\end{tabular}
\end{table}

\section{Sampling the Halide Perovskite Chemical Space}
\label{sec:orgcf27dc8}
Pure (non-alloyed) possibilities were exhaustively sampled using \(5*6*3 = 90\) compounds.
Starting from these pure perovskite structures systematic mixing was performed at the A, B, and X sites.
Figure \ref{fig:lot_mix_org} shows the shares of different types of mixing in our sample.
Again, for simplicity, only cardinal mixing was considered in this study: that is, mixing is not performed at multiple A/B/X-sites simultaneously.
The sample contains a reasonable balance of points representing each one of the cardinal mixing categories.
This helped to ensure the ML algorithms learn relationships between fidelities, not differences in mix site or constituency distributions within each fidelity.
Additionally, within each mix both purely inorganic samples and hybrid organic-inorganic samples were represented equally.

See the coverage of this sample in figure \ref{fig:coverage}.

 
\begin{figure}[htbp]
\centering
\includesvg[inkscapeformat=png, inkscapedpi=300,width=450pt]{./.ob-jupyter/201e8a95043e8f43d649342b896cab0dfa4ef32e}
\caption{\label{fig:lot_mix_org} Share by count of total data apportioned from each experimental subcategory}
\end{figure}

Most importantly, this sample gave even coverage of the cardinal mixing domain as shown in figure \ref{fig:coverage}.
The clusters in this figure were determined using the t-distributed stochastic neighbor embedding (t-SNE) method.
This is a nonparametric dimensionality reduction intended for visualizing statistically relevant clusters in a high dimensional dataset in only two or three dimensions.
In this case, the clusters correspond to the mix site of the member data points.
This sample provides the opportunity to comfortably interpolate the properties of other members of the cardinal mixing domain.
See Discussion.

 
\begin{figure}[htbp]
\centering
\includesvg[width=450pt]{./.ob-jupyter/8fb481208df6e81245038b7eaa4a261669d515f5}
\caption{\label{fig:coverage} Samples overlaid on cardinal mixing chemical domain projected from fourteen to two dimensions via t-SNE}
\end{figure}

Our multi-fidelity computational halide perovskite alloy dataset generated with the methods described here is one of the most comprehensive to date.
It is publicly available in the hopes further physical and engineering insights can be extracted by the broader research community.
It serves as the foundation for the modeling work presented in the following chapter.


\include{ch-do-not-use-these-packages}

% Summary and/or conclusions are optional but often used.
% The summary and/or conclusions often are the last
% the last major division(s) of the text.
% Reference: TM2017 page 32.
\include{ch-summary}

% Recommendations are optional.
% You may include recommendations as a major division if your
% subject matter and research dictate.
% Reference: TM2017 page 32.
\include{ch-recommendations}

% Test \begin{refsection}...\end{refsection}.
\include{ch-test}

% \immediate\setlength{\bibhang}{-3in}
% \immediate\setlength{\itemindent}{3in}
% \immediate\setlength{\rightmargin}{3in}

%
% This is only done if you are using BibLaTeX.
%
\makeatletter  % commented out on 2022-01-26
  \defbibenvironment{bibliography}
    {%
      \list
        {%
          \printtext[labelnumberwidth]%
          {%
            \printfield{prefixnumber}%
            \printfield{labelnumber}%
          }%
        }%
        {%
          \setlength{\bibhang}{1in} %%%%% was 0pt
          \setlength{\itemindent}{1in}%  -\leftmargin} %%%%% was 0pt
          \setlength{\itemsep}{\bibitemsep}%
          \setlength{\leftmargin}{0pt}%  .22in} % 0.42in}
          \setlength{\parsep}{\bibparsep}%
           \setlength{\rightmargin}{0.33in}%
        }%
    }
    {\endlist}
    {\item}
\makeatother  % commented out on 2022-01-26

% \immediate\setlength{\labelnumberwidth}{1.5in} %%%%% was commented out
\setlength{\labelwidth}{1.5in}
\def\sllnsez{[999] }

{%
  % Make _ in URLs visible.
  % \def\t{\char'137}%
  \catcode`*=\active
  \def*{\char'137}%  \char'137 is _
  \PrintBibliography
}

% Appendices are optional.  Not all theses contain appendices.
% An appendix is used for supplementary illustrative material,
% original data, computer programs, and other material that is not
% necessarily appropriate for inclusion within the text of your
% thesis.
% Reference: TM2017 page 33.
%
% Use ``\appendix'' for one appendix or ``\appendices'' for more than
% one appendix.
\appendices

% My filename conventions:
%     FILE THAT START WITH    ARE
%     ap-                     appendices
%     ch-                     chapters
%     gr-                     graphics
%     pa-                     packages
%     z                       temporary files

  % "About Appendices" appendix.
  \include{ap-about-appendices}

  % "Bugs" appendix.
  \include{ap-bugs}

  % Check margins.
  \include{ap-check-margins}

  % Demonstrate how to do separate appendices per chapter.
  \include{ap-chapter-appendices}

  % Demonstrate how to do separate references per chapter.
  % \include{ap-chapter-references}

  % Citations and references.
  \include{ap-citations-references}

  % Common mistakes.
  \include{ap-common-mistakes}

  % Defining commands.
  \include{ap-defining-commands}

  % Figures.
  \include{ap-figures}

  % Frequently Asked Questions.
  \include{ap-frequently-asked-questions}

  % Graphics.
  \include{ap-graphics}

  % Ignore these references.
  \include{ap-ignore-these-references}

  % Logos.
  \include{ap-logos}

  % Miscellaneous.
  \include{ap-miscellaneous}
  
  % Numbers and Units.
  \include{ap-numbers-and-units}

  % Resources.
  \include{ap-resources}

  % Tables.
  \include{ap-tables}

  % Special characters.
  \include{ap-special-characters}

  % Testing.
  \include{ap-testing}

  % Text.
  \include{ap-text}

  % Video.
  % \include{ap-video}

  % Astronomy.
  \include{ap-astronomy}

  % Biology.
  \include{ap-biology}

  % Chemistry.
  \include{ap-chemistry}

  % Computer Science.
  \include{ap-computer-science}

  % Electrical Engineering.
  \include{ap-electrical-engineering}

  % Linguistics.
  \include{ap-linguistics}

  % Mathematics.
  \include{ap-mathematics}

  % Music.
  \include{ap-music}

  % The examples in ap-physics require LuaLaTeX but LuaLaTeX
  % screws up the spacing in the List of Figures.  So, the
  % ap-physics file is not included.
  %
  % For some reason, ap-physics doesn't work when using BibTeX.
  % Just enclosing \include{ap-physics} in braces, i..e.,
  %     {
  %       \include{ap-physics}
  %     }
  % doesn't help so it is only loaded if we are using BibLaTeX.
  %
  % Physics-related exmples.
  % \include{ap-physics}

  % Notes and footnotes are optional.
  % Reference: TM2017 page 34.
  % I have not implemented this yet.  Mark Senn 2002-06-03

  % A vita is optional for masters theses
  % and required for doctoral dissertations.
  % Reference: TM2017 page 13.
  \include{ap-vita}

  % Listing or including publications(s) is optional.
  \chapter*{PUBLICATIONS}
\label{sec:org6abab80}

\nocite{manganaris-2022-mrs-comput}
\nocite{yang-2023-high-throug}
\nocite{manganaris-2023-multi-fidel}
\nocite{gollapalli-2023-graph-neural}
\nocite{edlabadkar-2023-drivin-halid}

\printbibliography[heading=none,category=myarticles]

\chapter*{PRESENTATIONS}
\label{sec:org001d17d}
\begin{itemize}
\item Poster for DS02 symposium MRS fall 2022
\item Talk for Purdue Soft Materials symposium
\item Developed MRS spring 2022 tutorial hosted on nanoHUB\autocite{manganaris-2022-mrs-comput}
\end{itemize}


  % Print the index.
  % The index is optional.
  \pdfbookmark{INDEX}{index}
  \printindex

  % If \ZZshowcolophon is true, print the colophon.
  \pdfbookmark{COLOPHON}{colophon}
  \ifthen{\equal{true}{\ZZshowcolophon}}
    {\include{ap-colophon}}

% LaTeX won't read after the \end{document} command.
% You can put notes to yourself or LaTeX input not
% ready for use after "\end{document}" if you'd like.
\end{document}
